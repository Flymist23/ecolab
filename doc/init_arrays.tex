\section{init\_arrays module}\label{init_arrays}

\begin{verbatim}
void initialize(array& data, char *tclname)
void initialize_offdiag(sparse_mat& interact, char *nm, int size)
\end{verbatim}

This module provides routines for initialising the model state
variables. {\tt initalize} provides a means of initialising an array
(or iarray) using a TCL handle {\tt tclname}. The Ecolab model arrays
are defined by the following code:
\begin{verbatim}
density = iarray(sum(nsp));
initialize( density,"density");
repro_rate = array(density.size);
initialize( repro_rate,"repro_rate");
mutation = array(density.size);
initialize( mutation,"mutation");
interaction = sparse_mat(density.size);
initialize( interaction.diag, "interaction(diag)" );
\end{verbatim}

Variables of type
\verb|array| or \verb|iarray| may be initialised by assigning the
following array elements:
\begin{description}
\item[list] The value of the array element is either a value, which is
copied into each element of the variable, or it is a Tcl list of
exactly \nsp\ elements, once repetitions have been expanded.
Repetition can be expressed by the syntax $n$\verb|*|{\em tcl-list},
where {\em tcl-list} may be a single value, or a list of values which
is repeated. One may also specify a range by means of the syntax {\em
low}:{\em high}[:{\em step}], where the step parameter may be
ommitted.
\item[random] if this array element is true, then the arrays values
are assigned randomly. Sub indices of this are \verb|seed|,
\verb|maxval| and \verb|minval|. The random values are allocated
uniformly in the range \verb|minval|\ldots\verb|maxval|.
\end{description}

Examples of the preceding (assuming \nsp=5, and the Ecolab model):

\begin{tabular}{lcl}
\verb|set density(list) 10| &$\Longrightarrow$& \verb|density| =
\{10,10,10,10,10\}\\
\verb|set density(list) {1,2*{2,3}}| &$\Longrightarrow$& \verb|density| =
\{1,2,3,2,3\}\\
\verb|set repro_rate(list) .1:.5:.1| &$\Longrightarrow$& \verb|repro_rate|
= \{.1,.2,.3,.4,.5\}\\
\verb|set repro_rate(random) true|\\
\verb|set repro_rate(random,seed) 10|&$\Longrightarrow$&
        \verb|repro_rate| is randomly assigned\\
\verb|set repro_rate(random,maxval) .1| & & values in the range $[-.05,.1]$\\
\verb|set repro_rate(random,minval) -.05| 
\end{tabular}


{\tt initialize\_offdiag} initializes the offdiagonal components of a
{\tt sparse\_mat}. The size parameter is the dimension of the matrix.
The diagonal component of a {\tt sparse\_mat} is just an array, so can
be initialized by {\tt initialize}. In terms of Ecolab, the call is:
\begin{verbatim}
initialize_offdiag( interaction, "interaction(offdiag)", density.size);
\end{verbatim}

The offdiagonal elements of a {\tt sparse\_mat} are set by either the
\verb|random| means above, viz.
\begin{verbatim}
set interaction(offdiag,random) true
set interaction(offdiag,random,minval) -0.01
set interaction(offdiag,random,maxval) 0.01
set interaction(offdiag,random,connectivity) 3
set interaction(offdiag,random,seed) 12
\end{verbatim}
where \verb|connectivity| (default \nsp/2) is the average number of
non-zero elements in each row, or by means of the \verb|list|
mechanism. In this case three lists need to be set, \verb|row|, {\tt
col} which contain the indices of the non-zero elements and
\verb|val|, which contains the interaction strength. So the standard
predator-prey equation may be specified by:
\begin{verbatim}
set repro_rate(list) {.1 -.1}
set interaction(diag,list) {-.0001 0}
set interaction(offdiag,val) {-0.001 0.001}
set interaction(offdiag,row) {0 1}
set interaction(offdiag,col) {1 0}
\end{verbatim}
