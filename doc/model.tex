\section{The \protect\EcoLab{} Model}\label{model}

The \EcoLab{} model is but one model implemented using the \EcoLab{}
software. This section documents the model itself, and may be skipped
if your intention is to use \EcoLab{} for other models.

We start with a generalised form of the Lotka-Volterra equation 
\begin{equation}\label{lotka-volterra}
\dot{\bn} = \br*\bn + \bn*\bbeta\bn.
\end{equation}
Here \bn\ is the population density, the component $n_i$ being the
number of individuals of species $i$, \br\ is the difference
between reproduction and death, \bbeta\ is the interaction matrix,
with $\beta_{ij}$ being the interaction between species $i$ and $j$, *
referring to elementwise multiplication and {\tt mutate} is the
mutation operator.

\subsection{Lotka-Volterra Dynamics}

The most obvious thing about equation (\ref{lotka-volterra}) is its
fixed point 
\begin{equation}\label{fixed point}
\hat{\bn} = -\bbeta^{-1}\br,
\end{equation}
where $\dot{\bn}=0$. For this point to be biologically meaningful, all
components of $\hat{\bn}$ must be positive, giving rise to the following
inequalities:
\begin{equation}\label{positive species}
\hat n_i = \left(\bbeta^{-1}\br\right)_i>0, \forall i
\end{equation}
The stability of this point is related to the
negative definiteness of derivative of $\dot{\bn}$ at $\hat{\bn}$. The
components of the derivative are given by
\begin{equation}\label{derivative}
\frac{\partial\dot{n}_i}{\partial n_j} =
\delta_{ij}\left(r_i+\sum_k\beta_{ik}n_k\right) + \beta_{ij}n_i
\end{equation}
Substituting eq (\ref{fixed point}) gives
\begin{equation}
\left.\frac{\partial\dot{n}_i}{\partial n_j}\right|_{\hat{\bn}}=
-\beta_{ij}\left(\bbeta^{-1}\br\right)_i
\end{equation}

Stability of the fixed point requires that this matrix should be
negative definite. Since the $\left(\bbeta^{-1}\br\right)_i$ are
all negative by virtue of (\ref{positive species}), each minor
determinant of this matrix is equal to a minor determinant of \bbeta\
multiplied by a positive number, stability of the equilibrium is
equivalent to \bbeta\ being negative definite.

A weaker condition is to require that the system remain bounded with
time:
\begin{equation}\label{boundedness}
\sum_i\dot{n_i}=\br\cdot\bn + \bn\cdot\bbeta\bn < 0, \;\forall \bn:
\sum_in_i>N \;\exists N
\end{equation}

As \bn\ becomes large in any direction, this functional is dominated
by the quadratic term, so this implies that  $\bn\cdot\bbeta\bn\leq0
\; \forall\bn: n_i>0$. Negative definiteness of \bbeta\ is sufficient,
but not necessary for this condition. For example, the predator-prey
relations (heavily normalised) have the following matrix as \bbeta:
\begin{math}
\bbeta=\left(\begin{array}{cc}
-1 & 2\\
-2 & 0\\
\end{array}\right)
\end{math}
which has eigenvalues $3/2, -5/2$. If we let $\bn=(x,y), x,y\geq0$, then
$\bn\cdot\bbeta\bn=-2x^2$, which is clearly non-positive
for all $x$.

Consider adding a new row and column to \bbeta. What is condition is
the new row and column required to satisfy such that equation
(\ref{boundedness}) is satisfied. Break up \bbeta\ in the following
way:
\begin{displaymath}
\left(
  \mbox{
     \begin{tabular}{c|c}
       $\begin{array}{ccc}\ddots\\&{\bf A}\\&&\ddots\end{array}$ & 
       $\begin{array}{c}\vdots\\{\bf B}\\\vdots\end{array}$ \\
       \hline
       $\begin{array}{ccc}\cdots&{\bf C}&\cdots\end{array}$ & D
     \end{tabular}
   }
\right)
\left(
  \mbox{
     \begin{tabular}{c}
     $\begin{array}{c}\vdots\\{\bf n_1}\\\vdots\end{array}$\\
     \hline
     $n_2$
     \end{tabular}
    }
\right)
\end{displaymath}

Condition (\ref{boundedness}) becomes:
\begin{equation}\label{boundedness2}
{\bf n_1}\cdot{\bf A}{\bf n_1} + {\bf n_1}\cdot({\bf B}+{\bf C})n_2 +
Dn_2^2 \leq 0
\end{equation}

Let 
\begin{displaymath}
a=\max_{n=1} \bn\cdot A\bn,\mbox{ and } b=\max_{i}B_i+C_i.
\end{displaymath}
 Then a sufficient but
not necessary condition for condition (\ref{boundedness2}) is
\begin{displaymath}
an_1^2+bn_1n_2+Dn_2^2\leq0
\end{displaymath}

The maximum value with respect to $n_2$ is $an_1^2-(bn_1)^2/4D$, so
this requires that
\begin{equation}\label{boundedness3}
b \geq 2\sqrt{aD}
\end{equation}

\subsection{Mutation}\label{mutation}

With mutation, equation (\ref{lotka-volterra}) reads
\begin{equation}
\dot{\bn} = \br*\bn + \bn*\bbeta\bn + {\tt mutate}(\bmu,\br,\bn).
\end{equation}


The difficulty with adding mutation to this model is how to define the
mapping between genotype space and phenotype space, or in other words,
what defines the {\em embryology}. A few studies, including Ray's
Tierra world, do this with an explicit mapping from the genotype to to
some particular organism property (e.g. interpreted as machine language
instructions, or as weight in a neural net). These organisms then
interact with one another to determine the population dynamics. In
this model, however, we are doing away with the organismal layer, and
so an explicit embryology is impossible. The only possibility left is
to use a statistical model of embryology. The mapping between
genotype space and the population parameters $\br$,
$\bbeta{}$  is expected to look like a rugged
landscape, however, if two genotypes are close together (in a Hamming
sense) then one might expect that the phenotypes are likely to be
similar, as would the population parameters. This I call {\em random
embryology with locality}.

In the simple case of point mutations, the probability $P(x)$ of any
child lying distance $x$ in genotype space from its parent follows a
Poisson distribution. Random embryology with locality implies that the
phenotypic parameters are distributed randomly about the parent
species, with a standard deviation that depends monotonically on the
genotypic displacement. The simplest such model is to distribute the
phenotypic parameters in a Gaussian fashion about the parent's values,
with standard deviation proportional to the genotypic displacement.
This constant of proportionality can be conflated with the species'
intrinsic mutation rate, to give rise another phenotypic parameter
$\bmu$.  It is assumed that the probability of a mutation generating a
previously existing species is negligible, and can be ignored. We also
need another arbitrary parameter $\rho$, ``species radius'', or
\verb+ecolab.sp_sep+,\index{sp\_sep} which can be understood as the
minimum genotypic distance separating species, conflated with the same
constant of proportionality as $\bmu$.

In summary, the mutation algorithm is as follows:
\begin{enumerate}
\item The number of mutant species arising from species $i$ within a
timestep is $\mu_i\alpha_in_i/\rho$. This number is rounded
stochastically to the nearest integer, e.g. 0.25 is rounded up to 1
25\% of the time and down to 0 75\% of the time.

\item Roll a random number from a Poisson distribution
$e^{-x/\mu+\rho}$ to determine the standard deviation $\sigma$ of phenotypic
variation. 

\item Vary $\br$ according to a Gaussian distribution about the
  parents' values, with $\sigma\alpha_0$ as the standard deviation,
  where $\alpha_0$ is the range of values that $\br$ is initialised to,
  ie $\alpha_0$=\verb|ecolab.repro_max|\index{ecolab.repro\_max}$-$
  \verb|ecolab.repro_min|\index{ecolab.repro\_min}

\item The diagonal part of $\bbeta$ must be negative, so vary $\bbeta$
according to a log-normal distribution. This means that if the old
value is $\beta$, the new value becomes
$\beta'=-\exp(\log_e(\beta)+\sigma)$. These values cannot become
arbitrarily small, however, as this would imply that some species make
arbitrarily small demands on the environment, and will become infinite
in number. In \EcoLab{}, the diagonal interactions terms are prevented from
becoming larger than $-r/(.1*{\tt INT\_MAX})$.

\item The offdiagonal components of $\bbeta$, are varied in a similar
fashion to $\br$. However new connections are added, or old ones
removed according to $\lfloor 1/r\rfloor$, where $r\in(-2,2)$ is
chosen from a stepped uniform distribution 
\begin{displaymath}
P(x)=\left\{
\begin{array}{ll}
0.25(1-g) & \mathrm{if}\; x\leq0\\
0.25(1+g) & \mathrm{if}\; x>0\\
\end{array}
\right.
\end{displaymath}
where $g\in[-1,1]$ (default of 0) is specified by the TCL variable
\verb+generalization_bias+\index{generalization\_bias}. The values on
the new connections are chosen from the same initial distribution that
the offdiagonal values where originally set with, ie the range
\verb|ecolab.odiag_min|\index{ecolab.odiag\_min} to
\verb|ecolab.odiag_max|\index{ecolab.odiag\_max}. Since $a$ in
condition (\ref{boundedness3}) is computationally expensive, we use a
slightly stronger criterion that is sufficient, computationally
tractable yet still allows ``interesting'' non-definite matrix
behaviour namely that the sum $\beta_{ij}+\beta_{ji}$ should be
nonpositive.


\item $\bmu$ must be positive, so should evolve according to the
  log-normal distribution like the diagonal components of $\bbeta$.
  Similar to $\bbeta$, it is a catastrophe to allow $\bmu$ to become
  arbitrarily large. In the real world, mutation normally exists at
  some fixed background rate --- species can reduce the level of
  mutation by improving their genetic repair algorithms. In \EcoLab{},
  this ceiling on $\bmu$ is given by the
  \verb|ecolab.mut_max|\index{ecolab.mut\_max} variable.

\end{enumerate}

\subsection{Input Parameters}\label{input parameters}

The model's parameters are set by TCL variables in {\tt model.tcl}.
The actual data structures of the model are initialised the first time
the model's generate step is called. An example input set is:
\begin{verbatim}
# initial condition
ecolab.species {1 2}
ecolab.density {100 100} 
ecolab.create {0 0}
ecolab.repro_rate {.1 -.1}
ecolab.interaction.diag {-.0001 -1e-5}
ecolab.interaction.val {-0.001 0.001}
ecolab.interaction.row {0 1}
ecolab.interaction.col {1 0}
ecolab.migration {.1 .1}


# mutation parameters
ecolab.mutation {.01 .01}
ecolab.sp_sep .1
ecolab.repro_min -.1
ecolab.repro_max .1
ecolab.odiag_min -1e-3
ecolab.odiag_max 1e-3
ecolab.mut_max .01
\end{verbatim}

Model variables define a TCL command of the same name as they appear
in the C++ source. So in the \EcoLab{} model, the C++ object {\tt
  ecolab} defines a set of TCL commands such as {\tt ecolab.density}
that can be used for setting or querying the values of {\tt ecolab}'s
members.  If an argument is specified, then that argument is used to
set the variable's value, otherwise, the variable's value is returned.
Array members in the model are initialised by specifying an TCL list
argument to the variables name, and return TCL lists when no argument
is specified. The above example starts the ecology off with a single
predator and prey (based on {\tt pred-prey.tcl}\index{pred-prey.tcl}).

\subsection{\protect\EcoLab{} Model commands}

\subsubsection{generate}\index{generate}

This implements the basic Lotka-Volterra equations:

\begin{math}
\dot{\bn} = \br*\bn + \bn*\bbeta\bn
\end{math}

with \br\ being the reproduction rate and \bbeta\ being the
interspecies interaction. This is implemented as a single line:

\begin{verbatim}
  density += repro_rate * density + (interaction * density) * density;
\end{verbatim}

This command also increments the timestep counter {\tt tstep}.

An optional argument specifies a number of timesteps to run the
generate step. This improves the speed by amortising the real to
integer conversion operation over a number of timesteps. The downside
is that computation may fail if the problem is ill-conditioned
(offdiagonal elements of \bbeta\ too large with respect to the
diagonal elements).

\subsubsection{condense}\label{condense}\index{condense}

This command compacts the systems of equations by removing extinct
species where $n_i=0$.

\subsubsection{mutate}\label{mutate}\index{mutate}

This applies the point mutations to the system. The precise algorithm
is described in \S\ref{mutation}. 

\subsubsection{migrate}\label{migrate}\index{migrate}

This operator implements migration within cellular \EcoLab{}. This
updates density values according to the difference with the 4 nearest
neighbours: $\bn+=\bgamma*(0.25(\bn_n+\bn_e+\bn_s+\bn_w)-\bn)$, where the
$n,e,s,w$ index the north, east, south and west neigbouring cells.

\subsubsection{maxeig}\label{maxeig}\index{maxeig}

\verb|maxeig| returns the maximum eigenvalue of \bbeta. If this number
is negative, the equilibrium point is stable, if positive, it is
unstable. As reported in Standish (1994)\cite{Standish94}, the mutation
drives the maximum eigenvalue slightly positive, then instabilities
act to push the eigenvalue back to zero. This command requires LAPACK\index{LAPACK}).

\subsubsection{lifetimes}\label{lifetimes}\index{lifetimes}

\verb|lifetimes| records the timestep when a
species passes a threshold (hardwired at 10) in the {\tt create}\index{create}
iarray. If a species has yet to pass the threshold, or has gone
extinct, the value in {\tt create} is zero. Upon return, this routine
returns the lifetime of the species that have gone extinct. This can
then be passed to a histogram routine, or written to a file.

\subsubsection{random\_interaction}\label{random_interaction}
\index{random\_interaction}

Calls \hyperref{{\protect\tt sparse\_mat::init\_rand()}}{(See \S}{)}{sparse_mat} to randomly initialising the nonzero pattern of
the offdiagonal elements. The average number of nonzeros per row is
{\tt conn}, and the standard deviation of the number of nonzeros is
{\tt sigma}.

\subsubsection{set\_grid}

Synopsis

{\tt ecolab.set\_grid} {\em x} {\em y}

\noindent Set up an $x\times y$ grid in spatial \EcoLab. See
\verb+ecolab_spatial.tcl+ for an example using this.

\subsubsection{get}

Synopsis

{\tt ecolab.get} {\em x} {\em y}

\noindent Create TCL method for accessing the internals of cell $x$
$y$. The new commands look like array elements, eg
\begin{verbatim}
ecolab(1,0).density
\end{verbatim}

\subsubsection{forall}

Synopsis

{\tt ecolab.forall ecolab.}{\em command} {\em args}

Run {\em command} on all cells.

\subsection{Spatial Variation}\label{spatial}

The ecolab model can run in multicellular mode by calling
\verb+ecolab.set_grid+ from TCL, specifying the dimensions of the
grid.\index{set\_grid}. See
\verb+ecolab_spatial.tcl+\index{ecolab\_spatial} for an example.

Only population density varies between the cells --- all other
variables are members of the ecolab variable so can be set or queried
in the usual way.

The usual ecolab model methods (generate, mutate, condense and
lifetimes)\index{generate}\index{mutate}\index{condense}\index{lifetimes}
can now be called, but operate on the entire grid. A new {\tt
migrate}\index{migrate} is defined to handle migration between
cells. You can also call a method of the ecolab cell on all cells
using the \verb+forall+\index{forall} command. For instance, to set
all cells to same initial density, use:
\begin{verbatim}
ecolab.forall ecolab.density [constant $nsp 100]
\end{verbatim}

Access to the individual cells can obtain by creating a TCL\_obj to
refer to it by using the \verb+ecolab.get+\index{get} method, which
creates commands like {\tt ecolab({\em x},{\em y})} to refer to the
cell. These can be fed to visualisers in the usual way.

\subsubsection{Parallel Execution}

Since the cells are pins in a \hyperref{graphcode Graph}{graphcode
  Graph (\S}{)}{graphcode}\index{graphcode}, they are distributed over
parallel processes if available.

Remember to call the \verb+gather+\index{gather} method to ensure node
0 is updated before running a visualiser on global data.

\begin{description}
\item[ecolab.gather] Bring processor 0's data up to date with the rest
  of the grid
\item[ecolab.distribute\_cells] Broadcast processor 0's data out to the
  rest of the grid.
\end{description}

\section{Palauan Jellyfish model}\label{jellyfish}

This is a model being developed by Mike Dawson and Russell Standish to
model the behaviour of jellyfish in a number of lakes on the island of
Palau. Jellyfish are photosynthetic animals, so have have a preference
for the sun and avoiding shadows. In this model, the jellyfish are
represented by agents that have a position and velocity. If the
jellyfish moves into shadow, or bumps into the side of the lake, it
will reverse its direction. From time to time, the jellyfish will
change direction and speed. The random generators governing these are
selectable at runtime through the experimental script. Also, jellyfish
do bump into each other. In the model display, a jellyfish will flash
green if it bumps into another one.

To run the jellyfish model, run the jellyfish.tcl script (located in
the models directory), specifying the lake as an argument, eg:
\begin{verbatim}
jellyfish.tcl lakes/OTM
\end{verbatim}

You can choose whether to compile the 2D version of the model or the
3D version, by (not) defining the preprocessor flag
\verb+THREE_D+\index{THREE\_D} (see Makefile).

The lake itself is represented by a Tk pixmap, with the blue component
representing water. The lake shapes were scanned into a GIF file, and
edited with a run-of-the-mill paint program to produce the lakes.

Visualising the lake involved creating a Tk canvas widget, displaying
the lake image in it, then overlaying it with shadow lines extending
from the pixels lying on the boundary, and finally representing the
jellyfish with arrow symbols (to indicate position and velocity).

This model illustrates the use of probes. Mouse clicks in the canvas
region are bound to a short method that determines which agent is
closest to the mouse position. It then colours that agent red (for
tracking purposes), creates a TCL\_obj representing that agent, and
returns the name back to TCL. TCL then calls the object browser
(\S\ref{object-browser}) on that TCL\_obj. In all, 14 lines of C++
code, and 3 lines of TCL code. The result is very effective.

The jellyfish model is written to be run in parallel using
\hyperref{Graphcode}{Graphcode
  (\S}{)}{graphcode}\index{graphcode}. The strategy is effectively a
{\em particle-in-cell} method. The lake is subdivided into a Cartesion
grid of cells, and each Jellyfish only needs to consult the cell that
it is in, as well as neighbouring cells to determine if it will
collide with any other jellyfish.

