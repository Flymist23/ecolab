\section{tcl++}\label{tcl++}\index{tcl++}

This section describes a light weight class library for creating
Tcl/Tk applications. \cite{Ousterhout94} It consists of a header file
\verb|tcl++.h|, and a main program \verb|tclmain.cc|. All you need to
do to create a Tcl/Tk application is write any application specific
Tcl commands using the {\tt NEWCMD}\index{NEWCMD}\label{NEWCMD} macro, link your code with the
\verb|tclmain| code, and then proceed to write the rest of your
application as a Tcl script. The \verb|tclmain| code uses argv[1] to
get the name of a script to execute, so assuming that the executable
image part of your application is called \verb|appl|, you would
start off your application script with
\begin{verbatim}
#!appl
...Your Tcl/Tk code goes here...
\end{verbatim}
then your script becomes the application, directly launchable from the
shell.

EcoLab processes any pending TCL events prior to executing the C++
implementation method. This keeps the GUI lively. However, it may
cause problems if TCL scripts depend on C++ state that may be changed
by event processing. There is a boolean global variable {\tt
  processEvents}\index{processEvents}, that disables the background
even processing if set to false. TCL commands {\tt
  enableEventProcessing}\index{enableEventProcessing} and {\tt
  disableEventProcessing}\index{disableEventProcessing} can be used to
set/clear this flag from the TCL interface.

\subsection{Global Variables}

The following global symbols are defined by \verb|tcl++|:
\begin{description}
\item[interp]\index{interp} The default interpreter used by Tcl/Tk
\item[mainWin]\index{mainWin} The window used by Tk, of type
  \verb|Tk_Window|, useful for obtaining X information such as
  colourmaps
\item[error]\index{error} To flag an error, throw a variable of type
  \verb+error+. Use \verb|error(char *format,...)| to construct such
  an object.
The format string is passed to sprintf, along with all remaining
arguments. The resulting string is what is return by the
  \verb+error::what()+ method. Standard execptions are usually caught,
  and the error message return by \verb+what()+ is attached to TCL's
  error log, and \verb+TCL_ERROR+ return back to the interpreter.
\end{description}

\subsection{Creating a Tcl command}
To register a procedure, use the \verb|NEWCMD| macro. It takes two
arguments, the first is the name of the Tcl command to be created, the
second is the number of arguments that the Tcl command takes.

Here is an example.
\begin{verbatim}
#include "tcl++.h"

NEWCMD(silly_cmd,1)
{printf("argv[1]=%s\n", argv[1]);}
\end{verbatim}
This creates a Tcl command that prints out its argument.

\subsection{tclvar}\index{tclvar}\label{tclvar}

Tcl variables can be accessed by means of the tclvar class. For
example, if the programmer declares:

\verb|tclvar hello("hello"); float floatvar;|

then the variable hello can be used just like a normal C variable in
expression such as 

\verb|floatvar=hello*3.4;|.

The class allows assignment to and from \verb|double| and \verb|char*|
variables, incrementing and decrementing the variables, compound
assignment and accessing array Tcl variables by means of the C array
syntax. The is also a function that tests for the existence of a Tcl
variable.

The complete class definition is given by:

\begin{verbatim}
class tclvar
{ 
 public:

/* constructors */
  tclvar();
  tclvar(char *nm, char* val=NULL);
  tclvar(tclvar&);
  ~tclvar();

/* These four statements allow tclvars to be freely mixed with arithmetic 
 expressions */
  double operator=(double x);
  char* operator=(char* x);
  operator double ();
  operator char* ();

  double operator++();
  double operator++(int);
  double operator--();
  double operator--(int); 
  double operator+=(double x);
  double operator-=(double x);
  double operator*=(double x);
  double operator/=(double x);

  tclvar operator=(tclvar x); 
/* arrays can be indexed either by integers, or by strings */
  tclvar operator[](int index);
  tclvar operator[](char* index);

  friend int exists(tclvar x);
};
\end{verbatim}

\subsection{tclcmd}\index{tclcmd}

A tclcmd allows Tcl commands to be executed by a simple \verb|x <<|
{\em Tcl command} syntax. The commands can be stacked, or
accumulated. Each time a linefeed is obtained, the command is
evaluated.

\pagebreak[4]
For example:
\begin{verbatim}
tclcmd cmd;

cmd << "set a 1\n puts stdout $a";
cmd << "for {set i 0} {$i < 10} {incr i}"
cmd << "{set a [expr $a+$i]; puts stdout $a}\n";
\end{verbatim}

The behaviour of this command is slightly different from the usual
stream classes (but identical to {\tt
  eco\_strstream}\index{eco\_strstream} from which it is derived, as a space is
automatically inserted between arguments of a \verb|<<|. This means
that arguments can be easily given - eg
\begin{verbatim}
cmd << "plot" << 1.2 << 3.4 << "\n";
\end{verbatim}
will perform the command {\tt plot 1.2 3.4}. In order to build up a
symbol from multiple arguments, use the append operator. So
\begin{verbatim}
(cmd << "plot"|3) << "\n";
\end{verbatim} 
executes the command {\tt plot3}.


\subsection{tclreturn}\index{tclreturn}

The {\tt tclreturn} class is used to supply a return value to a TCL
command. It inherits from {\tt eco\_strstream}\index{eco\_strstream},
so has exactly the same behaviour as that class. When this variable
goes out of scope, the contents of the stream buffer is written as the
TCL return value. It is an error to have more than one {\tt tclreturn}
in a TCL command, the result is undefined in this case.

\subsection{tclindex}\index{tclindex}

This class is used to index through a TCL array. TCL arrays can have
arbitrary strings for indices, and are not necessarily ordered. The
class has three main methods: {\tt start}, which takes a {\tt tclvar}
variable that refers to the array, and returns the first element in
the array; {\tt incr}, which returns the next element of the array;
and {\tt last} which returns 1 if the last element has been read. A
sample usage might be (for computing the product of all elements in
the array {\tt dims}):
\begin{verbatim}
tclvar dims="dims";
tclindex idx;
for (ncells *= (int)idx.start(dims); !idx.last(); ncells *= (int)idx.incr() );
\end{verbatim}

The class definition is given by:
\begin{verbatim}
class tclindex  
{
public:
  tclindex();
  tclindex(tclindex&);
  ~tclindex() {done();}
  tclvar start(tclvar&);
  inline tclvar incr();
  tclvar incr(tclvar& x) {incr();}  
  int last();
};
\end{verbatim}





