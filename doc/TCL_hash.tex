\section{TCL_hash class}

TCL_hash\index{TCL\_hash}\index{hash\_map}\index{eco\_hashmap.h} is an
STL hash_map replacement, based around TCL hash tables. It is not the
full implementation, but enought to be useful. To access it, either
include \verb+<TCL_hash.h>+ directly, or include
\verb+<eco_hashmap.h>+ with the preprocessor symbol
\verb+HASH_TCL_hash+ defined. If \verb+HASH_hash_map+ is defined then,
tradtional STL hash maps are used (if available), otherwise standard
C++ maps are employed.

The TCL hash\_map implementation is:
\begin{verbatim}
template <class key, class T>
struct hash_map
{
  class entry 
  {
    entry& operator=(const T& x);
    operator T();
  };

  struct iterator
  {
    entry operator*();
    iterator operator++();
    iterator operator++(int); 
    int operator==(iterator x);
    int operator!=(iterator x);
  };

  entry operator[](const key x);
  int count(const key x);
  void erase(const key x);
  void erase(iterator x);
  iterator begin(); 
  iterator end();
};
\end{verbatim}

A helper class is defined to address the different types of keys:
\index{hashtype}
\begin{verbatim}
template <class T>
struct hashtype
{ 
  int keytype;
  /* we assume TCL_ONE_WORD_KEYS=1 here */
  hashtype() {keytype = sizeof(T)/sizeof(int);}
  char * operator()(const T& x) {return (keytype>1)? (char*)&x: (char*)x;}
};
\end{verbatim}

In most instances, this is exactly what you want. \verb+keytype+ is one of
\verb+TCL_STRING_KEYS+ (if T=\verb+char *+), \verb+TCL_ONE_WORD_KEYS+
or a number giving the number of words occupied by the key.

If you have a class that needs to do something different, the
\verb+operator()+ member returns what needs to passed as the key to the
\verb+Tcl_FindHashEntry+ routine. 
