\section{Graph library}

The \EcoLab{} graph library is a library providing a simple and
lightweight structure for representing graphs (aka networks). A graph
consists of a set of nodes labelled $N=\{0\ldots n-1\}$, and a set of
edges $E\subset N\times N$, which have an optional weight factor
attached.

An edge is represented by\index{Edge}
\begin{verbatim}
struct Edge: public std::pair<unsigned,unsigned>
{
    unsigned source();
    unsigned target();
    float weight;
};
\end{verbatim}

The abstract Graph\index{Graph} interface has the following definition:
\begin{verbatim}
struct Graph
{
  struct const_iterator
  {
    Edge operator*() const;
    const Edge* operator->() const;
    const_iterator& operator++();
    bool operator==(const const_iterator& x) const;
    bool operator!=(const const_iterator& x) const;
  };
 
  virtual const_iterator begin() const;
  virtual const_iterator end() const;

  virtual unsigned nodes() const;
  virtual unsigned links() const;
  virtual void push_back(const Edge& e);
  virtual bool contains(const Edge& e) const;
  virtual bool directed() const;
  virtual void clear(unsigned nodes=0);
  const Graph& operator=(const Graph& x);
  void input(const std::string& format, const std::string& filename);
  void output(const std::string& format, const std::string& filename) const;
  template <class BG> Graph_back_insert_iterator<Graph,BG>
    back_inserter(const BG& bg);
}
\end{verbatim}

The \verb+begin/end+ methods allow one to iterate over the edges. Only a \verb+const_iterator+
is supplied, as it is an error to change the value of an edge. One can
only reset a graph to the empty graph via clear, and construct the
graph incrementally using the \verb+push_back()+ method.

The \verb+contains+ method allows one to test whether a given edge is
in the graph, and \verb+directed+ indicates whether the underlying graph
structure has directed edges or not. A bidirectional graph otherwise
appears as a directed graph where each edge appears twice, once for
each direction.

The input/output methods allow for the graph to be read/written
from/to a file, in a variety of formats, given by the format
parameter. Currently, the following formats are supported:
\begin{tabular}{|l|l|}
\hline
name & description\\
\hline
pajek\index{pajek} & Pajek's .net format\\
lgl\index{lgl} & LGL's .lgl format\\
dot\index{dot} & Graphviz\index{Graphviz} format\\
gengraph & Gengraph\index{Gengraph}\\
\hline
\end{tabular}
Also graphs can be streamed to/from standard I/O streams, and will appear
in dot format. In particular, this means that TCL scripts can access
Graph objects as strings containing the Graphviz representation. See
some of the examples in the \verb+models/netcomplexity_scripts+ directory.

\verb+back_inserter+ creates an output iterator suitable for use with
Boost Graph algorithms. As a simple example, to construct an \EcoLab{}
graph from a Boost Graph, do
\begin{verbatim}
  std::pair<BG::edge_iterator,BG::edge_iterator> r=edges(bg);
  ConcreteGraph<DiGraph> g1;
  copy(r.first, r.second, g1.back_inserter(bg));
\end{verbatim}


This interface can be used in both a dynamic polymorphism fashion (ie
Graph is an abstract base class) and in a static polymorphism fashion.

The \verb+graph.h+ header file provides two concrete graph types -
DiGraph\index{DiGraph} and BiDirectionalGraph\index{BiDirectionalGraph}, which differ just in whether each edge
is directed or not.

\subsection{Boost}

\EcoLab{} has a deliberate policy of not having a dependency on
Boost, but to be as compatible with Boost as far as it is
possible. The \EcoLab{} graph library has been designed to be
interoperable with the Boost Graph Library. The file
\verb+test/test_boostgraph.cc+ gives some examples of interoperation
with Boost.

\subsection{Graph Library Functions}

\verb+Degrees degrees(const G& g)+ returns array of node degrees
return as the following structure:
\begin{verbatim}
struct Degrees
{
  array<unsigned> in_degree, out_degree;
};
\end{verbatim}

\verb+void ErdosRenyi_generate(Graph& g, unsigned nodes, unsigned
links, urand& uni);+

Generate an Erd\"os-R\'enyi random graph with given nodes and
links. Links are attached randomly to the nodes, drawn from uni.

\verb+PreferentialAttach_generate(Graph& g, unsigned nodes, urand& uni,
                                 random_gen& indegree_dist=default_indegree)+

Barabasi-Albert Preferential attachment algorithm. For each node, an
indegree value is drawn from \verb+indegree_dist+ (defaults to a
constant value of 1). This many links are then preferentially attached
to other nodes, according to their outdegree. 

\subsection{Network complexity}

\begin{verbatim}
class BitRep implements Graph
{
  BiDirectionalBitRep symmetrise() const;
  bool operator()(i,j);
  bool next_perm();
};

class BiDirectionalBitRep implements Graph
{
  bool operator()(i,j);
  bool next_perm();
};
 
class NautyRep implements Graph
{
  bool operator()(i,j);
  double lnomega() const;
  NautyRep canonicalise() const;
};

\end{verbatim}
These classes implement the Graph interface by storing the linklist as
a bitset. NautyRep specifically uses the bitset representation of the
Nauty package. One can freely convert between these types and others
implementing the Graph interface. All of these types support direct
setting/testing of the i,j th edge through the operator(i,j).

BitRep and BiDirectionalBitRep allow one to iterate through the
permutations (thus cycling over all graph representations of a given
edge count). This function returns false when no further permutations exist.

NautyRep has member functions for calling the Nauty library. If you
need both the lnomega and canonical representations, then it is more
efficient to call them at the same time via the \verb+ecolab_nauty()+
function.

\begin{verbatim}
void ecolab_nauty(const NautyRep& g, NautyRep& canonical, double& lnomega, bool do_canonical);
\end{verbatim}
call Nauty on g, returning a canonical representation canonical and
lnomega ($\ln\Omega$). If canonical is not needed, then set \verb+do_canonical+ to
false.

\begin{verbatim}
double canonical(BitRep& canon, const Graph& g);
\end{verbatim}
SuperNOVA canonical algorithm. Returns $\ln\Omega$ and canonical
representation (which needn't correspond to that returned by Nauty).

\begin{verbatim}
double complexity(const Graph& g);
\end{verbatim}
Network complexity. If all links have the same weight, this
corresponds to $2\lceil\log_2(n+1)\rceil+\lceil\log_2(n(n-1))\rceil+1+
\log_2\left(\begin{array}{c}n(n-1)\\l\end{array}\right) - \log_2\Omega$.
