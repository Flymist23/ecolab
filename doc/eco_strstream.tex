\section{eco\_strstream class}

\subsection{eco\_string class}\index{eco\_string}

\begin{verbatim}
class eco_string
{
public:
  eco_string();
  eco_string(const eco_string& x);
  eco_string(const char *x);
  ~eco_string();

  eco_string& operator=(const eco_string& x);
  eco_string& operator=(const char* x);
  eco_string operator+(const char* y);
  eco_string& operator+=(const char* y); 
  eco_string operator+(const char y); 
  eco_string& operator+=(const char y); 
  eco_string operator+(const eco_string& y);
  eco_string& operator+=(const eco_string& y);
  operator char *() const;
  int length();
  void freeze(int x);
};
\end{verbatim}

Basic string class, broadly compatible with the STL string class, but
can be used in the classdesc\index{classdesc} generated code. The
pointer returned by {\tt char *} refers to an ordinary C string, but
is no longer valid once the destructor is called. This behaviour can
be altered calling {\tt freeze(1)}. Unlike the GNU string class,
casting the string to a {\tt char *} does not freeze the string.

\subsection{eco\_strstream}\index{eco\_strstream}

This is a modified version of ostrstream for \EcoLab{} use, with
slightly altered behaviour making it more suited for \EcoLab{} use.

In particular, \verb+operator<<+ add spaces in between its arguments,
and a new \verb+operator|+ is defined that is similar to ostrstream
\verb+operator<<+. This makes it easier to construct TCL commands.

Note that \verb+|+ has higher precedence than \verb+<<+, so when
mixing the two, ensure that no \verb+<<+ operator appears to the right
of \verb+|+, or use brackets to ensure the correct interpretaion:

eg.
\begin{verbatim}
    s << a << b | c;
\end{verbatim}
or
\begin{verbatim}
    (s | a) << b | c;
\end{verbatim}
but not
\begin{verbatim}
    s | a << b | c;
\end{verbatim}

In any case, you'll most likely get a compiler warning if you do the
wrong thing.

{\tt eco\_strstream} is derived from {\tt eco\_string} class, so has
all the string behaviour. The \verb+str()+ member is defiend as
equivalent to a \verb+(char*)+ cast for compatibility with ostrstream.
However, the contents of the \verb+char*+ are not frozen, unless
explicitly specified with freeze() (opposite to ostrstream behaviour).

\begin{verbatim}
class eco_strstream: public eco_string
{
  public:
    char* str() const;
    
    template<class T>
    eco_strstream& operator<<(const T x); 

    template<class T>
    eco_strstream& operator|(T);
}

inline ostream& operator<<(ostream& x, eco_strstream& y);
\end{verbatim}

