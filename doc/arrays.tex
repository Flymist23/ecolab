\section{arrays}

This section documents the array and related data types. Arrays are
dynamically sized (like std::vector), but are numerical, so can be
used in numerical array expressions like \verb"a+b=0;". Array
expressions use the expression template technique\cite{Veldhuizen95}
to maximise the performance of array expressions.

The code has been optimised for use with Intel's vectorising compiler,
and code will be generated using SSE instructions if compiled with icc.

Arrays and associated functions are defined in the
\verb+array_ns+\index{array\_ns} namespace. 


\subsection{array}\label{array}\index{array}\index{iarray}

\begin{verbatim}

//E is an array expression - eg array<double> or array<double>+array<int>
//S is a scalar data type - eg int
//T must be a C plain old data type, usually numerical

template <class T>
class array
{
 public:
  typedef T value_type;

  size_t size();   //return size of array
  T* data();       //return pointer to array's contents

  explicit array(size_t s=0); //construct array of size s
  array(size_t s, T val);     //construct and initialise array of size s
  void resize(size_t s);      //resize array, data undefined

  array& array(const E& x);   //copy constructor
  array& operator=(const E& x); //assignment or broadcast
  T& operator[](size_t i);    //reference an element
  T& operator[](const E& i);    //vector indexing

  operator+() etc.    //see below
};
\end{verbatim}

\subsubsection{Array operators}

Arithmetical operators \verb"+,-,*,/,%" are defined elementwise
between array expressions, as well as broadcastwise if one argument is
a scalar. For example if x and y and z are arrays, and a is a scalar,
\begin{verbatim}
  z=x+a*y    =>  for (i=0; i<idx.size(); i++) z[i]=x[i]+a*y[i];
\end{verbatim}

Indexing operators [] can either take a single integer argument, which
refers to a single array or expression element, or it can take an
integer array expression, which performs vector indexing. So 
\begin{verbatim}
y=x[idx];    =>  for (i=0; i<idx.size(); i++) y[i] = x[idx[i]];
y[idx]=x;    =>  for (i=0; i<idx.size(); i++) y[idx[i] = x[i];
\end{verbatim}

Comparison operators \verb+<+, \verb+>+ etc, and logical operators
\verb+&&+, \verb+||+ are also defined in elementwise and broadcast versions.

\verb+operator<<(expression1,expression2)+ is a concatenation
operator, appending the elements of expression2 to the end of the
elements of expression1. 

Compound assignment variants also exist:
\begin{verbatim}
x+=y;    =>   x=x+y;
x*=y;    =>   x=x*y;
   ...
x<<=y;   =>   x=x<<y;
\end{verbatim}

\subsection{{\tt sparse\_mat}}\label{sparse_mat}\index{sparse\_mat}

The {\tt sparse\_mat} data type is designed to hold the
interaction matrix \bbeta, which is generally a sparse matrix. It is
built up of array<double> and iarray<int> components:

\begin{verbatim}
class sparse_mat
{
 public:
  array diag, val;
  iarray row, col;
  sparse_mat(int s=0, int o=0)
    {diag=array(s); val=array(o); row=iarray(o); col=iarray(o);}
  array operator*(iarray& x);  /* matrix multiplication */
  sparse_mat operator=(sparse_mat x);
  void init_rand(int conn, double sigma);
};
\end{verbatim}

{\tt diag} stores the diagonal components of the array, {\tt val} is
the packed list of offdiagonal values, with {\tt row} and {\tt col}
being the index lists. The only important operator defined for this
class is the matrix operation, which is defined as
\begin{verbatim}
beta*x == beta.diag*x + (beta.val*x[beta.col])[beta.row]
\end{verbatim}
but is implemented separately for efficiency reasons.

{\tt init\_rand}\index{init\_rand} is a utility routine for randomly
initialising the nonzero pattern of the offdiagonal elements. The
average number of nonzeros per row is {\tt conn}, and the standard
deviation of the number of nonzeros is {\tt sigma}.

It is possible to represent the offdiagonal array differently  for
efficiency reasons. For example, if it is desired to represent the
offdiagonal elements as a dense 2D array, one can create an extra
pointer in the underlying implementation of array. Most of the time,
this pointer is NULL, but when the \verb|cs_arrays| routine
\verb|offmul| is called, it will check this pointer for the {\tt val}
array. If it is NULL, it will create the efficient
representation, otherwise it will reuse the existing one. Because the
only way the array's actual value will get out of synch with the
efficient representation is by index assignment, \verb|put_double()|
and \verb|put_double_array| (as well as obviously
\verb|delete_array()|) will need to be modified to deallocate the
efficient representation.

\subsection{Global functions}

\begin{itemize}
\item array {\tt merge}\index{merge}( iarray {\em mask}, array {\em
    a}, array {\em b} )
  
  return an array (or iarray) {\em r} such that where {\em
    mask[i]==1}, {\em r[i]=a[i]}, otherwise {\em r[i]=b[i]}.

\item array {\tt pack}\index{pack}( array {\em x}, iarray {\em mask}, [int {\em
ntrue}])

Construct a new array from elements of  {\em x} that correspond to
where the mask is true. {\em ntrue} is an optional parameter equal to
the number of true elements of {\em mask}. If not given, then {\tt
pack} will count the true elements of {\em mask}. Give this parameter
if you are using the same {\em mask} in multiple {\tt pack} statements.

\item iarray {\tt enumerate}\index{enumerate}(iarray {\em mask})

Return the running sum of mask.

\item iarray {\tt pcoord}\index{pcoord}( int {\em size})

return [0..size-1]

\item iarray {\tt gen\_index}\index{gen\_index}(iarray {\em x}) 

generate a list of index values, with each number appearing in the
list according to the value passed in its position. For example, if
{\em x}=\{0,0,1,2,0,1\} then the output from this function will be
\{3,4,4,6\}

\item void {\tt lgspread}\index{lgspread}( array\& {\em a}, array {\em s} )

Perform lognormal variation of the components of {\em a}, with
standard error given by {\em s}, ie
\begin{displaymath}
a_i=\mathrm{sgn}(a_i) \exp(\log|a_i| + s_i\xi_i)
\end{displaymath}
where $\xi_i$ is normal random variate.

\item void {\tt gspread}\index{lgspread}( array\& {\em a}, array {\em s} )

Perform normal variation of the components of {\em a}, with
standard error given by {\em s}, ie
\begin{displaymath}
a_i = a_i + s_i\xi_i
\end{displaymath}
where $\xi_i$ is normal random variate.



\end{itemize}

\subsubsection{Reduction functions}

\begin{itemize}
\item double {\tt max}\index{max}( array {\em x}) 
\item double {\tt min}\index{min}( array {\em x}) 
\item double {\tt sum}\index{sum}( array {\em x}) 
\item bool {\tt all}\index{all}( array {\em x}) 
  return true is all of x[i] are nonzero
\item bool {\tt any}\index{any}( array {\em x})
  return true if any x[i] are nonzero
\end{itemize}

\subsubsection{Array functions}
\begin{itemize}
\item array {\tt abs}\index{abs}(array {\em x})
\item iarray {\tt sign}\index{sign}(array {\em x})
\item array {\tt exp}\index{exp}(array {\em x})
\item array {\tt log}\index{log}(array {\em x})
\item array{\tt <int>} ProbRound\index{ProbRound}(array{\tt <double>} {\em x})
    Round up or down with probability given by the difference between
    real value and the integers either side. For instance, if the
    argument's value is 0.2, then it has an 80\% chance of being
    rounded down to 0, and a 20\% chance of being rounded up to 1.
    Uses \hyperref{{\tt array\_urand}}{{\tt
    array\_urand}\S}{)}{random fun}.
\end{itemize}

\subsubsection{Random number functions}\label{random fun}

array's random number functions allow arrays to be filled with random
numbers efficiently in a single call. Most of these functions use the
\verb+array_urand+\index{array\_urand} uniform random object (of type \hyperref{urand}{urand\S}{)}{random}\index{urand}), which is
accessible from TCL. The \verb+fillgrand()+ function makes use of the
TCL accessible \verb+array_grand+\index{array\_grand} object, which is
of type \hyperref{gaussrand}{gaussrand\S}{)}{random}\index{gaussrand}.


\begin{itemize}
\item void {\tt fillrand}\index{fillrand}(array {\em x}) 

Fill {\em x} with random numbers from the uniform distribution over $[0,1]$

\item void {\tt fillprand}\index{fillprand}(array {\em x})

Fill {\em x} with random numbers from the Poisson distribution $e^{-x}$

\item void {\tt fillgrand}\index{fillgrand}(array {\em x})

Fill {\em x} with random numbers from the Gaussian distribution $e^{-x^2/4}$

\item void {\tt fill\_unique\_rand}\index{fill\_unique\_rand}(array {\em
    x}, int {\em max})

Fill {\em x} with random integers from the range [0..{\em max}] such
that no two integers are the same.

\end{itemize}

\subsubsection{Stream Functions}
\begin{itemize}
\item \verb|ostream& operator<<(ostream& s, expr x)|
\item \verb|istream& operator>>(istream& s, array& x)|
\end{itemize}

\subsection{Old arrays}

The original array library had a somewhat different API. The new array
library can be accessed through the old interface by including
\verb+#include "oldarrays.h"+ The only slight difference that may be
noticed is that oldarray::size is not an integer, but an object that
converts to an integer. The problem is that overloaded functions may
ge confused.
