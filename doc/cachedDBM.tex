\section{cachedDBM}\label{cachedDBM}

{\tt cachedDBM} implements a notion of persistent objects. First and
foremostly, it has the syntax of std::map, ie it is a template map
object, with a key and value type pair. However, by calling the
\verb+init()+ method, you can attach a database file, so that values
saved in the cachedDBM are stored on disk, to be accessible at a later
time.

The iterator range \verb+begin()+ to \verb+end()+ refers to everything
stored in the database. An alternative interface that iterates over
the database keys is provided by \verb+keys.begin()+ and
\verb+keys.end()+ The database is committed when \verb+begin()+ is
called, unless the database was opened readonly. In this latter case,
there are potentially items stored in the map which will not be
iterated over. The alternative iteration
methods \verb+firstkey()+, \verb+nextkey()+ and \verb+eof()+ is an
older interface for iterating over \verb+keys.begin()+ to
\verb+keys.end()+.  In this instance, the cachedDBM is not committed
when \verb+firstkey()+ is called.

Elements stored in the cachedDBM are not actually written to disk until
\verb+commit()+ is called (or the cachedDBM object is
destroyed). 

Entries in the database can be removed via \verb+del()+. However, if
an item with the same key is in the cache, it will need to be removed via
\verb+erase()+ as well, otherwise it will be reinserted in the
database at commit time.

Only a very simple caching algorithm is employed, but it seems
sufficient for many purposes. If the member \verb+max_elem+ is set,
then this acts as an upper limit to the number of items stored in
memory. If you request a new item to be loaded via the [] operator,
and it will cause the number of items to be exceeded, the cachedDBM
object is committed. At least the oldest quarter of the cache is cleared,
and up to half of the cache. So if \verb+max_elem+=100, then one is
guaranteed that the previous 50 accessed objects will always be in
memory, so can be assigned to a reference.

Classdesc serialisation (XDR serialisation) is used store both keys
and value data. The database file are therefore machine independent.
As a special exception to serialisation rules, \verb+char*+ can be
used as key and data types.

\subsection{Synopsis}

\begin{verbatim}
template<class key, class val>
class cachedDBM_base : public base_map<key,val>
{
public:
  int max_elem;   /* limit number of elements to this value */
  void init(const char *fname, char mode='w'); /* open database file */
  void Init(TCL_args args); /* TCL access to init() */
  void close();    /* commit and close database file */
  bool opened();   /* is a database attached? */
  bool key_exists(const key& k); /* does the key exist in db */
  val& operator[] (const key& k); /* access element with key k */
  void commit();    /* write any changes out to the file, and clear cache */
  void del(key k); /* delete key from database (but not cache!) */    
  key firstkey();  /* return first key in database */
  key nextkey();   /* return next key in database in an iteration */ 
  int eof();       /* true if all keys have been accessed */
  KeyValueIterator begin() const;
  KeyValueIterator end() const;
  KeyIterator keys.begin() const;
  KeyIterator keys.end() const;
};
\end{verbatim}

\subsection{TCL access}

cachedDBM objects are addressable from TCL. Individual
objects stored in the cachedDBM are addressable from TCL by means of
the \verb+elem()+ method, but also the following members are
exported to the TCL interface, allowing scripting access to
manipulating the database:

\begin{itemize}
\item \verb+cachedDBM::max_elem+
\item \verb+cachedDBM::Init+ {\em filename} {\em r$|$w}
\item \verb+cachedDBM::close+
\item \verb+cachedDBM::opened+
\item \verb+cachedDBM::elem+ With one argument, get the element value,
  with two arguments, set it.
\item \verb+cachedDBM::commit+
\item \verb+cachedDBM::firstkey+
\item \verb+cachedDBM::nextkey+
\item \verb+cachedDBM::eof+
\end{itemize}

\subsection{Types of database}

By default, if EcoLab's Makefile detects the presence of Berkeley DB on
your system, it will be used. Otherwise it use the ndbm API. If
libgdbm and/or libgdbm\_compat (the NDBM compatibility layer in GDBM)
these will be added to the linker flags, otherwise it will assume that
ndbm is available as part of the standard system library.

You can't mix and match database types. If you have some data stored
in one type, and need to access it using a different database type,
the utilities convtoNDBM and convtoBDB found in the util directory may
be useful.
