\section{Graphcode}\label{graphcode}
\psection{Graph}

A {\em Graph}\index{Graph} is a container of references to
\hyperref{{\em objects}}{ (\S}{)}{object}\index{object}
(called \hyperref{{\em objrefs}}{
    (\S}{)}{objref})\index{objref} that may be linked to an arbitrary
  number of other objects. The objects themselves may be located on
  other processors, ie the Graph may be distributed. Objects are
  polymorphic --- the only properties Graph needs to know is how
  create, copy, and serialise them, as well as what other objects they
  are linked to.

Because the objects are polymorphic, it is possible to create
hypergraphs. Simply have two types of object in the graph --- {\em
  pins} and {\em wires}, say. A pin may be connected to multiple wire
objects, just as wires may be connected to multiple pins.

The objrefs themselves are stored in a maplike object called an
\hyperref{{\em omap}}{ (\S}{)}{omap}, which is replicated
across all processors. 

A short synopsis of Graph is as follows:
\begin{verbatim}
class Graph: public Ptrlist
{
public:
  omap objects;

  Graph& operator=(const Graph&);
  Graph(Graph&);
  Graph();

  /* object management */
  objref& AddObject(object* o, GraphID_t id, bool managed=false); 
  template <class T>
  objref& AddObject(GraphID_t id); 
  template <class T>
  objref& AddObject(const T& master_copy, GraphID_t id); 

  /* these methods must be called on all processors simultaneously */
  void Prepare_Neighbours(bool cache_requests=false);
  void Partition_Objects();
  void Distribute_Objects();
  void gather();

  void rebuild_local_list();   
  void clear_non_local()
  void print(int proc) 
};                 

\end{verbatim}

%  void DelObject(GraphID_t id);

\begin{description}
\item[Ptrlist](see \S\ref{Ptrlist})\index{Ptrlist} is a list of
  references to \hyperref{{\em objrefs}}{
    (\S}{)}{objref})\index{objref}, pointing to objects stored locally
  on the current processor.
\item[AddObject]\index{AddObject} In the first form, add an already
  created object to the Graph. In the second form create a new object
  of type {\em T}, and add it to the Graph. {\em T} must be derived
  from the abstract base class \verb+object+\index{object}. You must
  explicitly supply the type of the object to be created as a template
  argument:
  \begin{verbatim}
    g.AddObject<foo>(id);
  \end{verbatim}
  In the third form, create a new object, and initialise its data with
  the contents of argument \verb+master_copy+.
%\item[DelObject] removes an object from the Graph, destroying all
%  links to that object.
\item[Prepare\_Neighbours()]\index{Prepare\_Neighbours} For each
  object on the local processor, ensure that all objects connected to
  it are brought up to date, by obtaining data from remote processors
  if necessary. If the network structure has not changed since the
  last call to this method, set the flag
  \verb+cache_requests+\index{cache\_requests} to \verb+true+, which
  substantially reduces the amount of interprocessor communication
  required.
\item[Partition\_Objects()] \index{Partition\_Objects}Call the ParMETIS
  partitioner to redistribute the graph in an optimal way over the
  processors. ParMETIS executes in parallel, and requires that the
  objects be distributed before this call. One way of achieving this is
  to make a simple assignment of objects to processors (by setting the
  \verb+proc+ member of each objref), then call \verb+Distribute_Objects()+.
\item[Distribute\_Objects()]\index{Distribute\_Objects} Broadcast graph data
  from processor 0, and call \\\verb+rebuild_local_list()+ on each
  processor.
\item[gather()]\index{gather} Bring the entire graph on processor 0 up
  to date, copying information from remote processors as necessary. A
  \verb+gather()+, followed by \verb+Distribute_Pins()+ brings all
  processors' graphs up-to-date. This is, naturally, an expensive
  operation, and should be done for startup or checkpointing purposes.
\item[rebuild\_local\_list()]\index{rebuild\_local\_list} Reconstruct
  the list of objrefs local to the current processor, according to the
  \verb+proc+\index{proc} member of the objrefs.
\item[clear\_non\_local()]\index{clear\_non\_local()} Nullify all
  objrefs that don't belong to the current processor. This can be used
  to save memory usage.
\end{description}

\psubsection{Basic usage of Graph}

{\em Graph} is designed to be used in a SPMD parallel environment,
using MPI to handle messages between processors. A copy of the Graph
object is maintained on each process. Each process has a copy of the
objref database (of type \verb+omap+), called
\verb+GRAPHCODE_NS::objectMap+. The \verb+Graph::objects+ reference
refers to this database. However the payload pointer of each objref
will tend to only point to an object if the object is located in the
current processes address space, or a cached copy of the remote object
is needed for some reason. Otherwise it may be set to NULL to save
space.

To call a method \verb+foo()+ on all objects of a Graph \verb+g+ (in
parallel), execute the following code:
\begin{verbatim}
for (Graph::iterator i=g.begin(); i!=g.end(); i++)
   (*i)->foo();
\end{verbatim}

If the method \verb+foo+ needs to know the values of neighbouring
nodes, then you may call \verb+Graph::Prepare_Neighbours()+, which
ensures that a cached copy of any remotely located node linked to a
local nodes is retrieved from the remote node. Thus arbitrary
communication patterns can be expressed simply by the form of the
network structure of the Graph.

\psection{Objects}\index{object}\label{object}

\verb+object+ is an abstract base class, from which all objects stored
in the graph must be derived. A synopsis of the ABC is:
\begin{verbatim}
  class object: public Ptrlist
  {
  public:
    /* serialisation methods */
    virtual void lpack(pack_t *buf)=0; 
    virtual void lunpack(pack_t *buf)=0;
    /* virtual "constructors" */
    virtual object* lnew() const=0;  
    virtual object* lcopy() const=0;  
    virtual ~object() {}
    virtual int type() const=0;     /* return index into archetype table */
    /* partition weightings - redefine in derived type if needed */
    virtual idxtype weight() const {return 1;}
    virtual idxtype edgeweight(const objref& x) const {return 1;}
  };
\end{verbatim}

\psubsection{Serialisation methods}

The first two virtual methods allow Graphcode to access Classdesc
generated serialisation routines. Assuming you have declared a class
foo as follows:\index{lpack}\index{lunpack}

\begin{verbatim}
class foo: public object
{
  ...
  virtual void lpack(pack_t *buf);
  virtual void lunpack(pack_t *buf);
}
\end{verbatim}

Then you may define the virtual functions as follows:
\begin{verbatim}
inline void pack(pack_t *,eco_string,foo&);
inline void unpack(pack_t *,eco_string,foo&);
inline void foo::lpack(pack_t *buf) {pack(buf,eco_string(),*this);}
inline void foo::lunpack(pack_t *buf) {unpack(buf,eco_string(),*this);}
\end{verbatim}

The definitions for \verb+pack(,,foo&)+ and \verb+unpack(,,foo&)+ will
then be created in the usual way by Classdesc.

It is important that \verb+pack(,,foo&)+ and \verb+unpack(,,foo&)+ be explicitly
declared before use, otherwise a default template function will be
linked in which will not work as expected. See the note on using
polymorphic objects under Classdesc.

\psubsection{Virtual Constructors}

Defining the virtual constructors for your objects type is also a
simple matter. Unlike the case of the serialisation routines, they can
even be done inline in the class definition:\index{lnew}\index{lcopy}
\begin{verbatim}
class foo: public object
{
 ...
 virtual object *lnew() {return vnew(this);}
 virtual object *lcopy() {return vcopy(this);}
};
\end{verbatim}

\psubsection{Run Time Type Identification}

To migrate an object from one thread to another, Graphcode needs to be
able to create an object of the correct type in the destination
address space. This is achieved by means of a {\em run time type
  identification} (RTTI)\index{RTTI} system. Given a type token
\verb+t+, an object of that type can be created by the call:\index{archetype}\index{lnew}
\begin{verbatim}
object *o=archetype[t]->lnew();
\end{verbatim}

Instead of using C++'s built-in RTTI system, where tokens are compound
objects of somewhat indeterminate size, Graphcode implements a simple
RTTI system using template programming, in which a type token is a
simple unsigned integer. This implies that each type of object
to be used with Graph must be registered first, before use. This is
taken care for you automatically if you use \verb+Graph::AddObject()+ to
add your object to the Graph.

Adding the virtual type method to your class is also easy:\index{type}



\begin{verbatim}
class foo: public object
{
 ...
 virtual int type() {return vtype(*this);}
};
\end{verbatim}

The first \verb+vtype+\index{vtype} is called on an object, an object
of that type is created (via its \verb+lcopy+ method), and added to
the archetype\index{archetype} vector. The index of that object within
the archetype vector become the type token. {\em It is vitally
  important that types are added to the archetype vector in the same
  order on all threads.} Clearly this is a trivial requirement if only
one type is used, but slightly more care needs to be taken in the case
of multiple types of object.

If you have multiple object types, consider using the
\verb+register_type+ template to ensure a consistent type registration
across the different address spaces.

\psubsection{Node and edge weights}

By default, the graph partitioning algorithm used in Graphcode weights
each node and link equally. However, it is possible to perform load
balancing by specifying a computational weight function on each node,
and a communication weight function for each edge. For example:
\begin{verbatim}
class foo: public object
{
  ...
  virtual idxtype weight() {return size()*size();}
  vitural idxtype edgeweight(const objref& x) {return (*x)->size();}
};
\end{verbatim}

\psubsection{Edge list}

An object is derived from \hyperref{Ptrlist}{
  (\S}{)}{Ptrlist}\index{Ptrlist}, which contains a list of objrefs
that the current object is connected to.

\psubsection{Self linking nodes}

If there is any reason for your node to access its objref (eg to find
out its GraphID, for example), then you can add the objref to its edge
list (say the first item on the edgelist by convention). Then you can
refer to things like \verb+begin()->ID+, \verb+begin()->proc+ etc.

The \verb+Graph::Prepare_Neighbours()+ and
\verb+Graph::Partition_Objects()+ methods ignore self-linking edges.

\psection{objref}\index{objref}\label{objref}

For every object in the Graphcode system, there is an \verb+objref+ on
every processor referring to it.

Synopsis:
\begin{verbatim}
  class objref
  {
  public:
    GraphID_t ID;
    unsigned int proc;

    objref(GraphID_t id=0, int proc=myid(), object *o=NULL);
    objref(GraphID_t id, int proc, object &o); 
    objref(const objref& x);
    ~objref();

    objref& operator=(const objref& x); 
    object& operator*();
    object* operator->();
    const object* operator->() const;
    const object* operator->() const;
    void addref(object* o, bool mflag=false); 
    bool nullref() const;
    void nullify();
  };
\end{verbatim}

\begin{description}
\item[ID] A unique integer value that identifies the object within a Graph
\item[proc] The location of the active copy of the object
\item[{\tt *, ->}] Dereferencing an objref allows one to access the
  object. It is an error to dereference a nullified objref.
\item[addref] Add an object to this reference. The \verb+mflag+
  parameter indicates whether the object is managed (created by \verb+new+ or
  \verb+lnew()+), and can be safely \verb+deleted+ when nullified, or
  is an external object that should not be deleted. An objref can also
  be instantiated already pointing to an unmanaged object via its constructor.
\item[nullref] whether this objref points to an object or not. If the
  active copy is on this processor, then nullref should be false.
  Otherwise, it may true or false depending on whether this processor
  has a cached copy of the object.
\item[nullify] Remove the object from the Graph (but leaving the
  objref in place). It is an error to remove the active copy, without
  replacing it with another object.
\end{description}


\psection{Ptrlist}\index{Ptrlist}\label{Ptrlist}

Ptrlists work a bit like \verb+std::vector<objref>+ --- objrefs can be
added with \verb+Ptrlist::push_back()+, indexed with standard array
indexing \verb+Ptrlist::operator[]+, and iterated over in the usual
way with \verb+Ptrlist::iterator+. However, unlike
\verb+std::vector<objref>+, only pointers to the objref is stored
within \verb+Ptrlist+, not copies.

Synopsis:

\begin{verbatim}
class Ptrlist 
  {
  public:
    
    class iterator
    {
    public:
      objref& operator*();
      objref* operator->();
      iterator operator++();
      iterator operator--();
      iterator operator++(int);
      iterator operator--(int);
      bool operator==(const iterator& x);
      bool operator!=(const iterator& x);
    };
    iterator begin() const;
    iterator end() const;
    objref &    front ()
    objref &    back ()
    unsigned size() const;
    objref& operator[](unsigned i) const; 
    void push_back(objref* x);
    void push_back(objref& x);
    void erase(GraphID_t i);
    void clear();
    Ptrlist& operator=(const Ptrlist &x);
  };
\end{verbatim}

Ptrlists can only refer to objects stored in objectMap.
Ptrlists can be serialised --- Ptrlists must be unpacked within the
context of an omapthe objectMap.

If you need to use a backing map, you can declare another omap object
and assign objectMap to it. This will create copies of all the objects
contained within objectMap.

\psection{omap}\index{omap}\label{omap}

An omap is a container for storing \hyperref{{\em objrefs}}{
  \S(}{)}{objref}\index{objref}, indexed by ID.

\begin{verbatim}
  class omap: public MAP
  {
  public:
    objref& operator[](GraphID_t i);
    omap& operator=(omap& x);
  };
\end{verbatim}

There are a few different possible ways of implementing omaps, with
differing performance characteristics.  Graphcode provides two
different models, {\em vmap} and {\em hmap} that may be readily
deployed by the user, however users can fairly easily provide their
own implementation if desired. Different implementations can be
selected by defining the \verb+MAP+ macro\index{MAP} to be the desired
omap implementation before including \verb+graphcode.h+. This will
declare everything in the namespace
\verb+graphcode_vmap+\index{graphcode\_vmap} or
\verb+graphcode_hmap+\index{graphcode\_hmap} as appropriate. Using
this scheme, it is possible to have two different omap types in the
one object file, by including graphcode.h twice. However, if you do
this, you will need to \verb+#undef GRAPHCODE_H+ guard variable prior
to subsequent includes.

vmap is intended for use with contiguous GraphID ranges. If there are
holes in the identifier range, then the iterator will return invalid
references for these holes, and the size() method will be incorrect.

If you need to have non-contiguous ID ranges (perhaps for dynamic
graph management --- note this is not currently supported), then please
use the hmap\index{hmap} implementation instead (which will have some performance
penalty).


MAP must provide the following members:
\begin{verbatim}
class MAP
{
  protected:
    objref& at(GraphID_t i);
  public:
    MAP();
    MAP(const MAP&)
    class iterator
     {
       iterator();
       iterator(const iterator&);
       iterator& operator=(const iterator&); 
       iterator operator++(int);
       iterator operator++();
       iterator operator--(int);
       iterator operator--();
       bool operator==(const iterator& x) const;
       bool operator!=(const iterator& x) const;
       objref& operator*();
       objref* operator->();
     };
    iterator begin();
    iterator end();
    unsigned size();
}
\end{verbatim}
   
The \verb+at+\index{at} method is essentially a replacement for \verb+operator[]()+. A
simple example of an omap implementation is provided by \verb+vmap+\index{vmap}:
\begin{verbatim}
class vmap: public std::vector<objref>
{
protected:
  objref& at(GraphID_t i) 
  {
    if (i>=size()) resize(i+1);
    return std::vector<objref>::operator[](i);
  }
};
\end{verbatim}

hmap is a hash map implementation. With all hash maps, performance of
the map is critically dependent upon the choice of hash function, which
is application dependent. hmap is simply defined as:
\begin{verbatim}
class hmap: public hashmap<simple_hash> {};
\end{verbatim}
You can provide your own omap definition (umap, say), with your own
user defined hash function in the following way:
\begin{enumerate}
\item Create a file ``umap'' somewhere in the default search path with
  the following:
\begin{verbatim}
#include "hashmap.h"
struct myhash
{
  unsigned operator()(GraphID_t i) {...}
};
class umap: public hashmap<myhash> {};
\end{verbatim}
\item Add the new omap definitions to the Graphcode library:
\begin{verbatim}
make MAP=umap
\end{verbatim}
\item Include the declarations of the \verb+graphcode_umap+ namespace
  in your application source file:
\begin{verbatim}
#define MAP umap
#undef GRAPHCODE_H
#include <graphcode.h>
\end{verbatim}
\end{enumerate}

\psubsection{Note on using macros to parametrise omap}

It might seem obvious that omap could be made a templated type, with a
single template parameter being the implementation of the
omap. Unfortunately, omaps contain objects, which in turn have
Ptrlists (the edge list), which used to maintain a reference to an
omap. So objects and objrefs must be similarly parametrised by by the
omap implementation type. However the omap implementation type will
also need to reference the parametrised objrefs, leading to an
implementation type parametrised by itself, an unparsable situation.

The way out of this dilemma is to make omap an abstract base class,
which can be made into a concrete implementation as part of the
definition of Graph. However, this introduces an extra level of
overhead in calling virtual functions when performing object
lookup, which is not insignificant. It also dramatically increases the complexity of coding
\verb+object::iterator+, which also would need to be an abstract base
class.

In view of this, the macro based solution finally chosen, seemed the
cleanest, and most efficient means of implementing different omap
implementations.

Obviously, this reason is no longer relevant, however there doesn't seem to
be any burning reason to change the macro parametrisation to a
template argument.

\psection{Building Graphcode}

In building graphcode, you configure the options you want by setting
Make variables on the command line:
\begin{description}
\item[MPI=1] Build the MPI version of Graphcode
\item[MAP=] specify which map to use for omap. hmap is the
  default. You can build a library supporting multiple different omap
  types by issuing successive make commands:
\begin{verbatim}
make MAP=vmap
make MAP=hmap
\end{verbatim}
\item[DEBUGGING=1] Used to enable assertions, as well as debugger
  symbols. Optimisation is turned off
\item[PREFIX=] specify an install directory when building the
  \verb+install+ target (default \verb+~/usr+).
\end{description}

\psection{Using graphcode in parallel}

To use graphcode in parallel, you need to install
Classdesc,\index{Classdesc} ParMETIS\index{Metis}\index{ParMETIS} and
MPI.\index{MPI} Define the preprocessor symbol
\verb+MPI_SUPPORT+\index{MPI\_SUPPORT} to enable the parallel
processing code. An example Makefile for the \verb+poisson_demo+
example illustrate how this is done.\index{poisson\_demo}

You will need to arrange the class definitions for your objects, as
well as the graphcode.h file to be processed by 
classdesc. One way of doing this is to \verb+#include+ .cd files into one of
the C++ source files, and have a \verb+.h.cd+ rule in your Makefile, as
suggested in the classdesc documentation.


\psection{Examples}

Graphcode comes with an example of solving the Poisson
equation. Graphcode is also deployed for implementing the spatial
Ecolab model, and the Palauan jellyfish model within the \EcoLab{}
package. The latter model illustrates a dynamic load balancing.
