\documentstyle[pstricks,pst-node,html,epsf,makeidx]{article}
\title{\EcoLab{} Documentation}
\author{Russell K. Standish}
\newcommand{\br}{\mbox{${\bf r}$}} %reproductive rate
\newcommand{\bbeta}{\mbox{\boldmath{$\beta$}}}   %interaction matrix
\newcommand{\bgamma}{\mbox{\boldmath{$\gamma$}}} %dispersal
\newcommand{\bmu}{\mbox{\boldmath{$\mu$}}}       %mutation rate
\newcommand{\bn}{\mbox{\boldmath{$n$}}}          %species density
\newcommand{\nsp}{\mbox{$n_{\rm sp}$}}           %no. of species           

\newcommand{\EcoLab}{{\sffamily\slshape
    \mbox{\raisebox{.5ex}{Eco}\hspace{-.4em}{\makebox[.5em]{L}ab}}}}

\newcommand{\psection}[1]{\subsection{#1}}
\newcommand{\psubsection}[1]{\subsubsection{#1}}

%begin{latexonly}
\renewcommand{\htmladdnormallinkfoot}[2]{#1\footnote{#2}}
%end{latexonly}

\makeindex

\begin{document}
\maketitle
\begin{center}
Version 5.56

\end{center}
\begin{abstract}
  \EcoLab{} is an object oriented simulation environment that
  implements an experiment oriented metaphor. It provides a series of
  instruments that can be coupled together with the user's model
  (written in C++) at runtime in order to visualise the model, as well
  as support for for distributing agents over an arbitrary topology
  graph, partitioned over multiple processors and checkpoint/restart
  support. \EcoLab{} was originally developed to simulate a particular
  model (the \EcoLab{} model) of an abstract
  ecology\cite{Standish94}. However, several other quite different
  models have been implemented using the software, demonstrating its
  general purpose nature.
\end{abstract}

\tableofcontents

\section{License}

\EcoLab{} is distributed as unrestricted public domain open-source
software, which you can use as you wish. The netcomplexity module
makes use of {\em nauty}, which has license restrictions documented in
include/nauty.h. Commercial or military applications of \EcoLab{} must
make do without netcomplexity functionality.

\section{Making \protect\EcoLab}

Once you have unpacked the gzipped tar file, you should be able to
make \EcoLab{} by running ``make'' or ``gmake'' in the top level
directory. \EcoLab{} needs GNU make (which available by default on
Linux, Cygwin/Windows or MacOS, but can be installed from sources on
most operating systems if needed). If you have a multiprocessor, or
multicore computer, you can speed up the build time by spreading the
compilation over multiple processors using the \verb+-j+ option to \verb+make+.
Currently \EcoLab{} has been developed against TCL/Tk 8.5 and gcc 4.5,
but should build with other versions.

\EcoLab{}'s Makefile searches for the software it needs, and sets
Makefile flags appropriately depending on what it finds. The only
software that is an absolute requirement is TCL, however it will also
use Tk, BLT (version 2.4.z), Cairo, zlib, readline, XDR, UNURAN or GNUSL,
Berkley DB or gdbm if available. 

For the plotting widgets, EcoLab will use Cairo by preference - if you
wish to use the older BLT-based widgets, specify BLT=1 on the makefile
build line. If Cairo is not installed on the system, Ecolab will fall
back to using BLT, and if that is not available either, will not
provide the plotting widgets.

For parallel programming, it also uses MPI and ParMETIS. \EcoLab{}'s
Makefile searches for the software in \verb+$HOME/usr+, then
\verb+/usr/local+ and finally \verb+/usr+. When installing 3rd party
libraries, please install them in either \verb+$HOME/usr+ or
\verb+/usr/local+, usually by specify the \verb+--prefix+ option to
the ``configure'' command of the package.

You can override the default settings by passing options to make. In
the following table, you can define an option by specifying (eg MPI=1)
on the make command line, or undefine it using (eg MPI=).

\noindent
\begin{tabular}{|l|p{9cm}|}
\hline
DEBUGGING & Turns on -g, and assertions\\
MEMDEBUG & replaces \verb+new+ and \verb+delete+ with a version that
tracks and reports memory usage\\
PROFILING & logs times executed by each \hyperref{TCL
  command}{TCL command (\S}{)}{NEWCMD}\\
GCOV & Prepares executable for test coverage analysis\\
VPROF & Prepares executables for use by the \htmladdnormallinkfoot{vprof}{http://aros.ca.sandia.gov/\~{}cljanss/perf/vprof/} tool \\
MPI & Enable distributed parallel version \\
OPENMP & Enable the use of OpenMP shared memory parallel constructs \\
PARALLEL & Enable the use of automatic parallelisation (Intel compiler) \\
XDR & Enable use of \hyperref{XDR}{XDR (\S}{)}{XDR} for checkpoints and
client-server. On some systems, XDR needs to be disabled for correct
compilation in MPI mode. This bug should be fixed in this version of \EcoLab.\\
GCC & Force use of gcc compiler. \\ 
NOGUI & Disables all GUI functionality. \\
BDB & Use Berkley database from Sleepycat for the
\hyperref{cachedDBM}{cachedDBM \S}{)}{cachedDBM} class. \\
UNURAN & select UNURAN random number library \\
GNUSL & select GNUSL random number library \\
PREFIX & installation location for ``make install''. Defaults to
\verb+$HOME/usr/ecolab+.\\
\hline
\end{tabular}

\EcoLab{} will use the \htmladdnormallinkfoot{UNURAN non-uniform
  random number library}{http://statistik.wu-wien.ac.at/unuran/} or
  the \htmladdnormallinkfoot{GNU Scientific
  Library}{http://www.gnu.org/software/gsl/} if
available. Note that the author uses UNURAN, so GNUSL is typically not
  tested. If neither library is available, a limited library based on
  the libc \verb+rand()+ function is used instead. The Make flags
  \verb+UNURAN+ and \verb+GNUSL+ can be used to override the defaults.


  Please note the following point when installing unuran (tested
  against unuran 1.6.0):
\begin{itemize}
\item EcoLab assumes that if PRNG is available on the system, then
  UNURAN will use it. However, UNURAN will not detect PRNG if it is installed
  in {\tt\$HOME/usr}. If you have a linkage error complaining about
  missing prng functions, either reinstall UNURAN with the
  --with-urng-prng option to its configure script, or compile EcoLab
  with the ``{\tt PRNG=}'' Makefile option. The former is the
  preferred option, as it allows runtime configuration of the uniform
  random generator via PRNG's string interface.
\end{itemize}


The other options are relatively straight forward.  
Define {\tt MPI}\index{MPI} if you wish to run \EcoLab{} in parallel
using the MPI message passing library. You will also need to install
the
\htmladdnormallinkfoot{ParMETIS}{http://www-users.cs.umn.edu/~karypis/metis/parmetis/}
library.

The standard build will also build in the example models directory.
You may also do a {\tt make install}, which will install the \EcoLab{}
software into the directory given by the {\tt PREFIX}\index{PREFIX}
variable in the top level Makefile. By default, this is {\tt
  \~/usr/ecolab}. You can change this value to something else (eg
/usr/local/ecolab) by giving the command {\tt make
  PREFIX=/usr/local/ecolab install}. After that, you can compile the
{\tt models} directory completely independently of the rest of the
distribution --- use the example Makefile in the {\tt models}
directory as a template. If you have installed \EcoLab{} in a
non-standard location, you may need to modify the {\tt ECOLAB\_HOME}
Makefile variable to point to the installed directory.

\section{Using Ecolab on Windows}

The preferred way of using \EcoLab{} on windows is to use the \htmladdnormallinkfoot{Cygwin
posix emulation environment}{http://cygwin.com}, which has most the dependent packages
available as installable options. However \EcoLab{} can also be built
using the MinGW compilers, although you will need some sort of
posix-like shell, eg Cygwin, or Msys to do the actual build on
Windows. EcoLab{} is, however, validated to build on
\htmladdnormallinkfoot{MXE}{http://mxe.cc}, a cross-compiler
environment based on MinGW to build Windows executables using Linux. Check out
\htmladdnormallinkfoot{my fork of
  MXE}{https://github.com/highperformancecoder/mxe} which may contain
additional packages needed for \EcoLab{}, that have not yet been
integrated into the MXE master branch.

\subsection{Cygwin}

\htmladdnormallinkfoot{Cygwin}{http://cygwin.com} is a
port of the GNU development environment to 32 bit Windows (Windows
95/98/ME and Windows NT/2000/XP), and runs fine in 64 bits too. As such, it provides as near to
Posix like environment as is possible in Windows. There are a few
hints needed to get Ecolab compiled and running under Cygwin.
\begin{itemize}
\item When installing Cygwin, select the following additional
  packages:
  \begin{description}
    \item[Devel] gcc-g++ and make
    \item[Libs] libcairo-devel, libdb4.8-devel, gsl-devel, libreadline-devel, zlib-devel,
    \item[Tcl] tcl-tk-devel
    \item[X11] xorg-server and xinit
  \end{description}
\item Optionally install the PRNG/UNURAN libraries by building them
  from source code. UNURAN is required for running the jellyfish model.
\item Unpack Ecolab and do {\tt make} in top level directory. 
\end{itemize}

\EcoLab{} must be used under X11 on Cygwin. To start the X11 server,
type ``startx'' at the Cygwin console. X11 programs run under this
server appear to be running natively under Windows --- its quite neat!

Once the X server has started, it will pop another shell window from
which you can run Xwindows programs. {\tt cd} to the {\tt models}
directory, and run the example ecolab script as {\tt ./ecolab
  ecolab.tcl}. The \verb+#!+ mechanism does not work with Cygwin!

\subsection{MXE}

To build EcoLab using MXE, you first need to build MXE by cloning the
mxe repository, and building everything:
\begin{verbatim}
git clone https://github.com/highperformancecoder/mxe.git
cd mxe
make
PATH=$PATH:`pwd`/usr/bin
\end{verbatim}
Now you can build ecolab by building in the toplevel source directory:
\begin{verbatim}
make MXE=1
\end{verbatim}
If you need to build additional third party packages, such as unuran,
install these into \verb+$HOME/usr/mxe+ It is generally quite easy to
build packages for MXE if they use autogen, cmake or qmake Makefile
generators.

To run a MXE-built ecolab executable, copy the entire Ecolab source
directory with the MXE binaries to your windows system. Then run
\verb+install.bat+, which installs the TCL libraries required to
support \EcoLab{} in the standard APPDATA location on Windows. Then
you can run \EcoLab{} as usual from any command line tool (Cygwin not
required, X-windows not required).

Whilst MXE can be used to build \EcoLab{} models for running, using
Cygwin is a lot simpler. However, MXE is useful for building
standalone applications that use the \EcoLab{} library, such as
\htmladdnormallinkfoot{Minsky}{http://minsky.sf.net}.


\section{Using Ecolab under Mac OSX}

\EcoLab{} compiles out of the box on OSX\index{OSX} using the Mac OSX
Dev Tools package, which is available for free download from
\htmladdnormallink{http://connect.apple.com}. You will also need a
copy of TCL/Tk - the
\htmladdnormallink{ActiveTCL}{http://www.activestate.com/activetcl}
works well, although building TCL/Tk from source is also possible. If
you do the latter, choose the unix build directory, not the macosx
one. 

You need to make a choice whether you want to run \EcoLab{} as a
native Aqua application, or as an X-windows application under the
XFree86 server software. Specify
\verb+--enable-aqua+ on the Tk configure line to build for Aqua,
rather than X11. On the \EcoLab build line, specify \verb+AQUA=1+ on
the make command line.

You may also use
\htmladdnormallink{MacPorts}{https://www.macports.org/} to easily
build the prerequisite packages. \EcoLab{}'s Makefiles will search for
Macports software located in \verb+/opt/local+ prior to searching
system locations. Note that if you install the MacPorts version of Tk,
it uses X11 only, Aqua mode is unavailable.

\subsection{Compiling and running \protect\EcoLab{} for Aqua}

Build \EcoLab{} with \verb+AQUA=1+. In the models directory, an
``application bundle'' is created for each model, for example
\verb+ecolab.app+, using the utility script mkmacapp.\index{mkmacapp}
If you try to run \EcoLab{} scripts in the usual way, you will get an
error message:\index{SetFrontProcess}
\begin{verbatim}
SetFrontProcess failed,-606
\end{verbatim}
Instead, you must invoke \EcoLab{} in a rather cumbersome way:
\begin{verbatim}
ecolab.app/Contents/MacOS/ecolab ecolab.tcl
\end{verbatim}
To make it a little easier, the \verb+macrun+\index{macrun} script is
provided to do this for you --- try
\begin{verbatim}
macrun jellyfish.tcl lakes/OTM
\end{verbatim}

\section{Structure of the \protect\EcoLab{} Simulation System}

\EcoLab{} is built on the following components:
\begin{description}
\item[tcl++] which provides bindings to the TCL scripting language,
\item[TCL\_obj] which exposes the internals of C++ objects to TCL
\item[pack] performs serialisation of objects, for checkpoint, and
  client/server applications
\item[ClassdescMP] parallel programming support
\item[Graphcode] provides an abstraction of objects moving on a
  distributed graph.
\item[arrays] which implements dynamic arrays used in a number of
  \EcoLab models
\item[analysis/Xecolab] which provides a number of generic instruments
  for observing the \EcoLab{} models
\item[random] a thin abstraction on random number generators, kindly
  supplied by UNURAN or GNUSSL (as preferred)
\item[cacheDBM] Persistent object map
\item[eco\_strstream] classdesc stringstream class
\item[eco\_hashmap] hashmap --- using either std::map or TCL's hash map
\item[netcomplexity] Network complexity measure
\item[C++ Standard Library] (of course!)
\end{description}

The \EcoLab{} model itself is defined in the model specific file {\tt
  ecolab.cc}. To define another model, replace this file by another
  one with similar functions. An example of this is {\tt
  jellyfish.cc}. See \S\ref{new model} for more details.

The whole computation is constructed from a TCL\cite{Ousterhout94}
script. Example scripts include {\tt ecolab.tcl}, {\tt pred-prey.tcl}
 and {\tt console.tcl/engine.tcl} for a sample
client/server system.

\begin{figure}
\pspicture(2,0)(15,7)
\rput(7,7){\rnode{ecolabtcl}{\psframebox[fillstyle=solid,fillcolor=lightgray]{ecolab.tcl}}}
\rput(3,5){\rnode{modeltcl}{\psframebox[fillstyle=solid,fillcolor=lightgray]{model.tcl}}}
\rput(5,5){\rnode{Xecolab}{\psframebox{Xecolab}}}
\rput(11,7){\rnode{ecolabcc}{\psframebox[fillstyle=solid,fillcolor=lightgray]{ecolab.cc}}}
\rput(8,5){\rnode{TCL_obj}{\psframebox{TCL\_obj}}}
\rput(5,3){\rnode{analysis}{\psframebox{analysis.cc}}}
\rput(3,3){\rnode{BLT}{\psframebox{BLT}}}
\rput(13,3){\rnode{pack}{\psframebox{pack}}}
\rput(13,4){\rnode{classdescMP}{\psframebox{ClassdescMP}}}
\rput(13,5){\rnode{graphcode}{\psframebox{Graphcode}}}
\rput(15,4){\rnode{MPI}{\psframebox{MPI}}}
\rput(10,3){\rnode{tclarrays}{\psframebox{tcl\_arrays.cc}}}
\rput(6,1){\rnode{ecostrstream}{\psframebox{eco\_strstream.cc}}}
\rput(8,0){\rnode{ecostring}{\psframebox{eco\_string.cc}}}
\rput(3,1){\rnode{tclpp}{\psframebox{tcl++.h}}}
\rput(12,1){\rnode{arrays}{\psframebox{arrays.h}}}
\rput(12,0){\rnode{carrays}{\psframebox{c\_arrays.c}}}
\rput(3,0){\rnode{tcltk}{\psframebox{TCL/TK}}}
\ncline{ecolabtcl}{modeltcl}
\ncline{ecolabtcl}{Xecolab}
\ncline{ecolabtcl}{TCL_obj}
\ncline{TCL_obj}{ecolabcc}
\ncline{Xecolab}{analysis}
%\ncline{ecolabcc}{tclpp}
\ncline{TCL_obj}{pack}
\ncdiagg[angleA=280,arm=2]{ecolabcc}{arrays}
\ncline{analysis}{tclpp}
\ncline{arrays}{carrays}
\ncdiagg[arm=2,angleA=90]{tclarrays}{ecolabtcl}
\ncline{tclarrays}{tclpp}
\ncline{tclarrays}{arrays}
\ncdiagg[arm=1.5,angleA=250]{TCL_obj}{tclpp}
\ncline{TCL_obj}{ecostring}
\ncline{tclpp}{ecostrstream}
\ncline{pack}{ecostring}
\ncline{ecostrstream}{ecostring}
\ncline{classdescMP}{pack}
\ncline{graphcode}{classdescMP}
\ncline{classdescMP}{MPI}
\ncline{ecolabcc}{graphcode}
\ncline{BLT}{analysis}
\ncline{BLT}{Xecolab}
\ncline{tcltk}{tclpp}
\endpspicture
\caption{Structure of the \protect\EcoLab{} Simulation System for
  the \protect\EcoLab{} model. Shaded boxes are specific to the
  \protect\EcoLab{} model, and are replaced by equivalent modules for other models.} 
\end{figure}

\begin{figure}
\pspicture(2,0)(15,7)
\rput(7,7){\rnode{jellyfishtcl}{\psframebox[fillstyle=solid,fillcolor=lightgray]{jellyfish.tcl}}}
\rput(3,5){\rnode{lakes}{\psframebox[fillstyle=solid,fillcolor=lightgray]{lakes/}}}
\rput(5,5){\rnode{Xecolab}{\psframebox{Xecolab}}}
\rput(11,7){\rnode{jellyfishcc}{\psframebox[fillstyle=solid,fillcolor=lightgray]{jellyfish.cc}}}
\rput(8,5){\rnode{TCL_obj}{\psframebox{TCL\_obj}}}
\rput(5,3){\rnode{analysis}{\psframebox{analysis.cc}}}
\rput(3,3){\rnode{BLT}{\psframebox{BLT}}}
\rput(10,2){\rnode{pack}{\psframebox{pack}}}
\rput(10,3){\rnode{classdescMP}{\psframebox{ClassdescMP}}}
\rput(10,4){\rnode{graphcode}{\psframebox{Graphcode}}}
\rput(12,3){\rnode{MPI}{\psframebox{MPI}}}
\rput(6,1){\rnode{ecostrstream}{\psframebox{eco\_strstream.cc}}}
\rput(8,0){\rnode{ecostring}{\psframebox{eco\_string.cc}}}
\rput(3,1){\rnode{tclpp}{\psframebox{tcl++.h}}}
\rput(3,0){\rnode{tcltk}{\psframebox{TCL/TK}}}
\rput(11,6){\rnode{arrays}{\psframebox{arrays.h}}}
\rput(11,5){\rnode{carrays}{\psframebox{c\_arrays.c}}}
\rput(13,7){\rnode{tk}{\psframebox{TK}}}
\rput(13,6){\rnode{random}{\psframebox{random}}}
\rput(13,5){\rnode{UNURAN}{\psframebox{UNURAN}}}
\ncline{jellyfishtcl}{lakes}
\ncline{jellyfishtcl}{Xecolab}
\ncline{jellyfishtcl}{TCL_obj}
\ncline{TCL_obj}{jellyfishcc}
\ncline{Xecolab}{analysis}
\ncdiagg[arm=2,angle=285]{TCL_obj}{pack}
\ncline{jellyfishcc}{arrays}
\ncline{analysis}{tclpp}
\ncline{tcltk}{tclpp}
\ncline{arrays}{carrays}
\ncline{jellyfishcc}{random}
\ncline{jellyfishcc}{tk}
\ncline{random}{UNURAN}
\ncdiagg[arm=1.5,angleA=250]{TCL_obj}{tclpp}
\ncline{TCL_obj}{ecostring}
\ncline{tclpp}{ecostrstream}
\ncline{pack}{ecostring}
\ncline{ecostrstream}{ecostring}
\ncline{classdescMP}{pack}
\ncline{graphcode}{classdescMP}
\ncline{classdescMP}{MPI}
\ncdiagg[arm=.85,angle=220]{jellyfishcc}{graphcode}
\ncline{BLT}{analysis}
\ncline{BLT}{Xecolab}
\endpspicture
\caption{Structure of the \protect\EcoLab{} Simulation System for
  the Jellyfish model. Shaded boxes are specific to the
  Jellyfish model, and are replaced by equivalent modules for other models.} 
\end{figure}

\section{Constructing an experiment}

An experiment consists of a TCL\cite{Ousterhout94} script, which binds
the various elements used (the model system, input parameters,
instruments) into an experimental run. An example experiment is {\tt
  ecolab.tcl}\index{ecolab.tcl}. This is an executable script --- once
you have made \EcoLab, you can run this script.

The script consists of several parts --- the first being a simulation
loop which steps the model through the generate and mutate operators,
then updates the various instruments ({\tt display, plot,
  connect\_plot} etc.) 
There is a {\tt running}\index{running} flag which controls whether
the simulation is running or not. This is used by the {\bf
  run}\index{run} and {\bf stop}\index{stop} buttons to control the
execution of the simulation.

The other parts of the experimental script have been broken into
separate TCL files --- model.tcl\index{model.tcl} contains the input
parameters to the model, and Xecolab.tcl\index{Xecolab.tcl} has the
TCL code relating to the X-windows interface.

\subsection{Input parameters}

\EcoLab{} exports all the members of the object named in the
\verb+make_model+ macro\index{make\_model} to the TCL environment. For
example, if the model object is called \verb+ecolab+, accessor
TCL commands for the members are created --- \verb+ecolab.sp_sep+ to
access the \verb+sp_sep+ member of the ecolab object. Calling the
accessor command with no arguments returns the current value of the
member. Calling it with an argument sets the value of the member.

The \verb+use_namespace+ \index{use\_namespace}(\S\ref{use_namespace})
command can be used to dump all model commands into global
namespace. This unclutters the setting of model parameters considerably.

An alternative technique for getting TCL data into your model is to
use a \hyperref{{\tt tcl\_var}}{{\tt tcl\_var}
  (\S}{)}{tclvar}\index{tcl\_var}. Of the two methods discussed here,
the former is recommended.

\subsection{The main button bar}

The main button bar has five predefined buttons, {\bf
  quit}\index{quit} (which causes the application to quit), {\bf
  run}\index{run}, which calls the {\tt simulate}\index{simulate}
procedure, and {\bf stop}\index{stop} which sets the {\tt
  running}\index{running} flag to zero, suspending the experiment.
{\bf Command} calls the \hyperref{command line
  interpreter}{(\S}{)}{cli}, allowing you to interact directly with
the TCL interpreter. {\bf Object Browser} is a widget allowing you to
graphically drill down into the objects defined in the model.

Additionally, there are two user defined
buttons\index{buttons, user
  defined}\index{.user1}\index{.user2}\index{.user3}, which can have
functions bound to them by adding TCL code after the {\tt
  GUI}\index{GUI} command like the following example:
\begin{verbatim}
GUI
.user1 configure -text condense -command condense
\end{verbatim}

\subsection{use\_namespace}\label{use_namespace}\index{use\_namespace}

The \verb+use_namespace+ {\em name} TCL command searches for all TCL
commands of the form {\em name}.{\em x}, and creates a new TCL command
{\em x} that calls {\em name}.{\em x}. This allows a simple ``dump''
of \EcoLab{} model methods and instance variables into TCL's current
namespace. The term ``namespace'' in \verb+use_namespace+ is unrelated
to TCL's namespace concept.

Note that \verb+use_namespace+ does not override an any existing
commands in the current namespace. Imagine what havoc an instance
variable called ``proc'' would make! Thus it may be necessary to refer
to an instance variable or method by its fully qualified name fro TCL
scripts to get the desired behaviour.

\subsection{Command Line Interpreter}\index{cli}\label{cli}

The command line interpreter allows you to type any TCL command at the
console, rather like the wish tool from Tk. Unlike wish though, the
GNU readline library is employed, so expect the usual command and
history editing functionality you expect from bash. It is simply
invoked by the TCL command \verb+cli+.

If the readline library is not available, then \verb+cli+ will not
provide command line editing. However, you can still supply a script
to \EcoLab's standard input.

TCL is not a thread aware system (since it runs on Windows and MacOS),
so when the command line interpreter is running, nothing else is --- eg
GUI widgets or the model. This restriction may be lifted in the future
on Posix compliant systems.

\subsection{Object Browser}\label{object-browser}

The object browser is a drilldown widget for probing the values of
model variables. Since model variables are implemented as TCL
commands, this tool lists all TCL commands. Commands containg a ``.''
in their name belong to objects, so these commands are grouped and
labelled in blue. TCL namespaces are grouped and labelled in green.
Clicking on this blue group opens another window containing just that
group of commands. Clicking on a green name drills down into a
namespace. Clicking on a red command runs it, and the result is
displayed next to it --- if this is a Tcl\_obj instance variable, the
result is the value of the variable. You can specify an argument to
the command in the {\bf Args} input box.

Shift clicking allows you to select a range of commands to execute, and
Control clicking allows a non-contiguous selection to be made. This is
more useful for the {\bf Repeat} option, which repeats the selected
commands every second, allowing you to dynamically follow a model's variables.

Right mouse click on any item brings up a menu of options. 
\begin{itemize}
\item ``hide'', which removes that command from the
display. In this way, you can customize what's in the display. This is
particularly useful when using the {\bf Repeat} option.
\item ``plot'' Feed the value of the variable into a plot widget, to
  track its value over time. Note that the x access in this case is
  approximate wallclock time --- it may have no connection whatsoever
  on model time. It is, however, a convenient diagnostic for model
  exploration and debugging.
\end{itemize}

\subsection{Instrumentation}

Currently four instrumented widgets are included in the \EcoLab{}
distribution. The first time these widgets are called, they
instantiate themselves in a separate window. This follows the
philosophy that initialisation should be a transparent operation. A
number of these widgets depend on the BLT toolkit, so it is wise to
include BLT\index{BLT} at the compilation stage. Each widget is
independent, and so multiple instruments can be operating at the same
time.

\label{widgets namespace}
Each such widget creates a new namespace {\em x}, the name of which is either
supplied by the user, or generated from the arguments. Also created is
a window heirarchy starting with .{\em x}. The widget command returns
the namespace name which can be captured for future reference. 


One of the uses of this namespace is to call the widget specific
\verb+print+ command\index{print}, which dumps a postscript
representation of the widget to a named file. For example, a plot
window start with \verb+plot nsp ...+ can be printed to ``plot.ps''
using
\begin{verbatim}
nsp::print plot.ps
\end{verbatim} 

\subsubsection{Plot}

Plot a number of TCL variables

Usage:

{\tt plot}\index{plot} {\em plotname} [{\tt -title} {\em title
  string}] {\em x y1} [{\em y2\ldots}]

The arguments can be TCL variable names, and the names are used as
labels in the graph. If you just supply numerical values, then the
variables will be just names arbitrarily x, y1, y2 etc. {\em plotname}
labels the particular widget, so multiple plots can be performed
simultaneously.

The plot can be zoomed by selecting a rectangular region using the
left mouse button. The right mouse button reverts to the original scale.

plot supports the following methods:
\begin{itemize}
\item {\em plotname}::print {\em filename} --- produce a postscript
  representation to a file
\item {\em  plotname}::clear --- clear the plot
\end{itemize}

\subsubsection{Histogram}

Usage:

{\tt histogram}\index{histogram} {\em plotname} [{\tt -title} {\em
  title string}] {\em list of values}

This command adds the values to a histogram plot. A record of all data
values so far added to the histogram is stored in the file {\em
  plotname}{\tt .dat}\index{.dat files}. This allows automatic
recalculation of the histogram's bins whenever a data value goes out
of bounds. The number of bins is also dynamically controlled by the
scale widget on the side. The histogram is recalculated next time {\tt
  histogram} is called after the number of bins has been altered. This
can be a time consuming business, so it is recommended that the
experiment is suspended before changing the number of bins.

There is also the option of plotting the histogram with logarithmic scales.

histogram supports the following methods:
\begin{itemize}
\item {\em plotname}::print {\em filename} --- produce a postscript
  representation to a file
\item {\em  plotname}::clear --- clear the plot
\item {\em  plotname}::outputdat {\em filename} --- output histogram
  to a text file
\item {\em  plotname}::xlogscale --- toggle logarithmic x axis
\item {\em  plotname}::ylogscale --- toggle logarithmic y axis
\item {\em  plotname}::setnbins {\em nbins} --- set the number of bins
\end{itemize}

\subsubsection{Display}\label{display}

Usage:

{\tt display}\index{display} {\em model\_var} {\em identity\_var}

This command plots each component of the {\em model variable} as a function
of time. The {\em identity variable} is used to ensure continuity of
the curves, and colouring them to aid identification of species. The
index of the component is not a good proxy for this, as the index of a
particular species may have changed through a {\tt condense}
operation. 

Traditional \EcoLab{} usage of this command is
\begin{verbatim}
display ecolab.density ecolab.species
\end{verbatim}

The colour of the lines used are controlled by the {\tt
  palette} TCL variable.(\S\ref{palette}) This is a list of X-window
colours to use.

The namespace for this widget is constructed by appending the 1st
argument to the string ``display\_''. Any unacceptable characters are
replaced by '\_', so it may take a little experimentation to find the
correct namespace name.
display supports the following methods:
\begin{itemize}
\item display\_{\em model\_var}::print {\em filename} --- produce a postscript
  representation to a file
\item display\_{\em  model\_var}::clear --- clear the plot
\end{itemize}

\subsubsection{Connect\_plot}\label{connect}

Usage:

{\tt connect\_plot} {\em interaction} {\em density} 


This command displays the connectivity of the interaction matrix. The
ecologies are ordered to show up the independent subecologies, and if
the {\tt palette} variable (\S\ref{palette}) has been defined, each of
the independent subecology is coloured differently.

The integer array {\em density} is used to mark rows and columns that
have become extinct. The are marked in a wheat colour.

There are two buttons which allow the user to zoom in and out by a
scale factor of two.  One can also zoom in by clicking with the right
mouse button at the location you wish to zoom in on.  The plot can
also be dragged with the left mouse button to change the view.

The namespace for this widget is constructed by appending the 1st
argument to the string ``connect\_''. Any unacceptable characters are
replaced by '\_', so it may take a little experienmtation to find the
correct namespace name.
display supports the following methods:
\begin{itemize}
\item connect\_{\em interaction}::print {\em filename} --- produce a postscript
  representation to a file
\end{itemize}


\subsection{Palette Variable}\label{palette}

This variable is used by the {\tt display}\index{display} and {\tt
  connect\_plot}\index{connect\_plot} and {\tt plot}\index{plot}
commands to define a palette of colours for colouring the species in
the instruments. If the TCL variable {\tt palette}\index{palette} is
assigned a list of X-windows colours (on many systems, a list of such
colours is found in {\tt /usr/lib/X11/rgb.txt}), then the palette
class can be used like an array within C++, returning the colour name
as a string:
\begin{verbatim}
palette[i]
\end{verbatim}
returns the \verb|i%n|th colour, where {\tt n} is the number of
colours in the palette list.

\subsection{Making movies}

Its actually quite easy to make a movie of an Ecolab run. Each of the
base widgets of the instruments has a method called \verb+print+
defined in the \hyperref{widgets namespace}{widgets namespace (\S}{)}{widgets namespace},
which will output a postscript representation of the widget to a
file. See {\tt gen-move.tcl}\index{gen-movie.tcl} for an example,
which produced these \htmladdnormallinkfoot{animated GIFs}
{http://parallel.hpc.unsw.edu.au/rks/ecolab-snaps/example-anim-gifs.html}.

Once a series of postscript files are created, you can convert them
into GIFs using the {\tt pstoimg}\index{pstoimg}\footnote{Requires
  ghostscript, perl and netpbm to be installed} utility that comes
with \htmladdnormallinkfoot{\LaTeX2HTML}
{http://parallel.hpc.unsw.edu.au/htmldocs/latex2html}, using something
like the following\footnote{The current {\tt pstoimg} script doesn't
  seem to work properly in any directory other than /tmp --- I will
  need to fix it and post a copy for \EcoLab{} users}:
\begin{verbatim}
cp *.ps /tmp
for i in *.ps; do pstoimg -gif  $i; done
cp /tmp/*.gif .
\end{verbatim}
%$


Finally, use {\tt gifmerge}\index{gifmerge}\footnote{gifmerge is
  available from a number of open source repositories} to produce an
animated GIF.
\begin{verbatim}
gifmerge *.gif >movie.gif
\end{verbatim}

An alternative is to produce an AVI file using the \verb+mencoder+
software that comes as part of the
\htmladdnormallinkfoot{mplayer}{http://www.mplayerhq.hu}
package. First you need to prepare a collection of jpeg files:
\begin{verbatim}
for i in *.ps; do 
  f=${i%%.ps}
  gs -sDEVICE=ppm -sOutputFile=$f.ppm -dNOPAUSE -dBATCH -g700x400 -r50  $i
  cjpeg $f.ppm >$f.jpg
  done
\end{verbatim}
You can control the final size of the bitmap using the \verb+-g+
option to gs, and control the scale with the \verb+-r+ option
(``pixels per inch''). Mplayer has problems if the bitmap is too
large, and if the dimensions are odd.

Finally, you can create an mpeg1 encoding using mencoder:
\begin{verbatim}
mencoder -mf on:type=jpeg -ovc lavc -lavcopts vcodec=mpeg1video \
              \*.jpg -o movie.avi
\end{verbatim}
Consult the mencoder man page for more details on codec options.

\subsection{Auxilliary Commands}

\subsubsection{get\_vars/data\_server}\label{get_vars/data_server}

Syntax:

{\em model.}{\tt get\_vars}\index{get\_vars} {\em server port}

{\em model.}{\tt data\_server}\index{data\_server} {\em port}

This pair implements a client server connection. {\tt
  data\_server}\index{data\_server}  is
executed on the compute server, and services any requests coming in on
the given port. {\tt get\_vars}\index{get\_vars} attaches to the compute
server, and downloads the compute server's model variable into
the client's model variable. The client can the follow on with the
usual instruments for analysing the data. The advantage in this
approach is that  X-window traffic can be avoided, the total amount of
traffic between client and server can be controlled by how often these
routines are called. An example setup is located in {\tt console.tcl}
and {\tt engine.tcl}. An alternative client/server scenario can be
contructed using Tcl sockets, and the console2.tcl/engine.tcl give an
example of just transferring the \bn{} of the predator-prey
example. This has a great deal of flexibility, allowing, for example,
messages to be propagated from client back to the server to allow user
interactivity into the model

\subsubsection{checkpoint/restart}\label{checkpoint/restart}

Syntax:

{\em model}.{\tt checkpoint}\index{checkpoint} {\em filename}

{\em model}.{\tt restart}\index{restart} {\em filename}

These commands dump the contents of the model's variables into a file,
and correspondingly reload the model's state variables from the file,
in order to implement checkpoint-restart for a batch, or long running
environment. These commands are defined in the \hyperref{{\tt
  TCL\_obj\_t} class}{(\S}{)}{TCL_obj}\index{TCL\_obj\_t}, and can be
overridden if desired.

By setting the variable {\em model}{\tt .xdr\_check}\index{xdr\_check}
to 1 (default value is zero), the checkpoint is written out using XDR
routines, so the checkpoint file can be restarted on a computer with a
different processor.

\subsubsection{Trapping signals}

{\tt trap} {\em signal} {\em command}\\
{\tt trapabort} [off]

Arrange for {\em command} to be executed whenever the signal {\em
  signal} is received. {\em signal} may be specified symbolically (eg
\verb+TERM+ or \verb+XCPU+) or numerically (15 and 24 in previous
example).

{\tt trapabort} arranges for the TCL error handler to be called
whenever a segmentation violation, illegal instruction or bus error
occurs. Often the error is mild enough, that processing can continue
--- it is used in particular with the Object Browser. Specifying {\tt
  trapabort off} turns off this behaviour --- you definitely need this
disabled when running \EcoLab{} in a debugger.

\subsection{ecolab\_version}\index{ecolab\_version}

This reports the version number of the EcoLab system, eg 4.D22. This
version number is also available in the file {\tt version.h} as the
macro {\tt VERSION}\index{VERSION}.

\section{Creating New Instruments}

The best way to proceed is to copy an existing instrument, and modify
it to your needs. The interface to the TCL language can be done almost
entirely through the \hyperref{{\tt tcl++} class library}
{{\tt tcl++} class library (\S}{)}{tcl++}\index{tcl++}. Use the {\tt
  NEWCMD}\index{NEWCMD} macro to instantiate a new TCL
command. This may be the complete widget, or merely a component of it.
The instrument may be encoded entirely in TCL code, accessing the
model's state variables in the usual way.  Alternatively, you may
write C++ code to access the state variables directly through the
model's class defintion. 

An alternative approach is taken in {\tt
  analysis.cc}\index{analysis.cc}. A variable that has been registered
  in the {\tt TCL\_obj} database can have a reference set up to it
  using the {\tt declare}\index{declare} macro, defined in {\tt
  TCL\_obj\_base.h}\index{TCL\_obj\_base.h}. For example, in the
  ecolab model, {\tt ecolab.density} is the TCL name for the {\tt
  density} member of the ecolab model. Then
\begin{verbatim}
declare(density,iarray,"ecolab.density");
\end{verbatim}
declares a variable of type \verb+iarray&+ that may be used to refer to
this member. Argument 3, the string parameter needn't be a constant
string, but can be any string, eg a string passed through {\tt argv[]}.

\section{Creating a New Model}\label{new model}

The code that explicitly defines the \EcoLab{} model is contained in
{\tt ecolab.cc}. Several examples of a completely different models is
provided with the \EcoLab{} distribution. Most of these are described
in different papers:
\begin{description}
\item[shadow] Ecolab model with a neutral shadow model, as described in
  \cite{Standish00c} and \cite{Standish02b}.
\item[newman] A variation of Mark Newman's evolutionary model,
  described in \cite{Standish98a}
\item[webworld] An implementation of Drossel {\em et al.}'s Webworld
  model, described in \cite{Drossel-etal01} and \cite{Standish04a}.
\item[jellyfish] A model of Jellyfish migration in Palauan lakes, in
  collaboration with Mike Dawson. This model will be described in more
  detail in \S\ref{jellyfish}.
\item[netcomplexity] A class providing TCL methods for computing
  network complexities. Used in the study reported in \cite{Standish05a}
\end{description}

A model typically consists of an interface file (.h), which is
processed by Classdesc, an implementation file (.cc) and one or more
experiment scripts (.tcl).

The model should, as far as possible, be implemented as a single
object (which may be a container). Let's call the model JoesFolly. The
model class (eg \verb+JoesFolly_t+) is defined in the interface file in a
regular C++ fashion. It is simplest if all members of the model class
are public, however read the \hyperref{Classdesc chapter}{Classdesc
  chapter (\S}{)}{classdesc}\index{classdesc} for how to handle
private members.

Consider whether your model maps naturally to the notion of objects
related by a network. In that case, you may find that
\hyperref{Graphcode}{Graphcode (\S}{)}{graphcode}\index{graphcode}
will effectively distribute your model across multiple processors of
an MPI parallel job. Both the spatial EcoLab model, and the Jellyfish
model are examples of Graphcode deployment.

Your model class needs to be derived from \verb+TCL_obj_t+. This adds
a few extra methods to your class, such as checkpoint/restart, and
client/server functionality.

\begin{verbatim}
class JoesFolly_t: public TCL_obj_t
{
  public:
   int an_instance_var;
   double a_method(TCL_args);
};
\end{verbatim}

You then define your model object in the implementation file, and pass
this object to the macro \verb+make_model+\index{make\_model}.

\begin{verbatim}
#include <ecolab.h>
#include "JoesFolly.h"
#include "JoesFolly.cd"
#include <ecolab_epilogue.h>

JoesFolly_t JoesFolly;
make_model(JoesFolly);

double JoesFolly(TCL_args args)
{
  double x=args, y=args;
  ...
}
\end{verbatim}

The supplied Makefile in the models directory can be used as a
template for your own project. It contains rules that generate
Makefile dependencies for all include files included with
\verb+""+. It will launch classdesc to generate descriptors for all
class definitions in \verb+JoesFolly.h+. All public instance variables
are visible to TCL, as are methods that take no arguments, or a single
\hyperref{TCL\_args}{TCL\_args (\S}{)}{TCL_obj}\index{TCL\_args} argument.

Of course these instance variables or members are only actual accessible from
TCL if their type \verb+T+ has a \verb+operator<<(ostream&,T)+
defined. Accessing other types of object will result in a runtime error.

Note that Standard C++ requires functions used by templates be
declared prior to templates being defined. Since the .cd files are
declaring functions like pack, TCL\_obj etc., templates that call
these functions must be declared after all .cd files are included,
otherwise the templates will not pick up the definitions in the .cd
files. This is achieved by including
\verb+ecolab_epilogue.h+\index{ecolab\_epilogue.h} after the all the
.cd files have been included in the .cc file. In fact, now \EcoLab{}
insists on the presence of this file if classdesc is used, and will
generate a link time failure if not provided:
\begin{verbatim}
undefined reference to `(anonymous namespace)::classdesc_epilogue_not_included()'
undefined reference to `(anonymous namespace)::TCL_obj_template_not_included()'
\end{verbatim}
\index{classdesc\_epilogue\_not\_included}\index{TCL\_obj\_template\_not\_included}

\section{Error reporting}

\EcoLab{} now uses C++ standard exceptions to report errors --- the
old \verb+longjmp+ mechanism has now gone. When you want to report an
error from within your user written commands and methods, throw an
object of type \verb+error+\index{error}, which is defined in
\verb%tcl++.h%\index{tcl++}. \verb+error+ has a convenient constructor
that works a bit like printf, eg
\begin{verbatim}
  throw error("%d arguments is too many!",nargs);
\end{verbatim}

\EcoLab{} will trap all exceptions in user written routines and
methods, and return a \verb+TCL_ERROR+ result. If its an exception derived
from the standard \verb+exception+ class (which \verb+error+ is), then
it will place the value returned by its \verb+what()+ method as the
result field of the TCL command. Any other caught exception will
report ``Unknown exception caught''.

What happens when \EcoLab{} traps an exception depends on exactly
where it occurs --- if it occurs as a result of executing a script,
the error is reported on standard output, and \EcoLab{} exits with a
non-zero return value (useful for writing test scripts, for
example). If it occurs while the GUI environment is in operation, then
a dialog box pops up, offering information to debug what went
wrong. If it occurs as part of the command line interpreter (ie when
\EcoLab{} is reading its commands from standard input) then the error
result is reported on standard output, and execution continues.

When the DEBUGGING flag set at make time, the error constructor will
call \verb+abort()+, which can be caught by a debugger.

\section{\protect\EcoLab{} coding style}

There is no particular \EcoLab{} coding style in terms of things like
choice of identifier capitalisation, indentation, use of whitespace,
naming schemes and the like. Non-syntactical information is always
misleading, and I can well advise studying the output of doxygen to
work what a particular identifier actually is. For singleton classes,
I have a habit of following the Java convention of capitalising the
class name, and using a lower case for the object. Similarly, for
namespaces, I append a \verb+_ns+.

Within header files, I tend to write more compactly, for instance
placing an opening brace on the previous line, or putting several
simple statements together on one line. In implementation files I tend
to space things out a bit more. Obviously, the idea is to only place
members with a few lines of definition in the interface file,
otherwise migrate the definition to an implementation file.

More important is information that the compiler can use to enforce
correctness and performance. The concept of const-correctness is very
important in determining flows of data dependencies, similarly
exception correctness is important for determining code flow. Use
references instead of pointers whereever possible.

Of crucial importance is a concept known as ``Resource Acquisition Is
Initialisation''. If you need to access some resource such as memory,
process or file, put the resource acquisition into the constructor of
some object, and the corresponding resource release as the
destructor. This has numerous benefits, ranging from eliminating
resource leaks to ensuring exeception-correctness. 

Much C++ code is written in a style I would call C/Java style. Objects
are allocated on the heap using \verb+new+, and users of class
libraries must ensure that the corresponding \verb+delete+ is called
to correctly clean up the opbject when it is destroyed. This style of
programming leads to a whole host of subtle problems that RAII avoids.

\EcoLab{} code tends to assume that objects are default constructible,
copiable, assignable and serialisable (DCAS). These concepts are heirarchical
--- a class composed of members satisfying these criteria, also
satisfies these criteria (or at least can be arranged to saisfy it
through automated technques such as Classdesc). Unless absolutely
necessary, try to ensure classes introduced in your \EcoLab{} model
satisfy DCAS. Pointers are not DCAS, so if you need to use a pointer
(eg to use a library having objects pointed to), then consider
wrapping the pointer in an RAII style. \EcoLab{} provides the
\verb+ref+ class, which is a DCAS type allowing shared
references. \verb+ref+ is not suitable for polymorphic data, however
\EcoLab{} provides \verb+poly{}+ type for handling polymorphic
objects, that is DCAS. The Boost classes \verb+shared_ptr+ and
\verb+intrusive_ptr+ (which are part of the TR1 standard addition to
C++) are DCA, but unfortunately are not serialisable per se. Depending on
your need, you may be better off using the \EcoLab{} \verb+ref+ and
\verb+poly+ data types instead of the boost routines.

\section{The \protect\EcoLab{} Model}\label{model}

The \EcoLab{} model is but one model implemented using the \EcoLab{}
software. This section documents the model itself, and may be skipped
if your intention is to use \EcoLab{} for other models.

We start with a generalised form of the Lotka-Volterra equation 
\begin{equation}\label{lotka-volterra}
\dot{\bn} = \br*\bn + \bn*\bbeta\bn.
\end{equation}
Here \bn\ is the population density, the component $n_i$ being the
number of individuals of species $i$, \br\ is the difference
between reproduction and death, \bbeta\ is the interaction matrix,
with $\beta_{ij}$ being the interaction between species $i$ and $j$, *
referring to elementwise multiplication and {\tt mutate} is the
mutation operator.

\subsection{Lotka-Volterra Dynamics}

The most obvious thing about equation (\ref{lotka-volterra}) is its
fixed point 
\begin{equation}\label{fixed point}
\hat{\bn} = -\bbeta^{-1}\br,
\end{equation}
where $\dot{\bn}=0$. For this point to be biologically meaningful, all
components of $\hat{\bn}$ must be positive, giving rise to the following
inequalities:
\begin{equation}\label{positive species}
\hat n_i = \left(\bbeta^{-1}\br\right)_i>0, \forall i
\end{equation}
The stability of this point is related to the
negative definiteness of derivative of $\dot{\bn}$ at $\hat{\bn}$. The
components of the derivative are given by
\begin{equation}\label{derivative}
\frac{\partial\dot{n}_i}{\partial n_j} =
\delta_{ij}\left(r_i+\sum_k\beta_{ik}n_k\right) + \beta_{ij}n_i
\end{equation}
Substituting eq (\ref{fixed point}) gives
\begin{equation}
\left.\frac{\partial\dot{n}_i}{\partial n_j}\right|_{\hat{\bn}}=
-\beta_{ij}\left(\bbeta^{-1}\br\right)_i
\end{equation}

Stability of the fixed point requires that this matrix should be
negative definite. Since the $\left(\bbeta^{-1}\br\right)_i$ are
all negative by virtue of (\ref{positive species}), each minor
determinant of this matrix is equal to a minor determinant of \bbeta\
multiplied by a positive number, stability of the equilibrium is
equivalent to \bbeta\ being negative definite.

A weaker condition is to require that the system remain bounded with
time:
\begin{equation}\label{boundedness}
\sum_i\dot{n_i}=\br\cdot\bn + \bn\cdot\bbeta\bn < 0, \;\forall \bn:
\sum_in_i>N \;\exists N
\end{equation}

As \bn\ becomes large in any direction, this functional is dominated
by the quadratic term, so this implies that  $\bn\cdot\bbeta\bn\leq0
\; \forall\bn: n_i>0$. Negative definiteness of \bbeta\ is sufficient,
but not necessary for this condition. For example, the predator-prey
relations (heavily normalised) have the following matrix as \bbeta:
\begin{math}
\bbeta=\left(\begin{array}{cc}
-1 & 2\\
-2 & 0\\
\end{array}\right)
\end{math}
which has eigenvalues $3/2, -5/2$. If we let $\bn=(x,y), x,y\geq0$, then
$\bn\cdot\bbeta\bn=-2x^2$, which is clearly non-positive
for all $x$.

Consider adding a new row and column to \bbeta. What is condition is
the new row and column required to satisfy such that equation
(\ref{boundedness}) is satisfied. Break up \bbeta\ in the following
way:
\begin{displaymath}
\left(
  \mbox{
     \begin{tabular}{c|c}
       $\begin{array}{ccc}\ddots\\&{\bf A}\\&&\ddots\end{array}$ & 
       $\begin{array}{c}\vdots\\{\bf B}\\\vdots\end{array}$ \\
       \hline
       $\begin{array}{ccc}\cdots&{\bf C}&\cdots\end{array}$ & D
     \end{tabular}
   }
\right)
\left(
  \mbox{
     \begin{tabular}{c}
     $\begin{array}{c}\vdots\\{\bf n_1}\\\vdots\end{array}$\\
     \hline
     $n_2$
     \end{tabular}
    }
\right)
\end{displaymath}

Condition (\ref{boundedness}) becomes:
\begin{equation}\label{boundedness2}
{\bf n_1}\cdot{\bf A}{\bf n_1} + {\bf n_1}\cdot({\bf B}+{\bf C})n_2 +
Dn_2^2 \leq 0
\end{equation}

Let 
\begin{displaymath}
a=\max_{n=1} \bn\cdot A\bn,\mbox{ and } b=\max_{i}B_i+C_i.
\end{displaymath}
 Then a sufficient but
not necessary condition for condition (\ref{boundedness2}) is
\begin{displaymath}
an_1^2+bn_1n_2+Dn_2^2\leq0
\end{displaymath}

The maximum value with respect to $n_2$ is $an_1^2-(bn_1)^2/4D$, so
this requires that
\begin{equation}\label{boundedness3}
b \geq 2\sqrt{aD}
\end{equation}

\subsection{Mutation}\label{mutation}

With mutation, equation (\ref{lotka-volterra}) reads
\begin{equation}
\dot{\bn} = \br*\bn + \bn*\bbeta\bn + {\tt mutate}(\bmu,\br,\bn).
\end{equation}


The difficulty with adding mutation to this model is how to define the
mapping between genotype space and phenotype space, or in other words,
what defines the {\em embryology}. A few studies, including Ray's
Tierra world, do this with an explicit mapping from the genotype to to
some particular organism property (e.g. interpreted as machine language
instructions, or as weight in a neural net). These organisms then
interact with one another to determine the population dynamics. In
this model, however, we are doing away with the organismal layer, and
so an explicit embryology is impossible. The only possibility left is
to use a statistical model of embryology. The mapping between
genotype space and the population parameters $\br$,
$\bbeta{}$  is expected to look like a rugged
landscape, however, if two genotypes are close together (in a Hamming
sense) then one might expect that the phenotypes are likely to be
similar, as would the population parameters. This I call {\em random
embryology with locality}.

In the simple case of point mutations, the probability $P(x)$ of any
child lying distance $x$ in genotype space from its parent follows a
Poisson distribution. Random embryology with locality implies that the
phenotypic parameters are distributed randomly about the parent
species, with a standard deviation that depends monotonically on the
genotypic displacement. The simplest such model is to distribute the
phenotypic parameters in a Gaussian fashion about the parent's values,
with standard deviation proportional to the genotypic displacement.
This constant of proportionality can be conflated with the species'
intrinsic mutation rate, to give rise another phenotypic parameter
$\bmu$.  It is assumed that the probability of a mutation generating a
previously existing species is negligible, and can be ignored. We also
need another arbitrary parameter $\rho$, ``species radius'', or
\verb+ecolab.sp_sep+,\index{sp\_sep} which can be understood as the
minimum genotypic distance separating species, conflated with the same
constant of proportionality as $\bmu$.

In summary, the mutation algorithm is as follows:
\begin{enumerate}
\item The number of mutant species arising from species $i$ within a
timestep is $\mu_i\alpha_in_i/\rho$. This number is rounded
stochastically to the nearest integer, e.g. 0.25 is rounded up to 1
25\% of the time and down to 0 75\% of the time.

\item Roll a random number from a Poisson distribution
$e^{-x/\mu+\rho}$ to determine the standard deviation $\sigma$ of phenotypic
variation. 

\item Vary $\br$ according to a Gaussian distribution about the
  parents' values, with $\sigma\alpha_0$ as the standard deviation,
  where $\alpha_0$ is the range of values that $\br$ is initialised to,
  ie $\alpha_0$=\verb|ecolab.repro_max|\index{ecolab.repro\_max}$-$
  \verb|ecolab.repro_min|\index{ecolab.repro\_min}

\item The diagonal part of $\bbeta$ must be negative, so vary $\bbeta$
according to a log-normal distribution. This means that if the old
value is $\beta$, the new value becomes
$\beta'=-\exp(\log_e(\beta)+\sigma)$. These values cannot become
arbitrarily small, however, as this would imply that some species make
arbitrarily small demands on the environment, and will become infinite
in number. In \EcoLab{}, the diagonal interactions terms are prevented from
becoming larger than $-r/(.1*{\tt INT\_MAX})$.

\item The offdiagonal components of $\bbeta$, are varied in a similar
fashion to $\br$. However new connections are added, or old ones
removed according to $\lfloor 1/r\rfloor$, where $r\in(-2,2)$ is
chosen from a stepped uniform distribution 
\begin{displaymath}
P(x)=\left\{
\begin{array}{ll}
0.25(1-g) & \mathrm{if}\; x\leq0\\
0.25(1+g) & \mathrm{if}\; x>0\\
\end{array}
\right.
\end{displaymath}
where $g\in[-1,1]$ (default of 0) is specified by the TCL variable
\verb+generalization_bias+\index{generalization\_bias}. The values on
the new connections are chosen from the same initial distribution that
the offdiagonal values where originally set with, ie the range
\verb|ecolab.odiag_min|\index{ecolab.odiag\_min} to
\verb|ecolab.odiag_max|\index{ecolab.odiag\_max}. Since $a$ in
condition (\ref{boundedness3}) is computationally expensive, we use a
slightly stronger criterion that is sufficient, computationally
tractable yet still allows ``interesting'' non-definite matrix
behaviour namely that the sum $\beta_{ij}+\beta_{ji}$ should be
nonpositive.


\item $\bmu$ must be positive, so should evolve according to the
  log-normal distribution like the diagonal components of $\bbeta$.
  Similar to $\bbeta$, it is a catastrophe to allow $\bmu$ to become
  arbitrarily large. In the real world, mutation normally exists at
  some fixed background rate --- species can reduce the level of
  mutation by improving their genetic repair algorithms. In \EcoLab{},
  this ceiling on $\bmu$ is given by the
  \verb|ecolab.mut_max|\index{ecolab.mut\_max} variable.

\end{enumerate}

\subsection{Input Parameters}\label{input parameters}

The model's parameters are set by TCL variables in {\tt model.tcl}.
The actual data structures of the model are initialised the first time
the model's generate step is called. An example input set is:
\begin{verbatim}
# initial condition
ecolab.species {1 2}
ecolab.density {100 100} 
ecolab.create {0 0}
ecolab.repro_rate {.1 -.1}
ecolab.interaction.diag {-.0001 -1e-5}
ecolab.interaction.val {-0.001 0.001}
ecolab.interaction.row {0 1}
ecolab.interaction.col {1 0}
ecolab.migration {.1 .1}


# mutation parameters
ecolab.mutation {.01 .01}
ecolab.sp_sep .1
ecolab.repro_min -.1
ecolab.repro_max .1
ecolab.odiag_min -1e-3
ecolab.odiag_max 1e-3
ecolab.mut_max .01
\end{verbatim}

Model variables define a TCL command of the same name as they appear
in the C++ source. So in the \EcoLab{} model, the C++ object {\tt
  ecolab} defines a set of TCL commands such as {\tt ecolab.density}
that can be used for setting or querying the values of {\tt ecolab}'s
members.  If an argument is specified, then that argument is used to
set the variable's value, otherwise, the variable's value is returned.
Array members in the model are initialised by specifying an TCL list
argument to the variables name, and return TCL lists when no argument
is specified. The above example starts the ecology off with a single
predator and prey (based on {\tt pred-prey.tcl}\index{pred-prey.tcl}).

\subsection{\protect\EcoLab{} Model commands}

\subsubsection{generate}\index{generate}

This implements the basic Lotka-Volterra equations:

\begin{math}
\dot{\bn} = \br*\bn + \bn*\bbeta\bn
\end{math}

with \br\ being the reproduction rate and \bbeta\ being the
interspecies interaction. This is implemented as a single line:

\begin{verbatim}
  density += repro_rate * density + (interaction * density) * density;
\end{verbatim}

This command also increments the timestep counter {\tt tstep}.

An optional argument specifies a number of timesteps to run the
generate step. This improves the speed by amortising the real to
integer conversion operation over a number of timesteps. The downside
is that computation may fail if the problem is ill-conditioned
(offdiagonal elements of \bbeta\ too large with respect to the
diagonal elements).

\subsubsection{condense}\label{condense}\index{condense}

This command compacts the systems of equations by removing extinct
species where $n_i=0$.

\subsubsection{mutate}\label{mutate}\index{mutate}

This applies the point mutations to the system. The precise algorithm
is described in \S\ref{mutation}. 

\subsubsection{migrate}\label{migrate}\index{migrate}

This operator implements migration within cellular \EcoLab{}. This
updates density values according to the difference with the 4 nearest
neighbours: $\bn+=\bgamma*(0.25(\bn_n+\bn_e+\bn_s+\bn_w)-\bn)$, where the
$n,e,s,w$ index the north, east, south and west neigbouring cells.

\subsubsection{maxeig}\label{maxeig}\index{maxeig}

\verb|maxeig| returns the maximum eigenvalue of \bbeta. If this number
is negative, the equilibrium point is stable, if positive, it is
unstable. As reported in Standish (1994)\cite{Standish94}, the mutation
drives the maximum eigenvalue slightly positive, then instabilities
act to push the eigenvalue back to zero. This command requires LAPACK\index{LAPACK}).

\subsubsection{lifetimes}\label{lifetimes}\index{lifetimes}

\verb|lifetimes| records the timestep when a
species passes a threshold (hardwired at 10) in the {\tt create}\index{create}
iarray. If a species has yet to pass the threshold, or has gone
extinct, the value in {\tt create} is zero. Upon return, this routine
returns the lifetime of the species that have gone extinct. This can
then be passed to a histogram routine, or written to a file.

\subsubsection{random\_interaction}\label{random_interaction}
\index{random\_interaction}

Calls \hyperref{{\protect\tt sparse\_mat::init\_rand()}}{(See \S}{)}{sparse_mat} to randomly initialising the nonzero pattern of
the offdiagonal elements. The average number of nonzeros per row is
{\tt conn}, and the standard deviation of the number of nonzeros is
{\tt sigma}.

\subsubsection{set\_grid}

Synopsis

{\tt ecolab.set\_grid} {\em x} {\em y}

\noindent Set up an $x\times y$ grid in spatial \EcoLab. See
\verb+ecolab_spatial.tcl+ for an example using this.

\subsubsection{get}

Synopsis

{\tt ecolab.get} {\em x} {\em y}

\noindent Create TCL method for accessing the internals of cell $x$
$y$. The new commands look like array elements, eg
\begin{verbatim}
ecolab(1,0).density
\end{verbatim}

\subsubsection{forall}

Synopsis

{\tt ecolab.forall ecolab.}{\em command} {\em args}

Run {\em command} on all cells.

\subsection{Spatial Variation}\label{spatial}

The ecolab model can run in multicellular mode by calling
\verb+ecolab.set_grid+ from TCL, specifying the dimensions of the
grid.\index{set\_grid}. See
\verb+ecolab_spatial.tcl+\index{ecolab\_spatial} for an example.

Only population density varies between the cells --- all other
variables are members of the ecolab variable so can be set or queried
in the usual way.

The usual ecolab model methods (generate, mutate, condense and
lifetimes)\index{generate}\index{mutate}\index{condense}\index{lifetimes}
can now be called, but operate on the entire grid. A new {\tt
migrate}\index{migrate} is defined to handle migration between
cells. You can also call a method of the ecolab cell on all cells
using the \verb+forall+\index{forall} command. For instance, to set
all cells to same initial density, use:
\begin{verbatim}
ecolab.forall ecolab.density [constant $nsp 100]
\end{verbatim}

Access to the individual cells can obtain by creating a TCL\_obj to
refer to it by using the \verb+ecolab.get+\index{get} method, which
creates commands like {\tt ecolab({\em x},{\em y})} to refer to the
cell. These can be fed to visualisers in the usual way.

\subsubsection{Parallel Execution}

Since the cells are pins in a \hyperref{graphcode Graph}{graphcode
  Graph (\S}{)}{graphcode}\index{graphcode}, they are distributed over
parallel processes if available.

Remember to call the \verb+gather+\index{gather} method to ensure node
0 is updated before running a visualiser on global data.

\begin{description}
\item[ecolab.gather] Bring processor 0's data up to date with the rest
  of the grid
\item[ecolab.distribute\_cells] Broadcast processor 0's data out to the
  rest of the grid.
\end{description}

\section{Palauan Jellyfish model}\label{jellyfish}

This is a model being developed by Mike Dawson and Russell Standish to
model the behaviour of jellyfish in a number of lakes on the island of
Palau. Jellyfish are photosynthetic animals, so have have a preference
for the sun and avoiding shadows. In this model, the jellyfish are
represented by agents that have a position and velocity. If the
jellyfish moves into shadow, or bumps into the side of the lake, it
will reverse its direction. From time to time, the jellyfish will
change direction and speed. The random generators governing these are
selectable at runtime through the experimental script. Also, jellyfish
do bump into each other. In the model display, a jellyfish will flash
green if it bumps into another one.

To run the jellyfish model, run the jellyfish.tcl script (located in
the models directory), specifying the lake as an argument, eg:
\begin{verbatim}
jellyfish.tcl lakes/OTM
\end{verbatim}

You can choose whether to compile the 2D version of the model or the
3D version, by (not) defining the preprocessor flag
\verb+THREE_D+\index{THREE\_D} (see Makefile).

The lake itself is represented by a Tk pixmap, with the blue component
representing water. The lake shapes were scanned into a GIF file, and
edited with a run-of-the-mill paint program to produce the lakes.

Visualising the lake involved creating a Tk canvas widget, displaying
the lake image in it, then overlaying it with shadow lines extending
from the pixels lying on the boundary, and finally representing the
jellyfish with arrow symbols (to indicate position and velocity).

This model illustrates the use of probes. Mouse clicks in the canvas
region are bound to a short method that determines which agent is
closest to the mouse position. It then colours that agent red (for
tracking purposes), creates a TCL\_obj representing that agent, and
returns the name back to TCL. TCL then calls the object browser
(\S\ref{object-browser}) on that TCL\_obj. In all, 14 lines of C++
code, and 3 lines of TCL code. The result is very effective.

The jellyfish model is written to be run in parallel using
\hyperref{Graphcode}{Graphcode
  (\S}{)}{graphcode}\index{graphcode}. The strategy is effectively a
{\em particle-in-cell} method. The lake is subdivided into a Cartesion
grid of cells, and each Jellyfish only needs to consult the cell that
it is in, as well as neighbouring cells to determine if it will
collide with any other jellyfish.


\section{Classdesc}

\input{classdesc-common}

\subsection{TCL\_obj}\label{TCL_obj}

The {\tt TCL\_obj}\index{TCL\_obj} function creates a set of TCL commands that
implement get/set functionality for the class members. For example,
with a class definition:
\begin{verbatim}
class foo: public TCL_obj_t {int a, b; void foobar(int,char**)} bar;
\end{verbatim}
\verb+TCL_obj(&bar,"bar",bar);+ creates the TCL commands {\tt bar.a}
and {\tt bar.b}. To set the value of {\tt bar.a}, use the command {\tt
  bar.a} {\em val} from TCL. To get the value, use {\tt [bar.a]}.

Also created is the TCL command {\tt bar.foobar}, which will run
respective member function of {\tt foo} when called from TCL.

Any nonoverloaded member function can be accessed from TCL, provided
the arguments and return types can be converted from/to TCL
objects. In particular, it is not possible at present to call methods
that take nonconstant references.

Overloaded method types in general cannot be called, but it is
possible to create variable length argument lists by declaring a
method with an \verb+(int,char**)+, or a \verb+(TCL_args)+
signature. Such methods are not easily called from C++, and generally,
one needs to define a set of overloaded functions of a different name
(eg capitalised) suitable for calling
from C++, as well as the variable length argument list for use from
TCL. However, as a special case of an accessor (emulating the
setting/getting of an member attribute), one may make use of the
Accessor\index{Accessor} class, which is equally callable from C++ as
TCL.

Accessor is not easily usable from within the C++98 language standard
(see acessor.h in the test directory), but makes much more sense in
the C++11 standard. For example, assume that \verb+Name()+ and
\verb+Name(const string&)+ have been defined as a getter and setter
method of the attribute \verb+name+, then one may define a member
\begin{verbatim}
Accessor<string> name {
    [this](){return this->Name();}, 
    [this](const std::string& x){return this->Name(x);}
};
\end{verbatim}
where the use of lambdas and brace initialisers makes it easy to
assign code for the getter and setter components of the accessor. This
member will be accessible as an attribute from TCL (just as if name
had been defined as a string member), and also callable from C++ as
\verb+name()+ or \verb+name("someName")+ as appropriate.

One downside of the Accessor class is that it is not copy
constructible, as copying the accessor will copy a reference to the
wrong accessed object. Consequently, if copy construction is required
for the object being accessed (eg for DCAS), then a custom copy
constructor needs to be provided.

As an alternative to \verb+(int,char**)+ arguments for implementing
TCL commands, one can also declare the arguments
\verb+(TCL_args)+\index{TCL\_args}. \verb+TCL_args+ variables can be
assigned to numerical variables or strings variables, and the
appropriate argument is popped off the argument stack:
\begin{verbatim}
int x=args;
double y=args;
\end{verbatim}
assigns the first argument to x and the second to y. This method use
the \verb+Tcl_Obj+ structure, so the values needn't be converted to
strings at all.

The arguments may also be assigned using the \verb+>>+ operator:
\begin{verbatim}
int x; double y;
args >> x >> y;
\end{verbatim}

A third style uses the \verb+[]+ operator:
\begin{verbatim}
int x=args[0]; double y=args[1];
\end{verbatim}
The number of remaining arguments is available as
\verb+TCL_args::count+.

If \verb+operator>>(istream,T)+ is defined, then you may also use the
\verb+>>+ operator to set a variable of type \verb+T+, eg:
\begin{verbatim}
void foo::bar(TCL_args args)
{
  iarray x;
  args>>x;
}
\end{verbatim}
the assignment operator cannot be used for this purpose, unlike simple
types, because nonmember assignment operators are disallowed in the
standard. Type conversion operators do not appear to work.

For technical reasons, the name of the TCL command is available as
\verb+args[-1]+. 

The {\tt TCL\_obj\_t} data type also implements \hyperref{checkpoint
  and restart functions}{checkpoint and restart
  functions (\S}{)}{checkpoint/restart},\index{checkpoint}\index{restart}
so that any class inheriting from {\tt TCL\_obj\_t}\index{TCL\_obj\_t}
also gains this functionality, as well as \hyperref{client-server
  functionality}{\S(}{)}{get_vars/data_server}.
\index{data\_server}\index{get\_vars}

A helper macro that performs the above is {\tt
  make\_model}\index{make\_model}, which is used in a declarative
sense, which also initialises the checkpoint functor.


Associated with each of these TCL commands, is an object of type
\verb+member_entry<T>+\index{member\_entry}. Pointers to these objects
are stored in a hash table called
\verb+TCL_obj_properties+\index{TCL\_obj\_properties}. The STL hash
table behaved rather stangely when used for this purpose, so a class
wrapper around TCL hash tables was employed instead:
\begin{verbatim}
template<class T>
struct TCL_obj_hash
{
  struct entry 
  {
    entry& operator=(const T& x);
    operator T();
  };
  entry operator[](const char*x);
};
\end{verbatim}
So objects of \verb+member_entry<T>*+ can be inserted into the hash
table as follows:
\begin{verbatim}
member_entry<T>* m; eco_string d;
TCL_obj_properties[d]=m;
\end{verbatim}
but to extract the data, use \verb+memberPtrCasted+
\begin{verbatim}
if (T* m=TCL_obj_properties[d]->memberPtrCasted<T>())
   ... *m 
\end{verbatim}
will allow you to access the TCL object \verb+d+, if it is castable to
an object of type \verb+T+ (is a \verb+T+, or is derived from a \verb+T+). 

A utility macro\index{declare} allows these objects to be accessed simply:
\begin{verse}
{\tt declare}({\em name},{\em typename}, {\em tcl\_name})
\end{verse}
where {\em name} is the name of a variable (in reality a reference),
of type {\em typename} that will refer to the variable having the TCL
handle {\em tcl\_name}. The macro performs error checking to ensure
such a variable actually exists, and that it is of the same type as
{\em typename}.

Objects can be placed into {\tt TCL\_obj\_properties} by a several
different means:
\begin{enumerate}
\item {\tt make\_model}\index{make\_model}{\em (x)}, which places all
  of the leaf objects of {\em x} (which must be derived from
  \verb+TCL_obj_t+) into {\tt TCL\_obj\_properties}, and also
  completes the construction of the \verb+TCL_obj_t+ object;
\item {\tt register}{\em (x)}\index{register}, which places {\em x} into {\tt
    TCL\_obj\_properties}, as well as the leaf objects --- can also be
    called as \verb+TCL_obj_register(+{\em object},{\em object name});
\item {\tt TCLTYPE}{\em (typename)}\index{TCLTYPE}\label{TCLTYPE},
TCLPOLYTYPE(typename, interface), where {\em
    typename} is defined C++ type, and interface is a base class of typename. This creates the TCL command {\em
    typename}, which takes one argument, a variable name for it to be
  referred to from TCL, and creates an object of that type which it
  registers in {\tt TCL\_obj\_properties}. If {\tt TCLPOLYTYPE} is
  used, the base class type is used for registration - so this object
  can be used wherever a polymorphic type with the specified interface
  is expected. For example, consider the
  following code which creates and initialises an object of type
  distrand and gives it the TCL name PDF (from
  testdistrand.tcl\index{testdistrand.tcl}):\index{distrand}
\begin{verbatim}
distrand PDF
PDF.min -10
PDF.max 10
PDF.nsamp 100
PDF.width 3
PDF.Init dist
.....
PDF.delete
\end{verbatim}
  This macro also defines an x.delete\index{delete} procedure for
  deleting that object, once no longer desired.
\end{enumerate}

A TCL registered object, particularly dynamically created
\verb+TCLTYPE+ objects can be assigned to a member of type
\verb+TCL_obj_ref+\index{TCL\_obj\_ref}. This is particularly useful
for random number generators:

\begin{verbatim}
class Foo: public TCL_obj_t
{
 public:
   TCL_obj_ref<random_gen> rng;

   ...
     rng->rand();
};
\end{verbatim}

Then the member \verb+Foo::rng+ can be assigned an arbitrary random number
generator within the TCL script, such as the PDF example above.

\begin{verbatim}
distrand PDF
PDF.min -10
...
foo.rng PDF
...
\end{verbatim}

Using \verb+TCL_obj_ref+ also allows that object to be serialised, and
to be reconnected after a restart, provided the object has been
created prior to the restart.

\subsection{Member hooks}\label{Member hooks}\index{Member\_entry\_hook}\index{Member\_entry\_thook}

\verb+TCL_obj_t+ has the following two members, that allow one to
assign a hook that is called after every TCL\_obj call into C++
code. This can be used, for example, to record what methods have been
called on the C++ model. There is one for each type of method
signature called from TCL.

\begin{verbatim}
struct TCL_obj_t
{
    typedef void (*Member_entry_hook)(int argc, CONST84 char **argv);
    Member_entry_hook member_entry_hook;
    typedef void (*Member_entry_thook)(int argc, Tcl_Obj *const argv[]);
    Member_entry_thook member_entry_thook;
};
\end{verbatim}

\subsection{TCL\_obj\_stl}\label{TCL_obj_stl}

The header file \verb+TCL_obj_stl+ provides \verb+TCL_obj+ support for
STL containers. If the \verb+value_type+ of an STL container (vector, deque
or list) or set is streamable to an iostream, then it is possible to
directly access the elements of the container as a simple list:

\begin{verbatim}
std::vector<int> vec;
make_model(vec);
   ...
vec {1 2 3}
set vec_elems [vec]
\end{verbatim}
If the \verb+value_type+ is not streamable, an exception will be thrown. This
feature makes the \verb+#members+ functionality of sets redundant.

The following TCL procedures are defined for the
following STL containers, which can be used from a TCL script or the
object browser to manipulate STL container objects. Procedures that do
not call member names are prefixed by the ``@'' symbol, which is a
valid identifier character in TCL, but is not a valid C++ identifier
character. This avoids any possible clash of member names.

\begin{description}
\item[vector, dequeue, list]\mbox{}
  \begin{description}
  \item[@is\_vector] A ``do nothing'' command, the presence of which
    indicates the object is a vector. @elem is more efficient in this case
  \item[@is\_sequence] A ``do nothing'' command, the presence of which
    indicates the object complies with the sequence concept.
  \item[size] returns the size of the vector
  \item[@elem] takes one argument, the {\em index} of an element. It creates a
    TCL command {\em name}({\em index}) that can be used in the usual
    way to access or modify the element's value.
  \end{description}

\item[set, map]\mbox{}
  \begin{description}
  \item[@is\_set] A ``do nothing'' command, the presence of which
    indicates the object complies with the set concept.
  \item[@is\_map] A ``do nothing'' command, the presence of which
    indicates the object complies with the map concept. @elem can be
    used to lookup elements by key.
  \item[size] Return number of entries in the set or map
  \item[count] Takes a single argument, and returns 1 or 0, according to
    whether that argument present within the set or map (as a member or
    key respectively).
  \item[\#members] Returns list of members of a set
  \item[\#keys] Returns list of keys of a map
  \item[@elem] Returns a TCL command name for accessing individual
    elements of a set or map. In the case of a set, the command accesses
  the $i$th element of the set. In the case of a map, the argument can
  be an arbitrary string (so long as it converts to the key type of
  the map), that can be used to address the map element. For example,
  if the map is \verb+map+<string,string>+, one can create an element
  \verb+m["hello"]="foo"+ by means of the following TCL commands:
\begin{verbatim}
m.@elem hello
m(hello) foo
\end{verbatim}
  \end{description}
\end{description}

\subsubsection{Extending TCL\_obj\_stl for nonstandard containers}

\verb+TCL_obj_stl.h+ uses the
\hyperref{\verb+classdesc::is_sequence+}{(see \S}{)}{is_sequence} and
\hyperref{\verb+classdesc::is_associative_container+}{}{}{is_associative_container}
type trait. This means that to enable TCL\_obj handling of your custom
container \verb+MyContainer+, you need to define the appropriate type
trait prior to including \verb+TCL_obj_stl.h+, eg:
\begin{verbatim}
namespace classdesc
{
  template <> struct is_sequence<MyContainer>: public true_type {};
}
\end{verbatim}


Also, if your custom container is more of a map, then you also need to
define the \verb+ecolab::is_map+\index{ecolab::is\_map} type trait for
it's \verb+value_type+, eg
\begin{verbatim}
namespace classdesc
{
  template <> struct is_associative<MyContainer>: public true_type {};
}
namespace ecolab
{
  template <> struct is_map<MyContainer::value_type>: 
    public is_map<std::pair<MyContainer::key_type, MyContainer::mapped_type> > {};
}
\end{verbatim}



\include{parallel}
\section{Graphcode}\label{graphcode}
\psection{Graph}

A {\em Graph}\index{Graph} is a container of references to
\hyperref{{\em objects}}{ (\S}{)}{object}\index{object}
(called \hyperref{{\em objrefs}}{
    (\S}{)}{objref})\index{objref} that may be linked to an arbitrary
  number of other objects. The objects themselves may be located on
  other processors, ie the Graph may be distributed. Objects are
  polymorphic --- the only properties Graph needs to know is how
  create, copy, and serialise them, as well as what other objects they
  are linked to.

Because the objects are polymorphic, it is possible to create
hypergraphs. Simply have two types of object in the graph --- {\em
  pins} and {\em wires}, say. A pin may be connected to multiple wire
objects, just as wires may be connected to multiple pins.

The objrefs themselves are stored in a maplike object called an
\hyperref{{\em omap}}{ (\S}{)}{omap}, which is replicated
across all processors. 

A short synopsis of Graph is as follows:
\begin{verbatim}
class Graph: public Ptrlist
{
public:
  omap objects;

  Graph& operator=(const Graph&);
  Graph(Graph&);
  Graph();

  /* object management */
  objref& AddObject(object* o, GraphID_t id, bool managed=false); 
  template <class T>
  objref& AddObject(GraphID_t id); 
  template <class T>
  objref& AddObject(const T& master_copy, GraphID_t id); 

  /* these methods must be called on all processors simultaneously */
  void Prepare_Neighbours(bool cache_requests=false);
  void Partition_Objects();
  void Distribute_Objects();
  void gather();

  void rebuild_local_list();   
  void clear_non_local()
  void print(int proc) 
};                 

\end{verbatim}

%  void DelObject(GraphID_t id);

\begin{description}
\item[Ptrlist](see \S\ref{Ptrlist})\index{Ptrlist} is a list of
  references to \hyperref{{\em objrefs}}{
    (\S}{)}{objref})\index{objref}, pointing to objects stored locally
  on the current processor.
\item[AddObject]\index{AddObject} In the first form, add an already
  created object to the Graph. In the second form create a new object
  of type {\em T}, and add it to the Graph. {\em T} must be derived
  from the abstract base class \verb+object+\index{object}. You must
  explicitly supply the type of the object to be created as a template
  argument:
  \begin{verbatim}
    g.AddObject<foo>(id);
  \end{verbatim}
  In the third form, create a new object, and initialise its data with
  the contents of argument \verb+master_copy+.
%\item[DelObject] removes an object from the Graph, destroying all
%  links to that object.
\item[Prepare\_Neighbours()]\index{Prepare\_Neighbours} For each
  object on the local processor, ensure that all objects connected to
  it are brought up to date, by obtaining data from remote processors
  if necessary. If the network structure has not changed since the
  last call to this method, set the flag
  \verb+cache_requests+\index{cache\_requests} to \verb+true+, which
  substantially reduces the amount of interprocessor communication
  required.
\item[Partition\_Objects()] \index{Partition\_Objects}Call the ParMETIS
  partitioner to redistribute the graph in an optimal way over the
  processors. ParMETIS executes in parallel, and requires that the
  objects be distributed before this call. One way of achieving this is
  to make a simple assignment of objects to processors (by setting the
  \verb+proc+ member of each objref), then call \verb+Distribute_Objects()+.
\item[Distribute\_Objects()]\index{Distribute\_Objects} Broadcast graph data
  from processor 0, and call \\\verb+rebuild_local_list()+ on each
  processor.
\item[gather()]\index{gather} Bring the entire graph on processor 0 up
  to date, copying information from remote processors as necessary. A
  \verb+gather()+, followed by \verb+Distribute_Pins()+ brings all
  processors' graphs up-to-date. This is, naturally, an expensive
  operation, and should be done for startup or checkpointing purposes.
\item[rebuild\_local\_list()]\index{rebuild\_local\_list} Reconstruct
  the list of objrefs local to the current processor, according to the
  \verb+proc+\index{proc} member of the objrefs.
\item[clear\_non\_local()]\index{clear\_non\_local()} Nullify all
  objrefs that don't belong to the current processor. This can be used
  to save memory usage.
\end{description}

\psubsection{Basic usage of Graph}

{\em Graph} is designed to be used in a SPMD parallel environment,
using MPI to handle messages between processors. A copy of the Graph
object is maintained on each process. Each process has a copy of the
objref database (of type \verb+omap+), called
\verb+GRAPHCODE_NS::objectMap+. The \verb+Graph::objects+ reference
refers to this database. However the payload pointer of each objref
will tend to only point to an object if the object is located in the
current processes address space, or a cached copy of the remote object
is needed for some reason. Otherwise it may be set to NULL to save
space.

To call a method \verb+foo()+ on all objects of a Graph \verb+g+ (in
parallel), execute the following code:
\begin{verbatim}
for (Graph::iterator i=g.begin(); i!=g.end(); i++)
   (*i)->foo();
\end{verbatim}

If the method \verb+foo+ needs to know the values of neighbouring
nodes, then you may call \verb+Graph::Prepare_Neighbours()+, which
ensures that a cached copy of any remotely located node linked to a
local nodes is retrieved from the remote node. Thus arbitrary
communication patterns can be expressed simply by the form of the
network structure of the Graph.

\psection{Objects}\index{object}\label{object}

\verb+object+ is an abstract base class, from which all objects stored
in the graph must be derived. A synopsis of the ABC is:
\begin{verbatim}
  class object: public Ptrlist
  {
  public:
    /* serialisation methods */
    virtual void lpack(pack_t *buf)=0; 
    virtual void lunpack(pack_t *buf)=0;
    /* virtual "constructors" */
    virtual object* lnew() const=0;  
    virtual object* lcopy() const=0;  
    virtual ~object() {}
    virtual int type() const=0;     /* return index into archetype table */
    /* partition weightings - redefine in derived type if needed */
    virtual idxtype weight() const {return 1;}
    virtual idxtype edgeweight(const objref& x) const {return 1;}
  };
\end{verbatim}

\psubsection{Serialisation methods}

The first two virtual methods allow Graphcode to access Classdesc
generated serialisation routines. Assuming you have declared a class
foo as follows:\index{lpack}\index{lunpack}

\begin{verbatim}
class foo: public object
{
  ...
  virtual void lpack(pack_t *buf);
  virtual void lunpack(pack_t *buf);
}
\end{verbatim}

Then you may define the virtual functions as follows:
\begin{verbatim}
inline void pack(pack_t *,eco_string,foo&);
inline void unpack(pack_t *,eco_string,foo&);
inline void foo::lpack(pack_t *buf) {pack(buf,eco_string(),*this);}
inline void foo::lunpack(pack_t *buf) {unpack(buf,eco_string(),*this);}
\end{verbatim}

The definitions for \verb+pack(,,foo&)+ and \verb+unpack(,,foo&)+ will
then be created in the usual way by Classdesc.

It is important that \verb+pack(,,foo&)+ and \verb+unpack(,,foo&)+ be explicitly
declared before use, otherwise a default template function will be
linked in which will not work as expected. See the note on using
polymorphic objects under Classdesc.

\psubsection{Virtual Constructors}

Defining the virtual constructors for your objects type is also a
simple matter. Unlike the case of the serialisation routines, they can
even be done inline in the class definition:\index{lnew}\index{lcopy}
\begin{verbatim}
class foo: public object
{
 ...
 virtual object *lnew() {return vnew(this);}
 virtual object *lcopy() {return vcopy(this);}
};
\end{verbatim}

\psubsection{Run Time Type Identification}

To migrate an object from one thread to another, Graphcode needs to be
able to create an object of the correct type in the destination
address space. This is achieved by means of a {\em run time type
  identification} (RTTI)\index{RTTI} system. Given a type token
\verb+t+, an object of that type can be created by the call:\index{archetype}\index{lnew}
\begin{verbatim}
object *o=archetype[t]->lnew();
\end{verbatim}

Instead of using C++'s built-in RTTI system, where tokens are compound
objects of somewhat indeterminate size, Graphcode implements a simple
RTTI system using template programming, in which a type token is a
simple unsigned integer. This implies that each type of object
to be used with Graph must be registered first, before use. This is
taken care for you automatically if you use \verb+Graph::AddObject()+ to
add your object to the Graph.

Adding the virtual type method to your class is also easy:\index{type}



\begin{verbatim}
class foo: public object
{
 ...
 virtual int type() {return vtype(*this);}
};
\end{verbatim}

The first \verb+vtype+\index{vtype} is called on an object, an object
of that type is created (via its \verb+lcopy+ method), and added to
the archetype\index{archetype} vector. The index of that object within
the archetype vector become the type token. {\em It is vitally
  important that types are added to the archetype vector in the same
  order on all threads.} Clearly this is a trivial requirement if only
one type is used, but slightly more care needs to be taken in the case
of multiple types of object.

If you have multiple object types, consider using the
\verb+register_type+ template to ensure a consistent type registration
across the different address spaces.

\psubsection{Node and edge weights}

By default, the graph partitioning algorithm used in Graphcode weights
each node and link equally. However, it is possible to perform load
balancing by specifying a computational weight function on each node,
and a communication weight function for each edge. For example:
\begin{verbatim}
class foo: public object
{
  ...
  virtual idxtype weight() {return size()*size();}
  vitural idxtype edgeweight(const objref& x) {return (*x)->size();}
};
\end{verbatim}

\psubsection{Edge list}

An object is derived from \hyperref{Ptrlist}{
  (\S}{)}{Ptrlist}\index{Ptrlist}, which contains a list of objrefs
that the current object is connected to.

\psubsection{Self linking nodes}

If there is any reason for your node to access its objref (eg to find
out its GraphID, for example), then you can add the objref to its edge
list (say the first item on the edgelist by convention). Then you can
refer to things like \verb+begin()->ID+, \verb+begin()->proc+ etc.

The \verb+Graph::Prepare_Neighbours()+ and
\verb+Graph::Partition_Objects()+ methods ignore self-linking edges.

\psection{objref}\index{objref}\label{objref}

For every object in the Graphcode system, there is an \verb+objref+ on
every processor referring to it.

Synopsis:
\begin{verbatim}
  class objref
  {
  public:
    GraphID_t ID;
    unsigned int proc;

    objref(GraphID_t id=0, int proc=myid(), object *o=NULL);
    objref(GraphID_t id, int proc, object &o); 
    objref(const objref& x);
    ~objref();

    objref& operator=(const objref& x); 
    object& operator*();
    object* operator->();
    const object* operator->() const;
    const object* operator->() const;
    void addref(object* o, bool mflag=false); 
    bool nullref() const;
    void nullify();
  };
\end{verbatim}

\begin{description}
\item[ID] A unique integer value that identifies the object within a Graph
\item[proc] The location of the active copy of the object
\item[{\tt *, ->}] Dereferencing an objref allows one to access the
  object. It is an error to dereference a nullified objref.
\item[addref] Add an object to this reference. The \verb+mflag+
  parameter indicates whether the object is managed (created by \verb+new+ or
  \verb+lnew()+), and can be safely \verb+deleted+ when nullified, or
  is an external object that should not be deleted. An objref can also
  be instantiated already pointing to an unmanaged object via its constructor.
\item[nullref] whether this objref points to an object or not. If the
  active copy is on this processor, then nullref should be false.
  Otherwise, it may true or false depending on whether this processor
  has a cached copy of the object.
\item[nullify] Remove the object from the Graph (but leaving the
  objref in place). It is an error to remove the active copy, without
  replacing it with another object.
\end{description}


\psection{Ptrlist}\index{Ptrlist}\label{Ptrlist}

Ptrlists work a bit like \verb+std::vector<objref>+ --- objrefs can be
added with \verb+Ptrlist::push_back()+, indexed with standard array
indexing \verb+Ptrlist::operator[]+, and iterated over in the usual
way with \verb+Ptrlist::iterator+. However, unlike
\verb+std::vector<objref>+, only pointers to the objref is stored
within \verb+Ptrlist+, not copies.

Synopsis:

\begin{verbatim}
class Ptrlist 
  {
  public:
    
    class iterator
    {
    public:
      objref& operator*();
      objref* operator->();
      iterator operator++();
      iterator operator--();
      iterator operator++(int);
      iterator operator--(int);
      bool operator==(const iterator& x);
      bool operator!=(const iterator& x);
    };
    iterator begin() const;
    iterator end() const;
    objref &    front ()
    objref &    back ()
    unsigned size() const;
    objref& operator[](unsigned i) const; 
    void push_back(objref* x);
    void push_back(objref& x);
    void erase(GraphID_t i);
    void clear();
    Ptrlist& operator=(const Ptrlist &x);
  };
\end{verbatim}

Ptrlists can only refer to objects stored in objectMap.
Ptrlists can be serialised --- Ptrlists must be unpacked within the
context of an omapthe objectMap.

If you need to use a backing map, you can declare another omap object
and assign objectMap to it. This will create copies of all the objects
contained within objectMap.

\psection{omap}\index{omap}\label{omap}

An omap is a container for storing \hyperref{{\em objrefs}}{
  \S(}{)}{objref}\index{objref}, indexed by ID.

\begin{verbatim}
  class omap: public MAP
  {
  public:
    objref& operator[](GraphID_t i);
    omap& operator=(omap& x);
  };
\end{verbatim}

There are a few different possible ways of implementing omaps, with
differing performance characteristics.  Graphcode provides two
different models, {\em vmap} and {\em hmap} that may be readily
deployed by the user, however users can fairly easily provide their
own implementation if desired. Different implementations can be
selected by defining the \verb+MAP+ macro\index{MAP} to be the desired
omap implementation before including \verb+graphcode.h+. This will
declare everything in the namespace
\verb+graphcode_vmap+\index{graphcode\_vmap} or
\verb+graphcode_hmap+\index{graphcode\_hmap} as appropriate. Using
this scheme, it is possible to have two different omap types in the
one object file, by including graphcode.h twice. However, if you do
this, you will need to \verb+#undef GRAPHCODE_H+ guard variable prior
to subsequent includes.

vmap is intended for use with contiguous GraphID ranges. If there are
holes in the identifier range, then the iterator will return invalid
references for these holes, and the size() method will be incorrect.

If you need to have non-contiguous ID ranges (perhaps for dynamic
graph management --- note this is not currently supported), then please
use the hmap\index{hmap} implementation instead (which will have some performance
penalty).


MAP must provide the following members:
\begin{verbatim}
class MAP
{
  protected:
    objref& at(GraphID_t i);
  public:
    MAP();
    MAP(const MAP&)
    class iterator
     {
       iterator();
       iterator(const iterator&);
       iterator& operator=(const iterator&); 
       iterator operator++(int);
       iterator operator++();
       iterator operator--(int);
       iterator operator--();
       bool operator==(const iterator& x) const;
       bool operator!=(const iterator& x) const;
       objref& operator*();
       objref* operator->();
     };
    iterator begin();
    iterator end();
    unsigned size();
}
\end{verbatim}
   
The \verb+at+\index{at} method is essentially a replacement for \verb+operator[]()+. A
simple example of an omap implementation is provided by \verb+vmap+\index{vmap}:
\begin{verbatim}
class vmap: public std::vector<objref>
{
protected:
  objref& at(GraphID_t i) 
  {
    if (i>=size()) resize(i+1);
    return std::vector<objref>::operator[](i);
  }
};
\end{verbatim}

hmap is a hash map implementation. With all hash maps, performance of
the map is critically dependent upon the choice of hash function, which
is application dependent. hmap is simply defined as:
\begin{verbatim}
class hmap: public hashmap<simple_hash> {};
\end{verbatim}
You can provide your own omap definition (umap, say), with your own
user defined hash function in the following way:
\begin{enumerate}
\item Create a file ``umap'' somewhere in the default search path with
  the following:
\begin{verbatim}
#include "hashmap.h"
struct myhash
{
  unsigned operator()(GraphID_t i) {...}
};
class umap: public hashmap<myhash> {};
\end{verbatim}
\item Add the new omap definitions to the Graphcode library:
\begin{verbatim}
make MAP=umap
\end{verbatim}
\item Include the declarations of the \verb+graphcode_umap+ namespace
  in your application source file:
\begin{verbatim}
#define MAP umap
#undef GRAPHCODE_H
#include <graphcode.h>
\end{verbatim}
\end{enumerate}

\psubsection{Note on using macros to parametrise omap}

It might seem obvious that omap could be made a templated type, with a
single template parameter being the implementation of the
omap. Unfortunately, omaps contain objects, which in turn have
Ptrlists (the edge list), which used to maintain a reference to an
omap. So objects and objrefs must be similarly parametrised by by the
omap implementation type. However the omap implementation type will
also need to reference the parametrised objrefs, leading to an
implementation type parametrised by itself, an unparsable situation.

The way out of this dilemma is to make omap an abstract base class,
which can be made into a concrete implementation as part of the
definition of Graph. However, this introduces an extra level of
overhead in calling virtual functions when performing object
lookup, which is not insignificant. It also dramatically increases the complexity of coding
\verb+object::iterator+, which also would need to be an abstract base
class.

In view of this, the macro based solution finally chosen, seemed the
cleanest, and most efficient means of implementing different omap
implementations.

Obviously, this reason is no longer relevant, however there doesn't seem to
be any burning reason to change the macro parametrisation to a
template argument.

\psection{Building Graphcode}

In building graphcode, you configure the options you want by setting
Make variables on the command line:
\begin{description}
\item[MPI=1] Build the MPI version of Graphcode
\item[MAP=] specify which map to use for omap. hmap is the
  default. You can build a library supporting multiple different omap
  types by issuing successive make commands:
\begin{verbatim}
make MAP=vmap
make MAP=hmap
\end{verbatim}
\item[DEBUGGING=1] Used to enable assertions, as well as debugger
  symbols. Optimisation is turned off
\item[PREFIX=] specify an install directory when building the
  \verb+install+ target (default \verb+~/usr+).
\end{description}

\psection{Using graphcode in parallel}

To use graphcode in parallel, you need to install
Classdesc,\index{Classdesc} ParMETIS\index{Metis}\index{ParMETIS} and
MPI.\index{MPI} Define the preprocessor symbol
\verb+MPI_SUPPORT+\index{MPI\_SUPPORT} to enable the parallel
processing code. An example Makefile for the \verb+poisson_demo+
example illustrate how this is done.\index{poisson\_demo}

You will need to arrange the class definitions for your objects, as
well as the graphcode.h file to be processed by 
classdesc. One way of doing this is to \verb+#include+ .cd files into one of
the C++ source files, and have a \verb+.h.cd+ rule in your Makefile, as
suggested in the classdesc documentation.


\psection{Examples}

Graphcode comes with an example of solving the Poisson
equation. Graphcode is also deployed for implementing the spatial
Ecolab model, and the Palauan jellyfish model within the \EcoLab{}
package. The latter model illustrates a dynamic load balancing.

\section{tcl++}\label{tcl++}\index{tcl++}

This section describes a light weight class library for creating
Tcl/Tk applications. \cite{Ousterhout94} It consists of a header file
\verb|tcl++.h|, and a main program \verb|tclmain.cc|. All you need to
do to create a Tcl/Tk application is write any application specific
Tcl commands using the {\tt NEWCMD}\index{NEWCMD}\label{NEWCMD} macro, link your code with the
\verb|tclmain| code, and then proceed to write the rest of your
application as a Tcl script. The \verb|tclmain| code uses argv[1] to
get the name of a script to execute, so assuming that the executable
image part of your application is called \verb|appl|, you would
start off your application script with
\begin{verbatim}
#!appl
...Your Tcl/Tk code goes here...
\end{verbatim}
then your script becomes the application, directly launchable from the
shell.

EcoLab processes any pending TCL events prior to executing the C++
implementation method. This keeps the GUI lively. However, it may
cause problems if TCL scripts depend on C++ state that may be changed
by event processing. There is a boolean global variable {\tt
  processEvents}\index{processEvents}, that disables the background
even processing if set to false. TCL commands {\tt
  enableEventProcessing}\index{enableEventProcessing} and {\tt
  disableEventProcessing}\index{disableEventProcessing} can be used to
set/clear this flag from the TCL interface.

\subsection{Global Variables}

The following global symbols are defined by \verb|tcl++|:
\begin{description}
\item[interp]\index{interp} The default interpreter used by Tcl/Tk
\item[mainWin]\index{mainWin} The window used by Tk, of type
  \verb|Tk_Window|, useful for obtaining X information such as
  colourmaps
\item[error]\index{error} To flag an error, throw a variable of type
  \verb+error+. Use \verb|error(char *format,...)| to construct such
  an object.
The format string is passed to sprintf, along with all remaining
arguments. The resulting string is what is return by the
  \verb+error::what()+ method. Standard execptions are usually caught,
  and the error message return by \verb+what()+ is attached to TCL's
  error log, and \verb+TCL_ERROR+ return back to the interpreter.
\end{description}

\subsection{Creating a Tcl command}
To register a procedure, use the \verb|NEWCMD| macro. It takes two
arguments, the first is the name of the Tcl command to be created, the
second is the number of arguments that the Tcl command takes.

Here is an example.
\begin{verbatim}
#include "tcl++.h"

NEWCMD(silly_cmd,1)
{printf("argv[1]=%s\n", argv[1]);}
\end{verbatim}
This creates a Tcl command that prints out its argument.

\subsection{tclvar}\index{tclvar}\label{tclvar}

Tcl variables can be accessed by means of the tclvar class. For
example, if the programmer declares:

\verb|tclvar hello("hello"); float floatvar;|

then the variable hello can be used just like a normal C variable in
expression such as 

\verb|floatvar=hello*3.4;|.

The class allows assignment to and from \verb|double| and \verb|char*|
variables, incrementing and decrementing the variables, compound
assignment and accessing array Tcl variables by means of the C array
syntax. The is also a function that tests for the existence of a Tcl
variable.

The complete class definition is given by:

\begin{verbatim}
class tclvar
{ 
 public:

/* constructors */
  tclvar();
  tclvar(char *nm, char* val=NULL);
  tclvar(tclvar&);
  ~tclvar();

/* These four statements allow tclvars to be freely mixed with arithmetic 
 expressions */
  double operator=(double x);
  char* operator=(char* x);
  operator double ();
  operator char* ();

  double operator++();
  double operator++(int);
  double operator--();
  double operator--(int); 
  double operator+=(double x);
  double operator-=(double x);
  double operator*=(double x);
  double operator/=(double x);

  tclvar operator=(tclvar x); 
/* arrays can be indexed either by integers, or by strings */
  tclvar operator[](int index);
  tclvar operator[](char* index);

  friend int exists(tclvar x);
};
\end{verbatim}

\subsection{tclcmd}\index{tclcmd}

A tclcmd allows Tcl commands to be executed by a simple \verb|x <<|
{\em Tcl command} syntax. The commands can be stacked, or
accumulated. Each time a linefeed is obtained, the command is
evaluated.

\pagebreak[4]
For example:
\begin{verbatim}
tclcmd cmd;

cmd << "set a 1\n puts stdout $a";
cmd << "for {set i 0} {$i < 10} {incr i}"
cmd << "{set a [expr $a+$i]; puts stdout $a}\n";
\end{verbatim}

The behaviour of this command is slightly different from the usual
stream classes (but identical to {\tt
  eco\_strstream}\index{eco\_strstream} from which it is derived, as a space is
automatically inserted between arguments of a \verb|<<|. This means
that arguments can be easily given - eg
\begin{verbatim}
cmd << "plot" << 1.2 << 3.4 << "\n";
\end{verbatim}
will perform the command {\tt plot 1.2 3.4}. In order to build up a
symbol from multiple arguments, use the append operator. So
\begin{verbatim}
(cmd << "plot"|3) << "\n";
\end{verbatim} 
executes the command {\tt plot3}.


\subsection{tclreturn}\index{tclreturn}

The {\tt tclreturn} class is used to supply a return value to a TCL
command. It inherits from {\tt eco\_strstream}\index{eco\_strstream},
so has exactly the same behaviour as that class. When this variable
goes out of scope, the contents of the stream buffer is written as the
TCL return value. It is an error to have more than one {\tt tclreturn}
in a TCL command, the result is undefined in this case.

\subsection{tclindex}\index{tclindex}

This class is used to index through a TCL array. TCL arrays can have
arbitrary strings for indices, and are not necessarily ordered. The
class has three main methods: {\tt start}, which takes a {\tt tclvar}
variable that refers to the array, and returns the first element in
the array; {\tt incr}, which returns the next element of the array;
and {\tt last} which returns 1 if the last element has been read. A
sample usage might be (for computing the product of all elements in
the array {\tt dims}):
\begin{verbatim}
tclvar dims="dims";
tclindex idx;
for (ncells *= (int)idx.start(dims); !idx.last(); ncells *= (int)idx.incr() );
\end{verbatim}

The class definition is given by:
\begin{verbatim}
class tclindex  
{
public:
  tclindex();
  tclindex(tclindex&);
  ~tclindex() {done();}
  tclvar start(tclvar&);
  inline tclvar incr();
  tclvar incr(tclvar& x) {incr();}  
  int last();
};
\end{verbatim}






\section{arrays}

This section documents the array and related data types. Arrays are
dynamically sized (like std::vector), but are numerical, so can be
used in numerical array expressions like \verb"a+b=0;". Array
expressions use the expression template technique\cite{Veldhuizen95}
to maximise the performance of array expressions.

The code has been optimised for use with Intel's vectorising compiler,
and code will be generated using SSE instructions if compiled with icc.

Arrays and associated functions are defined in the
\verb+array_ns+\index{array\_ns} namespace. 


\subsection{array}\label{array}\index{array}\index{iarray}

\begin{verbatim}

//E is an array expression - eg array<double> or array<double>+array<int>
//S is a scalar data type - eg int
//T must be a C plain old data type, usually numerical

template <class T>
class array
{
 public:
  typedef T value_type;

  size_t size();   //return size of array
  T* data();       //return pointer to array's contents

  explicit array(size_t s=0); //construct array of size s
  array(size_t s, T val);     //construct and initialise array of size s
  void resize(size_t s);      //resize array, data undefined

  array& array(const E& x);   //copy constructor
  array& operator=(const E& x); //assignment or broadcast
  T& operator[](size_t i);    //reference an element
  T& operator[](const E& i);    //vector indexing

  operator+() etc.    //see below
};
\end{verbatim}

\subsubsection{Array operators}

Arithmetical operators \verb"+,-,*,/,%" are defined elementwise
between array expressions, as well as broadcastwise if one argument is
a scalar. For example if x and y and z are arrays, and a is a scalar,
\begin{verbatim}
  z=x+a*y    =>  for (i=0; i<idx.size(); i++) z[i]=x[i]+a*y[i];
\end{verbatim}

Indexing operators [] can either take a single integer argument, which
refers to a single array or expression element, or it can take an
integer array expression, which performs vector indexing. So 
\begin{verbatim}
y=x[idx];    =>  for (i=0; i<idx.size(); i++) y[i] = x[idx[i]];
y[idx]=x;    =>  for (i=0; i<idx.size(); i++) y[idx[i] = x[i];
\end{verbatim}

Comparison operators \verb+<+, \verb+>+ etc, and logical operators
\verb+&&+, \verb+||+ are also defined in elementwise and broadcast versions.

\verb+operator<<(expression1,expression2)+ is a concatenation
operator, appending the elements of expression2 to the end of the
elements of expression1. 

Compound assignment variants also exist:
\begin{verbatim}
x+=y;    =>   x=x+y;
x*=y;    =>   x=x*y;
   ...
x<<=y;   =>   x=x<<y;
\end{verbatim}

\subsection{{\tt sparse\_mat}}\label{sparse_mat}\index{sparse\_mat}

The {\tt sparse\_mat} data type is designed to hold the
interaction matrix \bbeta, which is generally a sparse matrix. It is
built up of array<double> and iarray<int> components:

\begin{verbatim}
class sparse_mat
{
 public:
  array diag, val;
  iarray row, col;
  sparse_mat(int s=0, int o=0)
    {diag=array(s); val=array(o); row=iarray(o); col=iarray(o);}
  array operator*(iarray& x);  /* matrix multiplication */
  sparse_mat operator=(sparse_mat x);
  void init_rand(int conn, double sigma);
};
\end{verbatim}

{\tt diag} stores the diagonal components of the array, {\tt val} is
the packed list of offdiagonal values, with {\tt row} and {\tt col}
being the index lists. The only important operator defined for this
class is the matrix operation, which is defined as
\begin{verbatim}
beta*x == beta.diag*x + (beta.val*x[beta.col])[beta.row]
\end{verbatim}
but is implemented separately for efficiency reasons.

{\tt init\_rand}\index{init\_rand} is a utility routine for randomly
initialising the nonzero pattern of the offdiagonal elements. The
average number of nonzeros per row is {\tt conn}, and the standard
deviation of the number of nonzeros is {\tt sigma}.

It is possible to represent the offdiagonal array differently  for
efficiency reasons. For example, if it is desired to represent the
offdiagonal elements as a dense 2D array, one can create an extra
pointer in the underlying implementation of array. Most of the time,
this pointer is NULL, but when the \verb|cs_arrays| routine
\verb|offmul| is called, it will check this pointer for the {\tt val}
array. If it is NULL, it will create the efficient
representation, otherwise it will reuse the existing one. Because the
only way the array's actual value will get out of synch with the
efficient representation is by index assignment, \verb|put_double()|
and \verb|put_double_array| (as well as obviously
\verb|delete_array()|) will need to be modified to deallocate the
efficient representation.

\subsection{Global functions}

\begin{itemize}
\item array {\tt merge}\index{merge}( iarray {\em mask}, array {\em
    a}, array {\em b} )
  
  return an array (or iarray) {\em r} such that where {\em
    mask[i]==1}, {\em r[i]=a[i]}, otherwise {\em r[i]=b[i]}.

\item array {\tt pack}\index{pack}( array {\em x}, iarray {\em mask}, [int {\em
ntrue}])

Construct a new array from elements of  {\em x} that correspond to
where the mask is true. {\em ntrue} is an optional parameter equal to
the number of true elements of {\em mask}. If not given, then {\tt
pack} will count the true elements of {\em mask}. Give this parameter
if you are using the same {\em mask} in multiple {\tt pack} statements.

\item iarray {\tt enumerate}\index{enumerate}(iarray {\em mask})

Return the running sum of mask.

\item iarray {\tt pcoord}\index{pcoord}( int {\em size})

return [0..size-1]

\item iarray {\tt gen\_index}\index{gen\_index}(iarray {\em x}) 

generate a list of index values, with each number appearing in the
list according to the value passed in its position. For example, if
{\em x}=\{0,0,1,2,0,1\} then the output from this function will be
\{3,4,4,6\}

\item void {\tt lgspread}\index{lgspread}( array\& {\em a}, array {\em s} )

Perform lognormal variation of the components of {\em a}, with
standard error given by {\em s}, ie
\begin{displaymath}
a_i=\mathrm{sgn}(a_i) \exp(\log|a_i| + s_i\xi_i)
\end{displaymath}
where $\xi_i$ is normal random variate.

\item void {\tt gspread}\index{lgspread}( array\& {\em a}, array {\em s} )

Perform normal variation of the components of {\em a}, with
standard error given by {\em s}, ie
\begin{displaymath}
a_i = a_i + s_i\xi_i
\end{displaymath}
where $\xi_i$ is normal random variate.



\end{itemize}

\subsubsection{Reduction functions}

\begin{itemize}
\item double {\tt max}\index{max}( array {\em x}) 
\item double {\tt min}\index{min}( array {\em x}) 
\item double {\tt sum}\index{sum}( array {\em x}) 
\item bool {\tt all}\index{all}( array {\em x}) 
  return true is all of x[i] are nonzero
\item bool {\tt any}\index{any}( array {\em x})
  return true if any x[i] are nonzero
\end{itemize}

\subsubsection{Array functions}
\begin{itemize}
\item array {\tt abs}\index{abs}(array {\em x})
\item iarray {\tt sign}\index{sign}(array {\em x})
\item array {\tt exp}\index{exp}(array {\em x})
\item array {\tt log}\index{log}(array {\em x})
\item array{\tt <int>} ProbRound\index{ProbRound}(array{\tt <double>} {\em x})
    Round up or down with probability given by the difference between
    real value and the integers either side. For instance, if the
    argument's value is 0.2, then it has an 80\% chance of being
    rounded down to 0, and a 20\% chance of being rounded up to 1.
    Uses \hyperref{{\tt array\_urand}}{{\tt
    array\_urand}\S}{)}{random fun}.
\end{itemize}

\subsubsection{Random number functions}\label{random fun}

array's random number functions allow arrays to be filled with random
numbers efficiently in a single call. Most of these functions use the
\verb+array_urand+\index{array\_urand} uniform random object (of type \hyperref{urand}{urand\S}{)}{random}\index{urand}), which is
accessible from TCL. The \verb+fillgrand()+ function makes use of the
TCL accessible \verb+array_grand+\index{array\_grand} object, which is
of type \hyperref{gaussrand}{gaussrand\S}{)}{random}\index{gaussrand}.


\begin{itemize}
\item void {\tt fillrand}\index{fillrand}(array {\em x}) 

Fill {\em x} with random numbers from the uniform distribution over $[0,1]$

\item void {\tt fillprand}\index{fillprand}(array {\em x})

Fill {\em x} with random numbers from the Poisson distribution $e^{-x}$

\item void {\tt fillgrand}\index{fillgrand}(array {\em x})

Fill {\em x} with random numbers from the Gaussian distribution $e^{-x^2/4}$

\item void {\tt fill\_unique\_rand}\index{fill\_unique\_rand}(array {\em
    x}, int {\em max})

Fill {\em x} with random integers from the range [0..{\em max}] such
that no two integers are the same.

\end{itemize}

\subsubsection{Stream Functions}
\begin{itemize}
\item \verb|ostream& operator<<(ostream& s, expr x)|
\item \verb|istream& operator>>(istream& s, array& x)|
\end{itemize}

\subsection{Old arrays}

The original array library had a somewhat different API. The new array
library can be accessed through the old interface by including
\verb+#include "oldarrays.h"+ The only slight difference that may be
noticed is that oldarray::size is not an integer, but an object that
converts to an integer. The problem is that overloaded functions may
ge confused.

\section{tcl\_arrays.cc}

This module defines number of utility TCL commands for initialising
array variables from TCL. Aside from {\tt srand}\index{srand}, they
all return a list of values, suitable for initialising an array:

\subsection{list generation}

\begin{description}
\item[srand {\em seed}] Initialises the random seed
\item[unirand {\em size} [{\em max} [{\em min}]]] \index{unirand} Returns
  a list (of length {\em size}) of uniform random numbers, in the range {\em
  min}--{\em max}. {\em max} and {\em min} default to 1 and 0
  respectively.
\item[grand {\em n} [{\em std} [{\em mean}]]] \index{grand} Returns
  a list (of length {\em size}) of normally distributed random
  numbers, with mean {\em mean} and standard deviation {\em std}. {\em
  std} and {\em mean} default to 1 and 0 respectively.
\item[prand {\em n} [{\em std} [{\em mean}]]] \index{prand} Returns
  a list (of length {\em size}) of Poisson distributed random
  numbers, with mean {\em mean} and standard deviation {\em std}. {\em
  std} and {\em mean} default to 1 and 0 respectively.
\item[unique\_rand {\em size} [{\em max} [{\em min}]]]
  \index{unique\_rand} Returns a list of unique random integers of
  length {\em size} between {\em min} and {\em max}, which default to
  0 and {\em size} respectively.
\item[constant {\em size} {\em value}] \index{constant} returns a list
  of {\em size} elements, each of value {\em value}.
\item[pcoord {\em size}] \index{pcoord} returns a list of {\em size}
  elements, whose values are the numbers 0\ldots{\em size}.
\end{description}

\subsection{Reduction Functions}

Syntax:\index{max}\index{min}\index{av}
\begin{quote}
{\tt max} {\em tcllist}\\
{\tt min} {\em tcllist}\\
{\tt av} {\em tcllist}\\
\end{quote}

The maximum, minimum and average of a TCL list.


\section{Statistical Analysis}

The Analysis module contains some simple statistics support in the
form of a \hyperref{TCL exported C++ data type}{TCL exported C++ data
  type (\S}{)}{TCLTYPE}. The class definition is:

\begin{verbatim}

struct Stats: public array_ns::array<float>
{
  double sum, sumsq;
  float max, min;
  Stats(): sum(0), sumsq(0), max(-std::numeric_limits<float>::max()), 
           min(std::numeric_limits<float>::max()) {}

  void clear();
  double av();
  double median();
  double stddev();

  Stats& operator<<=(float x);
  Stats& operator<<=(const array_ns::array<float>& x);
  Stats& operator<<=(const array_ns::array<double>& x);

  void add_data(TCL_args args); 
};

struct HistoStats: public Stats
{
  unsigned nbins;
  bool logbins;
  HistoStats(): nbins(100), logbins(false) {}
  array_ns::array<double> histogram();
  array_ns::array<double> bins();

  double loglikelihood(TCL_args args);

  array_ns::array<double> fitPowerLaw(); //< fit x^{-a} - return a and x_min
  double fitExponential()                //< fit exp(-x/a) - return a 
  array_ns::array<double> fitNormal();   //< fit exp(-(x-m)^2/2s, return m, s
    
  array_ns::array<double> fitLogNormal(); //< fit exp(-(log(x)-log m)^2/2s, return m, s
};
\end{verbatim}

These classes can be used from the TCL programming environment like
this example:

\begin{verbatim}

HistoStats h


unuran rand
rand.set_gen {distr=pareto(.5,1.7);}

for {set i 0} {$i<10000} {incr i} {
    h.add_data [rand.rand]
}

# obtain average, median, min, max, standard deviation and no. samples
puts stdout "[h.av] [h.median] [h.min] [h.max] [h.stddev] [h.size]"

# fit power law distribution, returning slope and xmin
puts stdout "[h.fitPowerLaw]"

# return log likelihood ratio for power law versus lognormal
array set pl [h.fitPowerLaw]
array set ln [h.fitLogNormal]
set R [h.loglikelihood powerlaw($pl(0),$pl(1)) 
                       lognormal($ln(0),$ln(1) $pl(1)]
puts stdout "R=$R p=[expr fabs([erfc $R])]"
\end{verbatim}

Fitting parameters is achieved using the likelihood method as
described in \cite{Clauset-etal07}.

The log likelihood ratio function returns ${\cal R}/\sqrt{2n}\sigma$,
where ${\cal R}=\ln \prod_i p_1(x_i)/p_2(x_i)$ is the logarithm of the
ratio of likelihoods for the two distributions $p_1$ and $p_2$.

If the log likelihood ratio is positive, it means the $p_1$ is more
likely to fit the data than $p_2$, and negative is the other way
around. $p=|\mathrm{erfc}({\cal R})|$ is the probability that this conclusion
is wrong.\cite{Clauset-etal07}

The histogram function allows one to do histograms without using the
GUI widget, which is useful for larger collections of data. The
parameters to the histogram method are the number of bins (default
100) and whether linear or logarithmic binning is used.

\section{random.cc}\label{random}

This module abstracts the concept of a random number generator. The
base types are:
\index{random\_gen}\index{urand}\index{rangerand}\index{gaussrand}
\index{distrand}

\begin{verbatim}
class random_gen
{
public:
  virtual double rand()=0;
};

class affinerand: public random_gen
{
public:
  double scale, offset;
  affinerand(double s=1, double o=0, random_gen *g=NULL);
  void Set_gen(random_gen*);
  void set_gen(TCL_args args) {Set_gen(args);}
  template <class T> void new_gen(const T&);
  double rand() {return scale*gen->rand()+offset;}
};
\end{verbatim}

\verb+random_gen+ is an abstract base class representing a random
number generator. \verb+affinerand+ performs a simple affine
transformation on the contained random generator. The
\verb+random_gen*+ argument to \verb+affine_rand()+ may be used to set
the random generator employed. If \verb+NULL+ (default) is passed for
this parameter, a uniform random generator of type urand is created
and used. After creation, you can either set the random generator to
be something else using \verb+Set_gen+, or create a new random
generator of the same type as the passed argument using
\verb+new_gen+. The \verb+set_gen+ method is callable from TCL, and
takes a named \verb+random_gen+ object as an argument. The class's
destructor will delete a generator created with newgen, but not delete
an object passed by setgen. For example, a normal distribution with
mean m and standard deviation s can be obtained with the declaration:
\begin{verbatim}
affinerand(s,m).newgen(gaussrand());
\end{verbatim}

The other classes are:

\begin{verbatim}
class urand: public random_gen
{
public:
  double rand();
  void Seed(int s);
  void seed(TCL_args args) {Seed(args);}
#if defined(UNURAN) || defined(GNUSL)
  void Set_gen(char *);
  void set_gen(TCL_args args) {Set_gen(args);}
  urand(const char* descr) {Set_gen(descr);}
#endif
};

class gaussrand: public random_gen
{
public:
  urand uni;
  double rand();
};
\end{verbatim}

\verb+urand+ simply returns a uniform random variate in [0,1]. The
basic \EcoLab{} code contains a uniform generator which is simply an
interface to the standard library \verb+rand()+ call, and a Gaussian
random generator, which is based on the algorithm described in
Abramowitz and Stegun (1964) sec. 26.8.6.a(2).

\verb+gaussrand+ returns a normal variate with mean 0 and standard
deviation 1. Use \verb+gaussrand+ coupled with \verb+affinerand+ to
change the mean and standard deviation:
\begin{verbatim}
affinerand gen(2,.5,new gaussrand);
\end{verbatim}
defines a gaussian generator variable {\tt gen} with standard
deviation 2 and mean .5.

As of \EcoLab.4.D7, replacements for these routines using freely
available libraries, unuran and GNUSL are available. The Makefile will
select the \htmladdnormallinkfoot{UNURAN
  library}{http://statistik.wu-wien.ac.at/unuran/} if available,
otherwise the \htmladdnormallinkfoot{GNU Scientific
  Library}{http://www.gnu.org/software/gsl/} will be selected. If
neither of these packages are available, the original basic behaviour
is selected. Please read the section on \hyperref{problems with the
  basic random number library}{(\S}{)}{basic RNG problems}.

Both UNURAN and GNUSL provide a text interface to selecting and
configuring the uniform random generator. The method \verb+Set_gen+
provides a way passing \htmladdnormallinkfoot{PRNG
  parameters}{http://statistik.wu-wien.ac.at/prng/prng/doc/prng.html}
to the underlying PRNG generator. By default the Mersenne Twister
algorithm is used, which is acknowledged as being one of the most
efficient random generator available. The algorithms available
through PRNG are:

\begin{description}
\item[EICG] explicit inversive congruential generator
\item[ICG] inversive congruential generator
\item[LCG] linear congruential generator
\item[QCG] quadratic congruential generator
\item[MT19937] Mersenne Twister
\item[MEICG] modified explicit inversive congruential generator
\item[DICG] digital inversive congruential generator
\end{description}

The set available through GNUSL are:
\begin{description}
\item[mt19937] Mersenne Twister
\item[ranlxs0,ranlxs1,ranlxs2,ranlxd1,ranlxd2] L\"uscher's RANLUX at
  different levels of precision and luxury.
\item[cmrg] combined multiple recursive generator by L'Ecuyer
\item[mrg] fifth-order multiple recursive generator by L'Ecuyer,
  Blouin and Coutre
\item[taus,taus2] maximally equidistributed combined Tausworthe
  generator by L'Ecuyer
\item[gfsr4] Four-tap shift-register-sequence random-number generator
  by Ziff.
\end{description}
Note that only the generator can be selected through the GNUSL string
interface --- parameters cannot be set. The GNUSL documentation
recommends mt19937, taus or gfsr4 for general purpose
simulation. mt19937 is the default.

\begin{verbatim}
class distrand: public random_gen
{
public:
  int nsamp;  /* no. of sample points in distribution */
  int width;  /* digits of precision (base 16) used from prob. distribution */
  double min, max;  /* distribution endpoints */
  distrand(): nsamp(10), width(3), min(0), max(1);
  void Init(int argc, char *argv[]);
  void init(double (*f)(double));
  double rand();
};

#ifdef UNURAN
class unuran: public random_gen
{
public:
  urand uni;
  UNUR_RAN *get_gen(); //get pointer to UNUR_RAN object for UNURAN API use
  /* specify a random generator according to unuran's string interface */
  void Set_gen(const char *descr);
  void set_gen(TCL_args args) {Set_gen(args);}
  unuran();
  unuran(const char* descr) {Set_gen(descr);}
  double rand();
};
#endif
\end{verbatim}

\verb+distrand+ returns a deviate from an arbitrary
distribution function (which needn't be normalised) supplied to
\verb+Init+ (or \verb+init+). The instance variables \verb+nsamp=10+,
\verb+width=3+, \verb+min=0+ and \verb+max=1+ should be modified
before calling \verb+init+. \verb+Init+ provides a TCL interface ---
it takes one argument, the name of a TCL procedure implementing the
distribution. This class implements the method due to
Marsaglia\cite{Marsaglia63}. The UNURAN and GNUSL libraries provide
other, perhaps better routines for doing the same things.

Finally, \verb+unuran+ allows an arbitrary UNURAN generator to be
specified using \htmladdnormallinkfoot{UNURAN's string
  interface}{http://statistik.wu-wien.ac.at/unuran/unuran/doc/unuran.html}.
For example, \verb+gaussrand+ is equivalent to
\verb+unuran("normal()")+. The string interface is powerful and
comprehensive, with a large number of predefined distributions and
methods provided, and the ability to specify arbitrary
distributions. It obviates the need to use \verb+distrand+, which is a
rather obsolete algorithm.

GNUSL does not provide a string interace to its general nonuniform
distributions. For the moment, you will need to implement your own
object interfaces to which ever routines you want to use. You may use
the gaussrand example as a template.

\subsection{TCL interface}

Each of the concrete random number types have been declared with
\hyperref{TCLTYPE}{TCLTYPE (See \S}{)}{TCLTYPE}, so
random generator objects can be created at the TCL level.\index{TCLTYPE} See
newman.tcl, or test-dist.tcl for an example. 

Using the unuran type, one can easily set up arbitrary distribution
functions at runtime.
\begin{verbatim}
unuran rand
rand.set_gen {distr=cont; pdf="x^2"; domain=(0,10)}
\end{verbatim}

\subsection{Problems with the basic random number library}\label{basic RNG problems}

\subsubsection{Basic urand objects are not independent}

Because the basic urand class uses the underlying Unix \verb+rand()+
call, objects of type urand are not independent. Rather, they
effectively refer to the same object. Setting the seed on one object
of this class sets the seed for all objects of that class.

If for example, one wants to perform simulations in parallel, and
compare results with different numbers of threads, then the simplest
way of arranging this is to ensure that each object in the simulation
requiring random numbers gets its own unique stream of random
numbers. This can be done by assigning a different urand object to
each simulation object, and giving them a distinct seed. Then no
matter what the distribution of objects over threads are, the sequence
random numbers will be the same from simulation to simulation,
allowing meaningful comparisons. The UNURAN and GNUSL libraries both
provide independently streamed random numbers.

\subsubsection{Basic urand algorithm not efficient}

Since the basic urand object depends on Unix \verb+rand()+, it is
neither the most efficient algorithm available (the Mersenne Twister
is generally better), nor is it necessarily acceptible in all cases.
All pseudo-random generators have some kind of structure to their
outputs, which may or may not be significant in the simulation. With
the UNURAN library or with GNUSL it is possible to select from a range
of different algorithms at runtime, making it easy to check that the
simulation results do not depend on the type of generator used.

\subsubsection{Basic Gaussrand is correlated}

The graph of $x_t$ versus $x_{t-1}$ shown below says it all. The
correlations of the basic Gaussrand algorithm can well be
unacceptable:

\noindent
\begin{tabular}{cc}
% GNUPLOT: LaTeX picture using PSTRICKS macros
% Define new PST objects, if not already defined
\small
\ifx\PSTloaded\undefined
\def\PSTloaded{t}
\psset{arrowsize=.01 3.2 1.4 .3}
\psset{dotsize=.05}
\catcode`@=11

\newpsobject{PST@Border}{psline}{linewidth=.0015,linestyle=solid}
\newpsobject{PST@Axes}{psline}{linewidth=.0015,linestyle=dotted,dotsep=.004}
\newpsobject{PST@Solid}{psline}{linewidth=.0015,linestyle=solid}
\newpsobject{PST@Dashed}{psline}{linewidth=.0015,linestyle=dashed,dash=.01 .01}
\newpsobject{PST@Dotted}{psline}{linewidth=.0025,linestyle=dotted,dotsep=.008}
\newpsobject{PST@LongDash}{psline}{linewidth=.0015,linestyle=dashed,dash=.02 .01}
\newpsobject{PST@Diamond}{psdots}{linewidth=.01,linestyle=solid,dotstyle=square,dotangle=45}
\newpsobject{PST@Filldiamond}{psdots}{linewidth=.001,linestyle=solid,dotstyle=square*,dotangle=45}
\newpsobject{PST@Cross}{psdots}{linewidth=.001,linestyle=solid,dotstyle=+,dotangle=45}
\newpsobject{PST@Plus}{psdots}{linewidth=.001,linestyle=solid,dotstyle=+}
\newpsobject{PST@Square}{psdots}{linewidth=.001,linestyle=solid,dotstyle=square}
\newpsobject{PST@Circle}{psdots}{linewidth=.001,linestyle=solid,dotstyle=o}
\newpsobject{PST@Triangle}{psdots}{linewidth=.001,linestyle=solid,dotstyle=triangle}
\newpsobject{PST@Pentagon}{psdots}{linewidth=.001,linestyle=solid,dotstyle=pentagon}
\newpsobject{PST@Fillsquare}{psdots}{linewidth=.001,linestyle=solid,dotstyle=square*}
\newpsobject{PST@Fillcircle}{psdots}{linewidth=.001,linestyle=solid,dotstyle=*}
\newpsobject{PST@Filltriangle}{psdots}{linewidth=.001,linestyle=solid,dotstyle=triangle*}
\newpsobject{PST@Fillpentagon}{psdots}{linewidth=.001,linestyle=solid,dotstyle=pentagon*}
\newpsobject{PST@Arrow}{psline}{linewidth=.001,linestyle=solid}
\catcode`@=12

\fi
\psset{unit=5.000000in,xunit=2.4in,yunit=1.5in}
\pspicture(0.05,0)(.9,1)
\ifx\nofigs\undefined
\catcode`@=11

\PST@Border(0.1270,0.1260)
(0.1420,0.1260)

\PST@Border(0.9470,0.1260)
(0.9320,0.1260)

\rput[r](0.1110,0.1260){-4}
\PST@Border(0.1270,0.2313)
(0.1420,0.2313)

\PST@Border(0.9470,0.2313)
(0.9320,0.2313)

\rput[r](0.1110,0.2313){-3}
\PST@Border(0.1270,0.3365)
(0.1420,0.3365)

\PST@Border(0.9470,0.3365)
(0.9320,0.3365)

\rput[r](0.1110,0.3365){-2}
\PST@Border(0.1270,0.4418)
(0.1420,0.4418)

\PST@Border(0.9470,0.4418)
(0.9320,0.4418)

\rput[r](0.1110,0.4418){-1}
\PST@Border(0.1270,0.5470)
(0.1420,0.5470)

\PST@Border(0.9470,0.5470)
(0.9320,0.5470)

\rput[r](0.1110,0.5470){0}
\PST@Border(0.1270,0.6523)
(0.1420,0.6523)

\PST@Border(0.9470,0.6523)
(0.9320,0.6523)

\rput[r](0.1110,0.6523){1}
\PST@Border(0.1270,0.7575)
(0.1420,0.7575)

\PST@Border(0.9470,0.7575)
(0.9320,0.7575)

\rput[r](0.1110,0.7575){2}
\PST@Border(0.1270,0.8628)
(0.1420,0.8628)

\PST@Border(0.9470,0.8628)
(0.9320,0.8628)

\rput[r](0.1110,0.8628){3}
\PST@Border(0.1270,0.9680)
(0.1420,0.9680)

\PST@Border(0.9470,0.9680)
(0.9320,0.9680)

\rput[r](0.1110,0.9680){4}
\PST@Border(0.1270,0.1260)
(0.1270,0.1460)

\PST@Border(0.1270,0.9680)
(0.1270,0.9480)

\rput(0.1270,0.0840){-4}
\PST@Border(0.2295,0.1260)
(0.2295,0.1460)

\PST@Border(0.2295,0.9680)
(0.2295,0.9480)

\rput(0.2295,0.0840){-3}
\PST@Border(0.3320,0.1260)
(0.3320,0.1460)

\PST@Border(0.3320,0.9680)
(0.3320,0.9480)

\rput(0.3320,0.0840){-2}
\PST@Border(0.4345,0.1260)
(0.4345,0.1460)

\PST@Border(0.4345,0.9680)
(0.4345,0.9480)

\rput(0.4345,0.0840){-1}
\PST@Border(0.5370,0.1260)
(0.5370,0.1460)

\PST@Border(0.5370,0.9680)
(0.5370,0.9480)

\rput(0.5370,0.0840){0}
\PST@Border(0.6395,0.1260)
(0.6395,0.1460)

\PST@Border(0.6395,0.9680)
(0.6395,0.9480)

\rput(0.6395,0.0840){1}
\PST@Border(0.7420,0.1260)
(0.7420,0.1460)

\PST@Border(0.7420,0.9680)
(0.7420,0.9480)

\rput(0.7420,0.0840){2}
\PST@Border(0.8445,0.1260)
(0.8445,0.1460)

\PST@Border(0.8445,0.9680)
(0.8445,0.9480)

\rput(0.8445,0.0840){3}
\PST@Border(0.9470,0.1260)
(0.9470,0.1460)

\PST@Border(0.9470,0.9680)
(0.9470,0.9480)

\rput(0.9470,0.0840){4}
\PST@Border(0.1270,0.1260)
(0.9470,0.1260)
(0.9470,0.9680)
(0.1270,0.9680)
(0.1270,0.1260)

\rput{L}(0.0420,0.5470){$x_t$}
\rput(0.5370,0.0210){$x_{t-1}$}
\PST@Diamond(0.5745,0.4118)
\PST@Diamond(0.4053,0.6615)
\PST@Diamond(0.6485,0.5592)
\PST@Diamond(0.5489,0.4815)
\PST@Diamond(0.4732,0.6127)
\PST@Diamond(0.6010,0.7517)
\PST@Diamond(0.7363,0.7254)
\PST@Diamond(0.7107,0.6415)
\PST@Diamond(0.6290,0.7058)
\PST@Diamond(0.6917,0.5463)
\PST@Diamond(0.5363,0.3881)
\PST@Diamond(0.3822,0.5399)
\PST@Diamond(0.5300,0.6057)
\PST@Diamond(0.5941,0.5214)
\PST@Diamond(0.5121,0.5272)
\PST@Diamond(0.5178,0.6182)
\PST@Diamond(0.6063,0.6851)
\PST@Diamond(0.6715,0.6013)
\PST@Diamond(0.5899,0.7161)
\PST@Diamond(0.7017,0.6523)
\PST@Diamond(0.6396,0.5247)
\PST@Diamond(0.5153,0.4638)
\PST@Diamond(0.4559,0.5289)
\PST@Diamond(0.5194,0.5068)
\PST@Diamond(0.4979,0.4022)
\PST@Diamond(0.3960,0.4649)
\PST@Diamond(0.4571,0.5982)
\PST@Diamond(0.5868,0.7201)
\PST@Diamond(0.7056,0.6204)
\PST@Diamond(0.6085,0.4863)
\PST@Diamond(0.4779,0.5213)
\PST@Diamond(0.5120,0.6076)
\PST@Diamond(0.5961,0.5605)
\PST@Diamond(0.5501,0.5723)
\PST@Diamond(0.5616,0.4669)
\PST@Diamond(0.4590,0.4997)
\PST@Diamond(0.4910,0.4877)
\PST@Diamond(0.4792,0.8255)
\PST@Diamond(0.8082,0.5248)
\PST@Diamond(0.5154,0.4807)
\PST@Diamond(0.4724,0.5204)
\PST@Diamond(0.5111,0.6450)
\PST@Diamond(0.6324,0.5314)
\PST@Diamond(0.5218,0.5500)
\PST@Diamond(0.5400,0.5185)
\PST@Diamond(0.5092,0.7074)
\PST@Diamond(0.6932,0.5280)
\PST@Diamond(0.5185,0.6003)
\PST@Diamond(0.5889,0.4850)
\PST@Diamond(0.4767,0.4431)
\PST@Diamond(0.4358,0.6781)
\PST@Diamond(0.6647,0.4989)
\PST@Diamond(0.4902,0.6473)
\PST@Diamond(0.6347,0.3201)
\PST@Diamond(0.3160,0.4905)
\PST@Diamond(0.4820,0.5705)
\PST@Diamond(0.5599,0.5270)
\PST@Diamond(0.5176,0.5230)
\PST@Diamond(0.5136,0.5762)
\PST@Diamond(0.5654,0.5166)
\PST@Diamond(0.5074,0.5581)
\PST@Diamond(0.5479,0.5431)
\PST@Diamond(0.5332,0.5380)
\PST@Diamond(0.5283,0.6593)
\PST@Diamond(0.6463,0.4693)
\PST@Diamond(0.4614,0.3872)
\PST@Diamond(0.3814,0.6129)
\PST@Diamond(0.6012,0.5892)
\PST@Diamond(0.5781,0.5633)
\PST@Diamond(0.5529,0.3669)
\PST@Diamond(0.3616,0.5448)
\PST@Diamond(0.5349,0.7296)
\PST@Diamond(0.7148,0.5591)
\PST@Diamond(0.5488,0.3931)
\PST@Diamond(0.3872,0.6259)
\PST@Diamond(0.6139,0.6825)
\PST@Diamond(0.6690,0.4934)
\PST@Diamond(0.4848,0.5115)
\PST@Diamond(0.5024,0.4323)
\PST@Diamond(0.4253,0.4710)
\PST@Diamond(0.4630,0.6948)
\PST@Diamond(0.6810,0.5335)
\PST@Diamond(0.5238,0.5493)
\PST@Diamond(0.5392,0.4504)
\PST@Diamond(0.4429,0.6799)
\PST@Diamond(0.6664,0.5599)
\PST@Diamond(0.5496,0.5629)
\PST@Diamond(0.5525,0.6160)
\PST@Diamond(0.6042,0.4822)
\PST@Diamond(0.4739,0.3989)
\PST@Diamond(0.3928,0.7364)
\PST@Diamond(0.7215,0.7325)
\PST@Diamond(0.7176,0.3480)
\PST@Diamond(0.3432,0.6214)
\PST@Diamond(0.6095,0.6215)
\PST@Diamond(0.6096,0.5568)
\PST@Diamond(0.5465,0.5119)
\PST@Diamond(0.5028,0.4392)
\PST@Diamond(0.4321,0.5819)
\PST@Diamond(0.5710,0.5766)
\PST@Diamond(0.5658,0.5540)
\PST@Diamond(0.5439,0.6202)
\PST@Diamond(0.6083,0.6611)
\PST@Diamond(0.6481,0.3726)
\PST@Diamond(0.3672,0.6038)
\PST@Diamond(0.5923,0.5199)
\PST@Diamond(0.5106,0.5712)
\PST@Diamond(0.5606,0.5705)
\PST@Diamond(0.5599,0.5180)
\PST@Diamond(0.5087,0.4786)
\PST@Diamond(0.4704,0.5102)
\PST@Diamond(0.5012,0.8293)
\PST@Diamond(0.8120,0.5339)
\PST@Diamond(0.5242,0.5388)
\PST@Diamond(0.5290,0.6217)
\PST@Diamond(0.6097,0.5445)
\PST@Diamond(0.5346,0.5458)
\PST@Diamond(0.5358,0.6775)
\PST@Diamond(0.6641,0.6419)
\PST@Diamond(0.6295,0.4776)
\PST@Diamond(0.4694,0.4853)
\PST@Diamond(0.4769,0.4601)
\PST@Diamond(0.4524,0.5056)
\PST@Diamond(0.4967,0.6560)
\PST@Diamond(0.6432,0.5835)
\PST@Diamond(0.5725,0.5674)
\PST@Diamond(0.5569,0.4174)
\PST@Diamond(0.4108,0.4144)
\PST@Diamond(0.4078,0.4913)
\PST@Diamond(0.4828,0.7161)
\PST@Diamond(0.7016,0.5749)
\PST@Diamond(0.5642,0.4819)
\PST@Diamond(0.4736,0.4397)
\PST@Diamond(0.4325,0.5046)
\PST@Diamond(0.4957,0.4521)
\PST@Diamond(0.4446,0.4879)
\PST@Diamond(0.4794,0.6201)
\PST@Diamond(0.6082,0.6347)
\PST@Diamond(0.6224,0.3946)
\PST@Diamond(0.3886,0.5659)
\PST@Diamond(0.5554,0.4608)
\PST@Diamond(0.4531,0.6771)
\PST@Diamond(0.6637,0.3716)
\PST@Diamond(0.3662,0.5923)
\PST@Diamond(0.5811,0.4547)
\PST@Diamond(0.4471,0.4274)
\PST@Diamond(0.4205,0.5263)
\PST@Diamond(0.5169,0.8054)
\PST@Diamond(0.7886,0.7323)
\PST@Diamond(0.7175,0.4574)
\PST@Diamond(0.4498,0.6089)
\PST@Diamond(0.5972,0.4945)
\PST@Diamond(0.4859,0.5879)
\PST@Diamond(0.5768,0.4460)
\PST@Diamond(0.4387,0.6204)
\PST@Diamond(0.6085,0.5359)
\PST@Diamond(0.5262,0.5718)
\PST@Diamond(0.5611,0.7733)
\PST@Diamond(0.7574,0.4627)
\PST@Diamond(0.4549,0.5869)
\PST@Diamond(0.5759,0.6371)
\PST@Diamond(0.6248,0.4175)
\PST@Diamond(0.4109,0.5817)
\PST@Diamond(0.5708,0.6714)
\PST@Diamond(0.6582,0.5994)
\PST@Diamond(0.5881,0.5108)
\PST@Diamond(0.5017,0.4962)
\PST@Diamond(0.4875,0.5838)
\PST@Diamond(0.5728,0.5038)
\PST@Diamond(0.4949,0.5408)
\PST@Diamond(0.5309,0.4868)
\PST@Diamond(0.4784,0.4554)
\PST@Diamond(0.4478,0.5013)
\PST@Diamond(0.4925,0.6532)
\PST@Diamond(0.6404,0.5570)
\PST@Diamond(0.5467,0.3295)
\PST@Diamond(0.3252,0.6650)
\PST@Diamond(0.6519,0.7255)
\PST@Diamond(0.7108,0.6107)
\PST@Diamond(0.5991,0.4382)
\PST@Diamond(0.4311,0.4538)
\PST@Diamond(0.4463,0.5805)
\PST@Diamond(0.5696,0.5553)
\PST@Diamond(0.5451,0.5776)
\PST@Diamond(0.5668,0.6081)
\PST@Diamond(0.5965,0.5658)
\PST@Diamond(0.5553,0.5173)
\PST@Diamond(0.5081,0.6137)
\PST@Diamond(0.6020,0.5239)
\PST@Diamond(0.5145,0.4684)
\PST@Diamond(0.4604,0.7043)
\PST@Diamond(0.6902,0.6897)
\PST@Diamond(0.6759,0.4976)
\PST@Diamond(0.4889,0.5916)
\PST@Diamond(0.5804,0.4954)
\PST@Diamond(0.4868,0.4256)
\PST@Diamond(0.4188,0.7304)
\PST@Diamond(0.7156,0.5185)
\PST@Diamond(0.5092,0.6377)
\PST@Diamond(0.6254,0.5764)
\PST@Diamond(0.5656,0.6160)
\PST@Diamond(0.6042,0.5310)
\PST@Diamond(0.5214,0.3237)
\PST@Diamond(0.3195,0.5675)
\PST@Diamond(0.5570,0.7216)
\PST@Diamond(0.7071,0.5880)
\PST@Diamond(0.5769,0.4829)
\PST@Diamond(0.4745,0.3240)
\PST@Diamond(0.3199,0.6071)
\PST@Diamond(0.5955,0.6817)
\PST@Diamond(0.6682,0.3970)
\PST@Diamond(0.3909,0.5968)
\PST@Diamond(0.5855,0.5666)
\PST@Diamond(0.5561,0.5576)
\PST@Diamond(0.5473,0.4122)
\PST@Diamond(0.4057,0.5426)
\PST@Diamond(0.5327,0.4859)
\PST@Diamond(0.4775,0.4359)
\PST@Diamond(0.4288,0.6259)
\PST@Diamond(0.6139,0.6508)
\PST@Diamond(0.6381,0.6063)
\PST@Diamond(0.5948,0.5278)
\PST@Diamond(0.5183,0.5445)
\PST@Diamond(0.5346,0.8215)
\PST@Diamond(0.8043,0.4768)
\PST@Diamond(0.4686,0.4961)
\PST@Diamond(0.4874,0.5715)
\PST@Diamond(0.5609,0.7018)
\PST@Diamond(0.6877,0.5740)
\PST@Diamond(0.5633,0.5324)
\PST@Diamond(0.5228,0.7344)
\PST@Diamond(0.7195,0.5972)
\PST@Diamond(0.5859,0.7045)
\PST@Diamond(0.6904,0.5333)
\PST@Diamond(0.5237,0.4332)
\PST@Diamond(0.4261,0.5655)
\PST@Diamond(0.5551,0.4881)
\PST@Diamond(0.4796,0.6816)
\PST@Diamond(0.6681,0.6076)
\PST@Diamond(0.5960,0.4011)
\PST@Diamond(0.3949,0.5545)
\PST@Diamond(0.5443,0.5357)
\PST@Diamond(0.5260,0.8014)
\PST@Diamond(0.7847,0.5170)
\PST@Diamond(0.5078,0.4333)
\PST@Diamond(0.4263,0.5339)
\PST@Diamond(0.5242,0.6040)
\PST@Diamond(0.5925,0.5004)
\PST@Diamond(0.4916,0.3796)
\PST@Diamond(0.3740,0.4583)
\PST@Diamond(0.4506,0.5055)
\PST@Diamond(0.4966,0.5062)
\PST@Diamond(0.4973,0.3705)
\PST@Diamond(0.3651,0.4831)
\PST@Diamond(0.4748,0.4941)
\PST@Diamond(0.4854,0.4094)
\PST@Diamond(0.4030,0.6217)
\PST@Diamond(0.6097,0.6806)
\PST@Diamond(0.6671,0.5823)
\PST@Diamond(0.5714,0.3367)
\PST@Diamond(0.3322,0.6791)
\PST@Diamond(0.6656,0.2996)
\PST@Diamond(0.2961,0.4466)
\PST@Diamond(0.4393,0.5563)
\PST@Diamond(0.5461,0.5741)
\PST@Diamond(0.5634,0.5890)
\PST@Diamond(0.5779,0.8155)
\PST@Diamond(0.7985,0.4954)
\PST@Diamond(0.4868,0.6765)
\PST@Diamond(0.6632,0.6438)
\PST@Diamond(0.6313,0.5475)
\PST@Diamond(0.5375,0.7631)
\PST@Diamond(0.7474,0.5649)
\PST@Diamond(0.5544,0.5902)
\PST@Diamond(0.5791,0.6081)
\PST@Diamond(0.5965,0.5565)
\PST@Diamond(0.5462,0.6757)
\PST@Diamond(0.6623,0.3238)
\PST@Diamond(0.3196,0.3892)
\PST@Diamond(0.3833,0.6817)
\PST@Diamond(0.6681,0.5862)
\PST@Diamond(0.5752,0.4499)
\PST@Diamond(0.4424,0.4817)
\PST@Diamond(0.4734,0.4659)
\PST@Diamond(0.4580,0.5189)
\PST@Diamond(0.5097,0.5508)
\PST@Diamond(0.5407,0.2122)
\PST@Diamond(0.2109,0.5562)
\PST@Diamond(0.5459,0.4308)
\PST@Diamond(0.4238,0.4661)
\PST@Diamond(0.4582,0.5182)
\PST@Diamond(0.5089,0.4564)
\PST@Diamond(0.4488,0.6877)
\PST@Diamond(0.6741,0.3449)
\PST@Diamond(0.3402,0.6409)
\PST@Diamond(0.6284,0.5007)
\PST@Diamond(0.4919,0.5433)
\PST@Diamond(0.5334,0.5447)
\PST@Diamond(0.5348,0.4611)
\PST@Diamond(0.4533,0.4101)
\PST@Diamond(0.4037,0.4192)
\PST@Diamond(0.4126,0.4978)
\PST@Diamond(0.4890,0.3189)
\PST@Diamond(0.3148,0.2713)
\PST@Diamond(0.2685,0.3939)
\PST@Diamond(0.3879,0.5906)
\PST@Diamond(0.5794,0.6672)
\PST@Diamond(0.6541,0.4556)
\PST@Diamond(0.4480,0.6513)
\PST@Diamond(0.6386,0.5744)
\PST@Diamond(0.5636,0.4184)
\PST@Diamond(0.4118,0.5452)
\PST@Diamond(0.5353,0.5149)
\PST@Diamond(0.5057,0.3285)
\PST@Diamond(0.3243,0.6018)
\PST@Diamond(0.5904,0.4284)
\PST@Diamond(0.4215,0.6569)
\PST@Diamond(0.6441,0.6141)
\PST@Diamond(0.6023,0.4702)
\PST@Diamond(0.4622,0.5013)
\PST@Diamond(0.4925,0.5821)
\PST@Diamond(0.5712,0.6420)
\PST@Diamond(0.6295,0.6950)
\PST@Diamond(0.6811,0.4943)
\PST@Diamond(0.4856,0.5835)
\PST@Diamond(0.5725,0.5146)
\PST@Diamond(0.5054,0.5616)
\PST@Diamond(0.5512,0.5527)
\PST@Diamond(0.5426,0.7640)
\PST@Diamond(0.7484,0.3553)
\PST@Diamond(0.3503,0.6978)
\PST@Diamond(0.6839,0.5713)
\PST@Diamond(0.5606,0.5536)
\PST@Diamond(0.5435,0.5711)
\PST@Diamond(0.5605,0.5351)
\PST@Diamond(0.5254,0.7124)
\PST@Diamond(0.6981,0.4039)
\PST@Diamond(0.3977,0.5207)
\PST@Diamond(0.5114,0.4865)
\PST@Diamond(0.4781,0.5861)
\PST@Diamond(0.5751,0.5751)
\PST@Diamond(0.5644,0.4268)
\PST@Diamond(0.4199,0.6700)
\PST@Diamond(0.6568,0.4017)
\PST@Diamond(0.3955,0.6696)
\PST@Diamond(0.6564,0.5535)
\PST@Diamond(0.5433,0.6202)
\PST@Diamond(0.6083,0.4964)
\PST@Diamond(0.4877,0.5295)
\PST@Diamond(0.5200,0.4281)
\PST@Diamond(0.4212,0.5519)
\PST@Diamond(0.5418,0.5483)
\PST@Diamond(0.5383,0.4590)
\PST@Diamond(0.4513,0.2809)
\PST@Diamond(0.2779,0.7219)
\PST@Diamond(0.7073,0.5998)
\PST@Diamond(0.5884,0.5422)
\PST@Diamond(0.5323,0.7821)
\PST@Diamond(0.7659,0.5543)
\PST@Diamond(0.5441,0.5925)
\PST@Diamond(0.5813,0.5605)
\PST@Diamond(0.5502,0.5904)
\PST@Diamond(0.5793,0.6455)
\PST@Diamond(0.6329,0.5254)
\PST@Diamond(0.5160,0.6419)
\PST@Diamond(0.6294,0.5230)
\PST@Diamond(0.5136,0.4383)
\PST@Diamond(0.4312,0.6658)
\PST@Diamond(0.6527,0.6309)
\PST@Diamond(0.6187,0.6180)
\PST@Diamond(0.6061,0.6413)
\PST@Diamond(0.6289,0.4014)
\PST@Diamond(0.3952,0.6495)
\PST@Diamond(0.6368,0.7225)
\PST@Diamond(0.7079,0.3558)
\PST@Diamond(0.3508,0.4454)
\PST@Diamond(0.4380,0.7458)
\PST@Diamond(0.7306,0.5568)
\PST@Diamond(0.5465,0.5951)
\PST@Diamond(0.5839,0.5675)
\PST@Diamond(0.5570,0.4111)
\PST@Diamond(0.4047,0.4851)
\PST@Diamond(0.4767,0.6676)
\PST@Diamond(0.6545,0.5976)
\PST@Diamond(0.5863,0.5361)
\PST@Diamond(0.5264,0.6136)
\PST@Diamond(0.6019,0.4605)
\PST@Diamond(0.4528,0.6779)
\PST@Diamond(0.6644,0.3853)
\PST@Diamond(0.3796,0.4722)
\PST@Diamond(0.4641,0.6695)
\PST@Diamond(0.6563,0.5172)
\PST@Diamond(0.5080,0.5489)
\PST@Diamond(0.5388,0.6156)
\PST@Diamond(0.6038,0.5418)
\PST@Diamond(0.5320,0.4399)
\PST@Diamond(0.4327,0.4581)
\PST@Diamond(0.4504,0.4817)
\PST@Diamond(0.4734,0.6734)
\PST@Diamond(0.6601,0.5630)
\PST@Diamond(0.5525,0.3905)
\PST@Diamond(0.3846,0.5756)
\PST@Diamond(0.5648,0.4607)
\PST@Diamond(0.4530,0.5576)
\PST@Diamond(0.5473,0.5540)
\PST@Diamond(0.5438,0.5588)
\PST@Diamond(0.5485,0.5670)
\PST@Diamond(0.5565,0.5825)
\PST@Diamond(0.5715,0.5336)
\PST@Diamond(0.5240,0.6132)
\PST@Diamond(0.6015,0.3924)
\PST@Diamond(0.3864,0.4996)
\PST@Diamond(0.4908,0.7129)
\PST@Diamond(0.6986,0.5619)
\PST@Diamond(0.5515,0.5227)
\PST@Diamond(0.5134,0.6972)
\PST@Diamond(0.6833,0.4546)
\PST@Diamond(0.4470,0.6004)
\PST@Diamond(0.5890,0.3094)
\PST@Diamond(0.3056,0.4298)
\PST@Diamond(0.4229,0.6164)
\PST@Diamond(0.6046,0.5455)
\PST@Diamond(0.5355,0.6098)
\PST@Diamond(0.5981,0.4601)
\PST@Diamond(0.4523,0.5813)
\PST@Diamond(0.5704,0.6816)
\PST@Diamond(0.6680,0.5708)
\PST@Diamond(0.5602,0.5677)
\PST@Diamond(0.5571,0.4984)
\PST@Diamond(0.4896,0.4983)
\PST@Diamond(0.4895,0.5770)
\PST@Diamond(0.5662,0.5208)
\PST@Diamond(0.5115,0.4859)
\PST@Diamond(0.4775,0.7083)
\PST@Diamond(0.6940,0.5733)
\PST@Diamond(0.5626,0.6142)
\PST@Diamond(0.6025,0.5248)
\PST@Diamond(0.5154,0.5847)
\PST@Diamond(0.5738,0.4641)
\PST@Diamond(0.4563,0.5420)
\PST@Diamond(0.5321,0.5398)
\PST@Diamond(0.5299,0.5269)
\PST@Diamond(0.5174,0.6901)
\PST@Diamond(0.6763,0.5118)
\PST@Diamond(0.5027,0.4365)
\PST@Diamond(0.4293,0.6521)
\PST@Diamond(0.6394,0.5760)
\PST@Diamond(0.5653,0.5651)
\PST@Diamond(0.5546,0.6093)
\PST@Diamond(0.5976,0.4829)
\PST@Diamond(0.4746,0.4031)
\PST@Diamond(0.3968,0.4265)
\PST@Diamond(0.4196,0.4895)
\PST@Diamond(0.4810,0.6824)
\PST@Diamond(0.6689,0.4953)
\PST@Diamond(0.4867,0.5443)
\PST@Diamond(0.5344,0.5297)
\PST@Diamond(0.5202,0.4886)
\PST@Diamond(0.4801,0.4991)
\PST@Diamond(0.4903,0.4403)
\PST@Diamond(0.4331,0.5858)
\PST@Diamond(0.5747,0.5035)
\PST@Diamond(0.4946,0.4540)
\PST@Diamond(0.4465,0.3049)
\PST@Diamond(0.3013,0.6467)
\PST@Diamond(0.6341,0.6475)
\PST@Diamond(0.6349,0.5548)
\PST@Diamond(0.5446,0.3906)
\PST@Diamond(0.3847,0.5305)
\PST@Diamond(0.5210,0.4892)
\PST@Diamond(0.4807,0.5383)
\PST@Diamond(0.5286,0.3530)
\PST@Diamond(0.3481,0.6773)
\PST@Diamond(0.6639,0.7227)
\PST@Diamond(0.7081,0.4359)
\PST@Diamond(0.4289,0.6355)
\PST@Diamond(0.6232,0.6574)
\PST@Diamond(0.6445,0.5709)
\PST@Diamond(0.5603,0.4239)
\PST@Diamond(0.4171,0.4652)
\PST@Diamond(0.4573,0.4045)
\PST@Diamond(0.3982,0.4687)
\PST@Diamond(0.4608,0.4978)
\PST@Diamond(0.4890,0.4970)
\PST@Diamond(0.4884,0.7178)
\PST@Diamond(0.7034,0.5266)
\PST@Diamond(0.5172,0.5107)
\PST@Diamond(0.5016,0.5989)
\PST@Diamond(0.5876,0.5562)
\PST@Diamond(0.5459,0.5365)
\PST@Diamond(0.5267,0.5537)
\PST@Diamond(0.5435,0.5171)
\PST@Diamond(0.5079,0.4368)
\PST@Diamond(0.4297,0.8779)
\PST@Diamond(0.8592,0.6710)
\PST@Diamond(0.6577,0.5859)
\PST@Diamond(0.5749,0.6242)
\PST@Diamond(0.6121,0.5609)
\PST@Diamond(0.5505,0.4778)
\PST@Diamond(0.4696,0.5587)
\PST@Diamond(0.5483,0.7551)
\PST@Diamond(0.7397,0.4544)
\PST@Diamond(0.4468,0.6870)
\PST@Diamond(0.6733,0.4327)
\PST@Diamond(0.4257,0.7019)
\PST@Diamond(0.6879,0.3328)
\PST@Diamond(0.3284,0.4400)
\PST@Diamond(0.4328,0.3106)
\PST@Diamond(0.3068,0.4937)
\PST@Diamond(0.4851,0.6157)
\PST@Diamond(0.6040,0.4543)
\PST@Diamond(0.4467,0.6394)
\PST@Diamond(0.6270,0.4680)
\PST@Diamond(0.4601,0.7245)
\PST@Diamond(0.7099,0.3584)
\catcode`@=12
\fi
\endpspicture
&
% GNUPLOT: LaTeX picture using PSTRICKS macros
% Define new PST objects, if not already defined
\ifx\PSTloaded\undefined
\def\PSTloaded{t}
\psset{arrowsize=.01 3.2 1.4 .3}
\psset{dotsize=.05}
\catcode`@=11

\newpsobject{PST@Border}{psline}{linewidth=.0015,linestyle=solid}
\newpsobject{PST@Axes}{psline}{linewidth=.0015,linestyle=dotted,dotsep=.004}
\newpsobject{PST@Solid}{psline}{linewidth=.0015,linestyle=solid}
\newpsobject{PST@Dashed}{psline}{linewidth=.0015,linestyle=dashed,dash=.01 .01}
\newpsobject{PST@Dotted}{psline}{linewidth=.0025,linestyle=dotted,dotsep=.008}
\newpsobject{PST@LongDash}{psline}{linewidth=.0015,linestyle=dashed,dash=.02 .01}
\newpsobject{PST@Diamond}{psdots}{linewidth=.01,linestyle=solid,dotstyle=square,dotangle=45}
\newpsobject{PST@Filldiamond}{psdots}{linewidth=.001,linestyle=solid,dotstyle=square*,dotangle=45}
\newpsobject{PST@Cross}{psdots}{linewidth=.001,linestyle=solid,dotstyle=+,dotangle=45}
\newpsobject{PST@Plus}{psdots}{linewidth=.001,linestyle=solid,dotstyle=+}
\newpsobject{PST@Square}{psdots}{linewidth=.001,linestyle=solid,dotstyle=square}
\newpsobject{PST@Circle}{psdots}{linewidth=.001,linestyle=solid,dotstyle=o}
\newpsobject{PST@Triangle}{psdots}{linewidth=.001,linestyle=solid,dotstyle=triangle}
\newpsobject{PST@Pentagon}{psdots}{linewidth=.001,linestyle=solid,dotstyle=pentagon}
\newpsobject{PST@Fillsquare}{psdots}{linewidth=.001,linestyle=solid,dotstyle=square*}
\newpsobject{PST@Fillcircle}{psdots}{linewidth=.001,linestyle=solid,dotstyle=*}
\newpsobject{PST@Filltriangle}{psdots}{linewidth=.001,linestyle=solid,dotstyle=triangle*}
\newpsobject{PST@Fillpentagon}{psdots}{linewidth=.001,linestyle=solid,dotstyle=pentagon*}
\newpsobject{PST@Arrow}{psline}{linewidth=.001,linestyle=solid}
\catcode`@=12

\fi
\psset{unit=5.000000in,xunit=2.4in,yunit=1.5in}
\pspicture(0.05,0)(.9,1)
\ifx\nofigs\undefined
\catcode`@=11

\PST@Border(0.1590,0.1260)
(0.1740,0.1260)

\PST@Border(0.9470,0.1260)
(0.9320,0.1260)

\rput[r](0.1430,0.1260){-2}
\PST@Border(0.1590,0.2663)
(0.1740,0.2663)

\PST@Border(0.9470,0.2663)
(0.9320,0.2663)

\rput[r](0.1430,0.2663){-1.5}
\PST@Border(0.1590,0.4067)
(0.1740,0.4067)

\PST@Border(0.9470,0.4067)
(0.9320,0.4067)

\rput[r](0.1430,0.4067){-1}
\PST@Border(0.1590,0.5470)
(0.1740,0.5470)

\PST@Border(0.9470,0.5470)
(0.9320,0.5470)

\rput[r](0.1430,0.5470){-0.5}
\PST@Border(0.1590,0.6873)
(0.1740,0.6873)

\PST@Border(0.9470,0.6873)
(0.9320,0.6873)

\rput[r](0.1430,0.6873){0}
\PST@Border(0.1590,0.8277)
(0.1740,0.8277)

\PST@Border(0.9470,0.8277)
(0.9320,0.8277)

\rput[r](0.1430,0.8277){0.5}
\PST@Border(0.1590,0.9680)
(0.1740,0.9680)

\PST@Border(0.9470,0.9680)
(0.9320,0.9680)

\rput[r](0.1430,0.9680){1}
\PST@Border(0.1590,0.1260)
(0.1590,0.1460)

\PST@Border(0.1590,0.9680)
(0.1590,0.9480)

\rput(0.1590,0.0840){-2}
\PST@Border(0.2903,0.1260)
(0.2903,0.1460)

\PST@Border(0.2903,0.9680)
(0.2903,0.9480)

\rput(0.2903,0.0840){-1.5}
\PST@Border(0.4217,0.1260)
(0.4217,0.1460)

\PST@Border(0.4217,0.9680)
(0.4217,0.9480)

\rput(0.4217,0.0840){-1}
\PST@Border(0.5530,0.1260)
(0.5530,0.1460)

\PST@Border(0.5530,0.9680)
(0.5530,0.9480)

\rput(0.5530,0.0840){-0.5}
\PST@Border(0.6843,0.1260)
(0.6843,0.1460)

\PST@Border(0.6843,0.9680)
(0.6843,0.9480)

\rput(0.6843,0.0840){0}
\PST@Border(0.8157,0.1260)
(0.8157,0.1460)

\PST@Border(0.8157,0.9680)
(0.8157,0.9480)

\rput(0.8157,0.0840){0.5}
\PST@Border(0.9470,0.1260)
(0.9470,0.1460)

\PST@Border(0.9470,0.9680)
(0.9470,0.9480)

\rput(0.9470,0.0840){1}
\PST@Border(0.1590,0.1260)
(0.9470,0.1260)
(0.9470,0.9680)
(0.1590,0.9680)
(0.1590,0.1260)

\rput{L}(0.0420,0.5470){$x_t$}
\rput(0.5530,0.0210){$x_{t-1}$}
\PST@Diamond(0.7442,0.8088)
\PST@Diamond(0.7980,0.7898)
\PST@Diamond(0.7802,0.6203)
\PST@Diamond(0.6216,0.7651)
\PST@Diamond(0.7571,0.6327)
\PST@Diamond(0.6332,0.6678)
\PST@Diamond(0.6661,0.6423)
\PST@Diamond(0.6422,0.4911)
\PST@Diamond(0.5007,0.4447)
\PST@Diamond(0.4572,0.3273)
\PST@Diamond(0.3474,0.2258)
\PST@Diamond(0.2524,0.3725)
\PST@Diamond(0.3897,0.5082)
\PST@Diamond(0.5167,0.6081)
\PST@Diamond(0.6102,0.5020)
\PST@Diamond(0.5109,0.4530)
\PST@Diamond(0.4651,0.3662)
\PST@Diamond(0.3838,0.4598)
\PST@Diamond(0.4714,0.4791)
\PST@Diamond(0.4895,0.5188)
\PST@Diamond(0.5266,0.6225)
\PST@Diamond(0.6237,0.7176)
\PST@Diamond(0.7126,0.6454)
\PST@Diamond(0.6451,0.5925)
\PST@Diamond(0.5955,0.5398)
\PST@Diamond(0.5463,0.4513)
\PST@Diamond(0.4634,0.5232)
\PST@Diamond(0.5308,0.4957)
\PST@Diamond(0.5050,0.5053)
\PST@Diamond(0.5140,0.5182)
\PST@Diamond(0.5260,0.5892)
\PST@Diamond(0.5925,0.5310)
\PST@Diamond(0.5380,0.4603)
\PST@Diamond(0.4719,0.3941)
\PST@Diamond(0.4099,0.4733)
\PST@Diamond(0.4841,0.4356)
\PST@Diamond(0.4488,0.4645)
\PST@Diamond(0.4758,0.4496)
\PST@Diamond(0.4618,0.4174)
\PST@Diamond(0.4317,0.4933)
\PST@Diamond(0.5028,0.4877)
\PST@Diamond(0.4975,0.4685)
\PST@Diamond(0.4796,0.5377)
\PST@Diamond(0.5443,0.5285)
\PST@Diamond(0.5357,0.4914)
\PST@Diamond(0.5010,0.4237)
\PST@Diamond(0.4376,0.4537)
\PST@Diamond(0.4657,0.4313)
\PST@Diamond(0.4447,0.4873)
\PST@Diamond(0.4972,0.4562)
\PST@Diamond(0.4680,0.5238)
\PST@Diamond(0.5313,0.5748)
\PST@Diamond(0.5790,0.5386)
\PST@Diamond(0.5451,0.4907)
\PST@Diamond(0.5003,0.4397)
\PST@Diamond(0.4526,0.4332)
\PST@Diamond(0.4465,0.3890)
\PST@Diamond(0.4051,0.4500)
\PST@Diamond(0.4622,0.4853)
\PST@Diamond(0.4952,0.5083)
\PST@Diamond(0.5168,0.5121)
\PST@Diamond(0.5204,0.5334)
\PST@Diamond(0.5403,0.5734)
\PST@Diamond(0.5777,0.5841)
\PST@Diamond(0.5877,0.6136)
\PST@Diamond(0.6153,0.6477)
\PST@Diamond(0.6473,0.6866)
\PST@Diamond(0.6836,0.6751)
\PST@Diamond(0.6729,0.6367)
\PST@Diamond(0.6369,0.5907)
\PST@Diamond(0.5939,0.5776)
\PST@Diamond(0.5816,0.5915)
\PST@Diamond(0.5947,0.5869)
\PST@Diamond(0.5903,0.5690)
\PST@Diamond(0.5736,0.5186)
\PST@Diamond(0.5264,0.5398)
\PST@Diamond(0.5463,0.5235)
\PST@Diamond(0.5310,0.5503)
\PST@Diamond(0.5560,0.4971)
\PST@Diamond(0.5063,0.4693)
\PST@Diamond(0.4803,0.5234)
\PST@Diamond(0.5310,0.5764)
\PST@Diamond(0.5806,0.5349)
\PST@Diamond(0.5417,0.5053)
\PST@Diamond(0.5140,0.4656)
\PST@Diamond(0.4769,0.4352)
\PST@Diamond(0.4484,0.4516)
\PST@Diamond(0.4637,0.4269)
\PST@Diamond(0.4406,0.3933)
\PST@Diamond(0.4092,0.3856)
\PST@Diamond(0.4020,0.4298)
\PST@Diamond(0.4433,0.4491)
\PST@Diamond(0.4613,0.4073)
\PST@Diamond(0.4223,0.4326)
\PST@Diamond(0.4460,0.4100)
\PST@Diamond(0.4248,0.4421)
\PST@Diamond(0.4549,0.4455)
\PST@Diamond(0.4580,0.4062)
\PST@Diamond(0.4213,0.3797)
\PST@Diamond(0.3964,0.3996)
\PST@Diamond(0.4151,0.3706)
\PST@Diamond(0.3879,0.3810)
\PST@Diamond(0.3977,0.3640)
\PST@Diamond(0.3818,0.3785)
\PST@Diamond(0.3953,0.4203)
\PST@Diamond(0.4345,0.4073)
\PST@Diamond(0.4222,0.3916)
\PST@Diamond(0.4075,0.3715)
\PST@Diamond(0.3888,0.3343)
\PST@Diamond(0.3539,0.3194)
\PST@Diamond(0.3400,0.3212)
\PST@Diamond(0.3417,0.2839)
\PST@Diamond(0.3068,0.2683)
\PST@Diamond(0.2922,0.2795)
\PST@Diamond(0.3027,0.2620)
\PST@Diamond(0.2862,0.2566)
\PST@Diamond(0.2812,0.2853)
\PST@Diamond(0.3081,0.3253)
\PST@Diamond(0.3455,0.3417)
\PST@Diamond(0.3608,0.3836)
\PST@Diamond(0.4001,0.3706)
\PST@Diamond(0.3880,0.3796)
\PST@Diamond(0.3963,0.3927)
\PST@Diamond(0.4086,0.3860)
\PST@Diamond(0.4023,0.3721)
\PST@Diamond(0.3893,0.4073)
\PST@Diamond(0.4223,0.3844)
\PST@Diamond(0.4009,0.4154)
\PST@Diamond(0.4298,0.4573)
\PST@Diamond(0.4690,0.4525)
\PST@Diamond(0.4646,0.4566)
\PST@Diamond(0.4684,0.4291)
\PST@Diamond(0.4426,0.3902)
\PST@Diamond(0.4063,0.4201)
\PST@Diamond(0.4343,0.4461)
\PST@Diamond(0.4586,0.4847)
\PST@Diamond(0.4947,0.4599)
\PST@Diamond(0.4714,0.4292)
\PST@Diamond(0.4428,0.4074)
\PST@Diamond(0.4223,0.3899)
\PST@Diamond(0.4060,0.3860)
\PST@Diamond(0.4024,0.4054)
\PST@Diamond(0.4205,0.3942)
\PST@Diamond(0.4100,0.4152)
\PST@Diamond(0.4297,0.4024)
\PST@Diamond(0.4177,0.4142)
\PST@Diamond(0.4287,0.3886)
\PST@Diamond(0.4048,0.3601)
\PST@Diamond(0.3781,0.3663)
\PST@Diamond(0.3839,0.3940)
\PST@Diamond(0.4098,0.3624)
\PST@Diamond(0.3803,0.3952)
\PST@Diamond(0.4109,0.3901)
\PST@Diamond(0.4062,0.4085)
\PST@Diamond(0.4234,0.3948)
\PST@Diamond(0.4106,0.4151)
\PST@Diamond(0.4296,0.4250)
\PST@Diamond(0.4388,0.4284)
\PST@Diamond(0.4420,0.4367)
\PST@Diamond(0.4498,0.4448)
\PST@Diamond(0.4574,0.4813)
\PST@Diamond(0.4915,0.4541)
\PST@Diamond(0.4661,0.4745)
\PST@Diamond(0.4851,0.5125)
\PST@Diamond(0.5207,0.5491)
\PST@Diamond(0.5550,0.5540)
\PST@Diamond(0.5595,0.5879)
\PST@Diamond(0.5913,0.5635)
\PST@Diamond(0.5685,0.5771)
\PST@Diamond(0.5812,0.5529)
\PST@Diamond(0.5585,0.5492)
\PST@Diamond(0.5551,0.5213)
\PST@Diamond(0.5289,0.5509)
\PST@Diamond(0.5567,0.5729)
\PST@Diamond(0.5772,0.6000)
\PST@Diamond(0.6026,0.5801)
\PST@Diamond(0.5840,0.5752)
\PST@Diamond(0.5794,0.5416)
\PST@Diamond(0.5480,0.5314)
\PST@Diamond(0.5384,0.5675)
\PST@Diamond(0.5722,0.5945)
\PST@Diamond(0.5975,0.5905)
\PST@Diamond(0.5937,0.6192)
\PST@Diamond(0.6206,0.6047)
\PST@Diamond(0.6070,0.5809)
\PST@Diamond(0.5847,0.5605)
\PST@Diamond(0.5657,0.5285)
\PST@Diamond(0.5357,0.5487)
\PST@Diamond(0.5546,0.5663)
\PST@Diamond(0.5710,0.5764)
\PST@Diamond(0.5805,0.5680)
\PST@Diamond(0.5727,0.5831)
\PST@Diamond(0.5868,0.6025)
\PST@Diamond(0.6050,0.5773)
\PST@Diamond(0.5814,0.5919)
\PST@Diamond(0.5950,0.6096)
\PST@Diamond(0.6116,0.6232)
\PST@Diamond(0.6243,0.6339)
\PST@Diamond(0.6343,0.6631)
\PST@Diamond(0.6617,0.6543)
\PST@Diamond(0.6534,0.6765)
\PST@Diamond(0.6742,0.6675)
\PST@Diamond(0.6657,0.6667)
\PST@Diamond(0.6650,0.6815)
\PST@Diamond(0.6789,0.6586)
\PST@Diamond(0.6575,0.6487)
\PST@Diamond(0.6482,0.6787)
\PST@Diamond(0.6762,0.6848)
\PST@Diamond(0.6819,0.6772)
\PST@Diamond(0.6748,0.6636)
\PST@Diamond(0.6621,0.6692)
\PST@Diamond(0.6674,0.6530)
\PST@Diamond(0.6522,0.6689)
\PST@Diamond(0.6670,0.6676)
\PST@Diamond(0.6659,0.6710)
\PST@Diamond(0.6690,0.6975)
\PST@Diamond(0.6938,0.7100)
\PST@Diamond(0.7056,0.7161)
\PST@Diamond(0.7112,0.7278)
\PST@Diamond(0.7222,0.7235)
\PST@Diamond(0.7182,0.7059)
\PST@Diamond(0.7017,0.6770)
\PST@Diamond(0.6747,0.6541)
\PST@Diamond(0.6532,0.6222)
\PST@Diamond(0.6233,0.6023)
\PST@Diamond(0.6048,0.6252)
\PST@Diamond(0.6262,0.6421)
\PST@Diamond(0.6420,0.6669)
\PST@Diamond(0.6652,0.6673)
\PST@Diamond(0.6656,0.6790)
\PST@Diamond(0.6765,0.6634)
\PST@Diamond(0.6620,0.6525)
\PST@Diamond(0.6517,0.6240)
\PST@Diamond(0.6250,0.6395)
\PST@Diamond(0.6396,0.6109)
\PST@Diamond(0.6128,0.5928)
\PST@Diamond(0.5959,0.5676)
\PST@Diamond(0.5723,0.5985)
\PST@Diamond(0.6012,0.6177)
\PST@Diamond(0.6192,0.6167)
\PST@Diamond(0.6183,0.6034)
\PST@Diamond(0.6058,0.5965)
\PST@Diamond(0.5993,0.6116)
\PST@Diamond(0.6134,0.5820)
\PST@Diamond(0.5857,0.6050)
\PST@Diamond(0.6072,0.5921)
\PST@Diamond(0.5952,0.6183)
\PST@Diamond(0.6197,0.6220)
\PST@Diamond(0.6232,0.6456)
\PST@Diamond(0.6453,0.6520)
\PST@Diamond(0.6512,0.6823)
\PST@Diamond(0.6796,0.6587)
\PST@Diamond(0.6575,0.6684)
\PST@Diamond(0.6666,0.6465)
\PST@Diamond(0.6461,0.6235)
\PST@Diamond(0.6246,0.6449)
\PST@Diamond(0.6446,0.6747)
\PST@Diamond(0.6725,0.6977)
\PST@Diamond(0.6941,0.7118)
\PST@Diamond(0.7073,0.7116)
\PST@Diamond(0.7071,0.7153)
\PST@Diamond(0.7105,0.7447)
\PST@Diamond(0.7380,0.7640)
\PST@Diamond(0.7560,0.7706)
\PST@Diamond(0.7622,0.7843)
\PST@Diamond(0.7751,0.8062)
\PST@Diamond(0.7955,0.8252)
\PST@Diamond(0.8134,0.8445)
\PST@Diamond(0.8314,0.8654)
\PST@Diamond(0.8510,0.8726)
\PST@Diamond(0.8577,0.8612)
\PST@Diamond(0.8470,0.8398)
\PST@Diamond(0.8270,0.8109)
\PST@Diamond(0.8000,0.7844)
\PST@Diamond(0.7752,0.7646)
\PST@Diamond(0.7566,0.7866)
\PST@Diamond(0.7773,0.7772)
\PST@Diamond(0.7684,0.7529)
\PST@Diamond(0.7457,0.7492)
\PST@Diamond(0.7422,0.7330)
\PST@Diamond(0.7271,0.7437)
\PST@Diamond(0.7371,0.7397)
\PST@Diamond(0.7333,0.7302)
\PST@Diamond(0.7245,0.7212)
\PST@Diamond(0.7160,0.7253)
\PST@Diamond(0.7199,0.7228)
\PST@Diamond(0.7175,0.7052)
\PST@Diamond(0.7010,0.7089)
\PST@Diamond(0.7045,0.6996)
\PST@Diamond(0.6958,0.7240)
\PST@Diamond(0.7186,0.6990)
\PST@Diamond(0.6952,0.7217)
\PST@Diamond(0.7165,0.7454)
\PST@Diamond(0.7386,0.7671)
\PST@Diamond(0.7590,0.7678)
\PST@Diamond(0.7596,0.7762)
\PST@Diamond(0.7675,0.7905)
\PST@Diamond(0.7808,0.7812)
\PST@Diamond(0.7722,0.7800)
\PST@Diamond(0.7711,0.7862)
\PST@Diamond(0.7769,0.8121)
\PST@Diamond(0.8011,0.8283)
\PST@Diamond(0.8163,0.8424)
\PST@Diamond(0.8295,0.8688)
\PST@Diamond(0.8541,0.8877)
\PST@Diamond(0.8719,0.9107)
\PST@Diamond(0.8933,0.9302)
\PST@Diamond(0.9116,0.9125)
\PST@Diamond(0.8950,0.8848)
\PST@Diamond(0.8691,0.8733)
\PST@Diamond(0.8584,0.8680)
\PST@Diamond(0.8534,0.8782)
\PST@Diamond(0.8630,0.8905)
\PST@Diamond(0.8745,0.9038)
\PST@Diamond(0.8869,0.8780)
\PST@Diamond(0.8627,0.8669)
\PST@Diamond(0.8524,0.8505)
\PST@Diamond(0.8371,0.8355)
\PST@Diamond(0.8230,0.8553)
\PST@Diamond(0.8415,0.8573)
\PST@Diamond(0.8434,0.8384)
\PST@Diamond(0.8257,0.8614)
\PST@Diamond(0.8473,0.8582)
\PST@Diamond(0.8442,0.8349)
\PST@Diamond(0.8225,0.8517)
\PST@Diamond(0.8382,0.8761)
\PST@Diamond(0.8610,0.8879)
\PST@Diamond(0.8721,0.8915)
\PST@Diamond(0.8754,0.8802)
\PST@Diamond(0.8649,0.8643)
\PST@Diamond(0.8500,0.8473)
\PST@Diamond(0.8340,0.8343)
\PST@Diamond(0.8219,0.8076)
\PST@Diamond(0.7969,0.8309)
\PST@Diamond(0.8187,0.8169)
\PST@Diamond(0.8056,0.8352)
\PST@Diamond(0.8227,0.8540)
\PST@Diamond(0.8404,0.8326)
\PST@Diamond(0.8203,0.8079)
\PST@Diamond(0.7972,0.8270)
\PST@Diamond(0.8150,0.8213)
\PST@Diamond(0.8097,0.8183)
\PST@Diamond(0.8069,0.8210)
\PST@Diamond(0.8094,0.8013)
\PST@Diamond(0.7910,0.7850)
\PST@Diamond(0.7758,0.7896)
\PST@Diamond(0.7801,0.7858)
\PST@Diamond(0.7765,0.7803)
\PST@Diamond(0.7714,0.7968)
\PST@Diamond(0.7867,0.7861)
\PST@Diamond(0.7768,0.8088)
\PST@Diamond(0.7980,0.8332)
\PST@Diamond(0.8209,0.8189)
\PST@Diamond(0.8075,0.8128)
\PST@Diamond(0.8018,0.7894)
\PST@Diamond(0.7798,0.7657)
\PST@Diamond(0.7577,0.7576)
\PST@Diamond(0.7501,0.7716)
\PST@Diamond(0.7632,0.7773)
\PST@Diamond(0.7685,0.7843)
\PST@Diamond(0.7751,0.8089)
\PST@Diamond(0.7981,0.8240)
\PST@Diamond(0.8122,0.8442)
\PST@Diamond(0.8312,0.8687)
\PST@Diamond(0.8541,0.8807)
\PST@Diamond(0.8653,0.8622)
\PST@Diamond(0.8480,0.8789)
\PST@Diamond(0.8636,0.8838)
\PST@Diamond(0.8682,0.8704)
\PST@Diamond(0.8556,0.8888)
\PST@Diamond(0.8729,0.8869)
\PST@Diamond(0.8711,0.8935)
\PST@Diamond(0.8772,0.8839)
\PST@Diamond(0.8683,0.8597)
\PST@Diamond(0.8456,0.8726)
\PST@Diamond(0.8577,0.8727)
\PST@Diamond(0.8579,0.8782)
\PST@Diamond(0.8630,0.8626)
\PST@Diamond(0.8483,0.8826)
\PST@Diamond(0.8670,0.8790)
\PST@Diamond(0.8637,0.8782)
\PST@Diamond(0.8630,0.8952)
\PST@Diamond(0.8788,0.8903)
\PST@Diamond(0.8743,0.9008)
\PST@Diamond(0.8841,0.8872)
\PST@Diamond(0.8714,0.8848)
\PST@Diamond(0.8692,0.8974)
\PST@Diamond(0.8809,0.9008)
\PST@Diamond(0.8841,0.8873)
\PST@Diamond(0.8715,0.8809)
\PST@Diamond(0.8654,0.8666)
\PST@Diamond(0.8521,0.8523)
\PST@Diamond(0.8387,0.8362)
\PST@Diamond(0.8236,0.8172)
\PST@Diamond(0.8058,0.8023)
\PST@Diamond(0.7919,0.8224)
\PST@Diamond(0.8108,0.8104)
\PST@Diamond(0.7995,0.7875)
\PST@Diamond(0.7780,0.7882)
\PST@Diamond(0.7788,0.7879)
\PST@Diamond(0.7784,0.8072)
\PST@Diamond(0.7965,0.8306)
\PST@Diamond(0.8185,0.8126)
\PST@Diamond(0.8016,0.7987)
\PST@Diamond(0.7886,0.7750)
\PST@Diamond(0.7664,0.7938)
\PST@Diamond(0.7840,0.8045)
\PST@Diamond(0.7939,0.8103)
\PST@Diamond(0.7994,0.7902)
\PST@Diamond(0.7806,0.7963)
\PST@Diamond(0.7863,0.7750)
\PST@Diamond(0.7664,0.7784)
\PST@Diamond(0.7696,0.7772)
\PST@Diamond(0.7684,0.7754)
\PST@Diamond(0.7668,0.7653)
\PST@Diamond(0.7573,0.7751)
\PST@Diamond(0.7665,0.7951)
\PST@Diamond(0.7852,0.7736)
\PST@Diamond(0.7651,0.7632)
\PST@Diamond(0.7553,0.7467)
\PST@Diamond(0.7399,0.7430)
\PST@Diamond(0.7365,0.7428)
\PST@Diamond(0.7362,0.7365)
\PST@Diamond(0.7304,0.7411)
\PST@Diamond(0.7346,0.7463)
\PST@Diamond(0.7395,0.7494)
\PST@Diamond(0.7424,0.7500)
\PST@Diamond(0.7430,0.7672)
\PST@Diamond(0.7591,0.7717)
\PST@Diamond(0.7633,0.7498)
\PST@Diamond(0.7428,0.7433)
\PST@Diamond(0.7367,0.7433)
\PST@Diamond(0.7367,0.7208)
\PST@Diamond(0.7156,0.7204)
\PST@Diamond(0.7153,0.7305)
\PST@Diamond(0.7247,0.7086)
\PST@Diamond(0.7043,0.7033)
\PST@Diamond(0.6993,0.7006)
\PST@Diamond(0.6967,0.7078)
\PST@Diamond(0.7035,0.7064)
\PST@Diamond(0.7022,0.6865)
\PST@Diamond(0.6835,0.6964)
\PST@Diamond(0.6928,0.6754)
\PST@Diamond(0.6731,0.6775)
\PST@Diamond(0.6751,0.6628)
\PST@Diamond(0.6613,0.6552)
\PST@Diamond(0.6543,0.6439)
\PST@Diamond(0.6437,0.6257)
\PST@Diamond(0.6267,0.6205)
\PST@Diamond(0.6218,0.6222)
\PST@Diamond(0.6234,0.6112)
\PST@Diamond(0.6131,0.6254)
\PST@Diamond(0.6263,0.6041)
\PST@Diamond(0.6064,0.6099)
\PST@Diamond(0.6118,0.6057)
\PST@Diamond(0.6080,0.6123)
\PST@Diamond(0.6142,0.5985)
\PST@Diamond(0.6012,0.6177)
\PST@Diamond(0.6192,0.6183)
\PST@Diamond(0.6197,0.6316)
\PST@Diamond(0.6321,0.6522)
\PST@Diamond(0.6514,0.6691)
\PST@Diamond(0.6673,0.6598)
\PST@Diamond(0.6586,0.6812)
\PST@Diamond(0.6786,0.6753)
\PST@Diamond(0.6730,0.6532)
\PST@Diamond(0.6524,0.6310)
\PST@Diamond(0.6316,0.6425)
\PST@Diamond(0.6423,0.6403)
\PST@Diamond(0.6403,0.6475)
\PST@Diamond(0.6471,0.6353)
\PST@Diamond(0.6356,0.6362)
\PST@Diamond(0.6365,0.6307)
\PST@Diamond(0.6314,0.6205)
\PST@Diamond(0.6218,0.6012)
\PST@Diamond(0.6037,0.6038)
\PST@Diamond(0.6061,0.6085)
\PST@Diamond(0.6106,0.6005)
\PST@Diamond(0.6031,0.6077)
\PST@Diamond(0.6098,0.6294)
\PST@Diamond(0.6301,0.6452)
\PST@Diamond(0.6449,0.6636)
\PST@Diamond(0.6621,0.6768)
\PST@Diamond(0.6745,0.6938)
\PST@Diamond(0.6904,0.6957)
\PST@Diamond(0.6921,0.6827)
\PST@Diamond(0.6800,0.6839)
\PST@Diamond(0.6812,0.6942)
\PST@Diamond(0.6908,0.6778)
\PST@Diamond(0.6754,0.6577)
\PST@Diamond(0.6566,0.6588)
\PST@Diamond(0.6577,0.6405)
\PST@Diamond(0.6405,0.6587)
\PST@Diamond(0.6576,0.6727)
\PST@Diamond(0.6706,0.6533)
\PST@Diamond(0.6525,0.6438)
\PST@Diamond(0.6436,0.6581)
\PST@Diamond(0.6570,0.6391)
\PST@Diamond(0.6392,0.6190)
\PST@Diamond(0.6203,0.6094)
\PST@Diamond(0.6114,0.6191)
\PST@Diamond(0.6205,0.6088)
\PST@Diamond(0.6108,0.6218)
\PST@Diamond(0.6230,0.6044)
\PST@Diamond(0.6068,0.6058)
\PST@Diamond(0.6081,0.6217)
\PST@Diamond(0.6229,0.6283)
\PST@Diamond(0.6291,0.6126)
\PST@Diamond(0.6144,0.5991)
\PST@Diamond(0.6017,0.5911)
\PST@Diamond(0.5942,0.5750)
\PST@Diamond(0.5792,0.5552)
\PST@Diamond(0.5607,0.5437)
\PST@Diamond(0.5499,0.5618)
\PST@Diamond(0.5669,0.5801)
\PST@Diamond(0.5840,0.5917)
\PST@Diamond(0.5948,0.5758)
\PST@Diamond(0.5800,0.5739)
\PST@Diamond(0.5781,0.5742)
\PST@Diamond(0.5784,0.5637)
\PST@Diamond(0.5686,0.5635)
\PST@Diamond(0.5684,0.5436)
\PST@Diamond(0.5498,0.5367)
\PST@Diamond(0.5434,0.5329)
\PST@Diamond(0.5398,0.5478)
\PST@Diamond(0.5537,0.5432)
\PST@Diamond(0.5495,0.5514)
\PST@Diamond(0.5571,0.5589)
\PST@Diamond(0.5642,0.5570)
\PST@Diamond(0.5623,0.5666)
\PST@Diamond(0.5714,0.5859)
\PST@Diamond(0.5894,0.5722)
\PST@Diamond(0.5765,0.5928)
\PST@Diamond(0.5958,0.6035)
\PST@Diamond(0.6058,0.5938)
\PST@Diamond(0.5968,0.5947)
\PST@Diamond(0.5976,0.5997)
\PST@Diamond(0.6023,0.6174)
\PST@Diamond(0.6189,0.6238)
\PST@Diamond(0.6249,0.6365)
\PST@Diamond(0.6368,0.6254)
\PST@Diamond(0.6264,0.6369)
\PST@Diamond(0.6371,0.6511)
\PST@Diamond(0.6504,0.6495)
\PST@Diamond(0.6489,0.6576)
\PST@Diamond(0.6565,0.6686)
\PST@Diamond(0.6668,0.6574)
\PST@Diamond(0.6563,0.6707)
\PST@Diamond(0.6688,0.6589)
\PST@Diamond(0.6578,0.6687)
\PST@Diamond(0.6669,0.6510)
\PST@Diamond(0.6503,0.6596)
\PST@Diamond(0.6584,0.6706)
\PST@Diamond(0.6686,0.6667)
\PST@Diamond(0.6650,0.6508)
\PST@Diamond(0.6502,0.6555)
\PST@Diamond(0.6545,0.6677)
\PST@Diamond(0.6659,0.6802)
\PST@Diamond(0.6776,0.6715)
\PST@Diamond(0.6695,0.6611)
\PST@Diamond(0.6598,0.6623)
\PST@Diamond(0.6609,0.6517)
\PST@Diamond(0.6510,0.6484)
\PST@Diamond(0.6479,0.6492)
\PST@Diamond(0.6486,0.6694)
\PST@Diamond(0.6675,0.6770)
\PST@Diamond(0.6747,0.6581)
\PST@Diamond(0.6569,0.6626)
\PST@Diamond(0.6612,0.6671)
\PST@Diamond(0.6654,0.6747)
\PST@Diamond(0.6725,0.6712)
\PST@Diamond(0.6692,0.6851)
\PST@Diamond(0.6822,0.6835)
\PST@Diamond(0.6807,0.6734)
\PST@Diamond(0.6713,0.6655)
\PST@Diamond(0.6639,0.6515)
\PST@Diamond(0.6508,0.6320)
\PST@Diamond(0.6325,0.6334)
\PST@Diamond(0.6338,0.6526)
\PST@Diamond(0.6518,0.6418)
\PST@Diamond(0.6417,0.6325)
\PST@Diamond(0.6330,0.6143)
\PST@Diamond(0.6160,0.6320)
\PST@Diamond(0.6325,0.6132)
\PST@Diamond(0.6150,0.6115)
\PST@Diamond(0.6133,0.5935)
\PST@Diamond(0.5966,0.5994)
\PST@Diamond(0.6021,0.5895)
\PST@Diamond(0.5928,0.6038)
\PST@Diamond(0.6062,0.6212)
\PST@Diamond(0.6224,0.6211)
\PST@Diamond(0.6223,0.6166)
\PST@Diamond(0.6181,0.6036)
\PST@Diamond(0.6060,0.6202)
\PST@Diamond(0.6215,0.6363)
\PST@Diamond(0.6366,0.6231)
\PST@Diamond(0.6242,0.6273)
\PST@Diamond(0.6281,0.6447)
\PST@Diamond(0.6444,0.6558)
\PST@Diamond(0.6548,0.6444)
\PST@Diamond(0.6441,0.6493)
\PST@Diamond(0.6487,0.6371)
\PST@Diamond(0.6373,0.6195)
\PST@Diamond(0.6209,0.6031)
\PST@Diamond(0.6055,0.6010)
\PST@Diamond(0.6035,0.5954)
\PST@Diamond(0.5983,0.5851)
\PST@Diamond(0.5887,0.5837)
\PST@Diamond(0.5873,0.5991)
\PST@Diamond(0.6018,0.5879)
\PST@Diamond(0.5913,0.5955)
\PST@Diamond(0.5984,0.5821)
\PST@Diamond(0.5859,0.5729)
\PST@Diamond(0.5772,0.5780)
\PST@Diamond(0.5820,0.5660)
\PST@Diamond(0.5708,0.5745)
\PST@Diamond(0.5788,0.5817)
\PST@Diamond(0.5855,0.5949)
\PST@Diamond(0.5978,0.6132)
\PST@Diamond(0.6149,0.6147)
\PST@Diamond(0.6163,0.6251)
\PST@Diamond(0.6261,0.6238)
\PST@Diamond(0.6248,0.6402)
\PST@Diamond(0.6403,0.6574)
\PST@Diamond(0.6563,0.6527)
\PST@Diamond(0.6519,0.6653)
\PST@Diamond(0.6637,0.6501)
\PST@Diamond(0.6495,0.6688)
\PST@Diamond(0.6670,0.6790)
\PST@Diamond(0.6765,0.6940)
\PST@Diamond(0.6905,0.6821)
\PST@Diamond(0.6794,0.6776)
\PST@Diamond(0.6752,0.6614)
\PST@Diamond(0.6600,0.6516)
\PST@Diamond(0.6509,0.6503)
\PST@Diamond(0.6497,0.6513)
\PST@Diamond(0.6506,0.6553)
\PST@Diamond(0.6544,0.6632)
\PST@Diamond(0.6617,0.6819)
\PST@Diamond(0.6793,0.6816)
\PST@Diamond(0.6790,0.6977)
\PST@Diamond(0.6940,0.7045)
\PST@Diamond(0.7004,0.7102)
\PST@Diamond(0.7057,0.6979)
\PST@Diamond(0.6942,0.6905)
\PST@Diamond(0.6873,0.7035)
\PST@Diamond(0.6995,0.7186)
\PST@Diamond(0.7136,0.7373)
\PST@Diamond(0.7311,0.7439)
\PST@Diamond(0.7372,0.7576)
\PST@Diamond(0.7501,0.7585)
\PST@Diamond(0.7510,0.7562)
\PST@Diamond(0.7488,0.7496)
\PST@Diamond(0.7426,0.7475)
\PST@Diamond(0.7407,0.7430)
\PST@Diamond(0.7364,0.7507)
\PST@Diamond(0.7436,0.7420)
\PST@Diamond(0.7355,0.7416)
\PST@Diamond(0.7351,0.7485)
\PST@Diamond(0.7416,0.7308)
\PST@Diamond(0.7251,0.7261)
\PST@Diamond(0.7207,0.7404)
\PST@Diamond(0.7340,0.7372)
\PST@Diamond(0.7310,0.7356)
\PST@Diamond(0.7295,0.7214)
\PST@Diamond(0.7162,0.7358)
\PST@Diamond(0.7297,0.7539)
\PST@Diamond(0.7466,0.7624)
\PST@Diamond(0.7546,0.7656)
\PST@Diamond(0.7576,0.7831)
\PST@Diamond(0.7740,0.7726)
\PST@Diamond(0.7641,0.7727)
\PST@Diamond(0.7642,0.7781)
\PST@Diamond(0.7693,0.7919)
\PST@Diamond(0.7822,0.7987)
\PST@Diamond(0.7886,0.8156)
\PST@Diamond(0.8044,0.8233)
\PST@Diamond(0.8116,0.8262)
\PST@Diamond(0.8143,0.8426)
\PST@Diamond(0.8297,0.8382)
\PST@Diamond(0.8255,0.8359)
\PST@Diamond(0.8234,0.8346)
\PST@Diamond(0.8222,0.8466)
\PST@Diamond(0.8334,0.8565)
\PST@Diamond(0.8426,0.8718)
\PST@Diamond(0.8569,0.8607)
\PST@Diamond(0.8466,0.8596)
\PST@Diamond(0.8456,0.8479)
\PST@Diamond(0.8346,0.8550)
\PST@Diamond(0.8413,0.8423)
\PST@Diamond(0.8293,0.8319)
\PST@Diamond(0.8196,0.8160)
\PST@Diamond(0.8047,0.7988)
\PST@Diamond(0.7886,0.8038)
\PST@Diamond(0.7933,0.8048)
\PST@Diamond(0.7942,0.7922)
\PST@Diamond(0.7825,0.7929)
\PST@Diamond(0.7832,0.7932)
\PST@Diamond(0.7834,0.8075)
\PST@Diamond(0.7968,0.7930)
\PST@Diamond(0.7833,0.7922)
\PST@Diamond(0.7824,0.7778)
\PST@Diamond(0.7690,0.7819)
\PST@Diamond(0.7728,0.7681)
\PST@Diamond(0.7599,0.7856)
\PST@Diamond(0.7763,0.7782)
\PST@Diamond(0.7693,0.7627)
\PST@Diamond(0.7549,0.7696)
\PST@Diamond(0.7613,0.7833)
\PST@Diamond(0.7742,0.7658)
\PST@Diamond(0.7578,0.7502)
\PST@Diamond(0.7432,0.7437)
\PST@Diamond(0.7371,0.7433)
\PST@Diamond(0.7367,0.7577)
\PST@Diamond(0.7502,0.7427)
\PST@Diamond(0.7362,0.7393)
\PST@Diamond(0.7329,0.7249)
\PST@Diamond(0.7195,0.7271)
\PST@Diamond(0.7215,0.7305)
\PST@Diamond(0.7247,0.7413)
\PST@Diamond(0.7348,0.7492)
\PST@Diamond(0.7423,0.7606)
\PST@Diamond(0.7529,0.7740)
\PST@Diamond(0.7654,0.7832)
\PST@Diamond(0.7740,0.7814)
\PST@Diamond(0.7724,0.7778)
\PST@Diamond(0.7690,0.7928)
\PST@Diamond(0.7830,0.8098)
\PST@Diamond(0.7990,0.8245)
\PST@Diamond(0.8127,0.8355)
\PST@Diamond(0.8230,0.8204)
\PST@Diamond(0.8089,0.8164)
\PST@Diamond(0.8051,0.8314)
\PST@Diamond(0.8191,0.8383)
\PST@Diamond(0.8256,0.8387)
\PST@Diamond(0.8260,0.8529)
\PST@Diamond(0.8393,0.8705)
\PST@Diamond(0.8558,0.8737)
\PST@Diamond(0.8588,0.8768)
\PST@Diamond(0.8616,0.8900)
\PST@Diamond(0.8740,0.8940)
\PST@Diamond(0.8777,0.8996)
\PST@Diamond(0.8830,0.8886)
\PST@Diamond(0.8727,0.8744)
\PST@Diamond(0.8594,0.8764)
\PST@Diamond(0.8613,0.8686)
\PST@Diamond(0.8540,0.8689)
\PST@Diamond(0.8542,0.8745)
\PST@Diamond(0.8595,0.8867)
\PST@Diamond(0.8709,0.8726)
\PST@Diamond(0.8577,0.8711)
\PST@Diamond(0.8563,0.8734)
\PST@Diamond(0.8585,0.8881)
\PST@Diamond(0.8723,0.8822)
\PST@Diamond(0.8667,0.8758)
\PST@Diamond(0.8607,0.8712)
\PST@Diamond(0.8564,0.8794)
\PST@Diamond(0.8641,0.8702)
\PST@Diamond(0.8555,0.8647)
\PST@Diamond(0.8503,0.8697)
\PST@Diamond(0.8550,0.8538)
\PST@Diamond(0.8402,0.8512)
\PST@Diamond(0.8377,0.8348)
\PST@Diamond(0.8223,0.8513)
\PST@Diamond(0.8378,0.8381)
\PST@Diamond(0.8254,0.8397)
\PST@Diamond(0.8269,0.8527)
\PST@Diamond(0.8391,0.8393)
\PST@Diamond(0.8265,0.8267)
\PST@Diamond(0.8147,0.8252)
\PST@Diamond(0.8134,0.8424)
\PST@Diamond(0.8295,0.8514)
\PST@Diamond(0.8378,0.8381)
\PST@Diamond(0.8254,0.8271)
\PST@Diamond(0.8151,0.8397)
\PST@Diamond(0.8269,0.8460)
\PST@Diamond(0.8328,0.8449)
\PST@Diamond(0.8318,0.8405)
\PST@Diamond(0.8277,0.8350)
\PST@Diamond(0.8226,0.8286)
\PST@Diamond(0.8166,0.8279)
\PST@Diamond(0.8159,0.8385)
\PST@Diamond(0.8258,0.8518)
\PST@Diamond(0.8383,0.8483)
\PST@Diamond(0.8350,0.8358)
\catcode`@=12
\fi
\endpspicture

\\
UNURAN Gaussian method &
Basic Gaussian method \\
\end{tabular}

\section{Graph library}

The \EcoLab{} graph library is a library providing a simple and
lightweight structure for representing graphs (aka networks). A graph
consists of a set of nodes labelled $N=\{0\ldots n-1\}$, and a set of
edges $E\subset N\times N$, which have an optional weight factor
attached.

An edge is represented by\index{Edge}
\begin{verbatim}
struct Edge: public std::pair<unsigned,unsigned>
{
    unsigned source();
    unsigned target();
    float weight;
};
\end{verbatim}

The abstract Graph\index{Graph} interface has the following definition:
\begin{verbatim}
struct Graph
{
  struct const_iterator
  {
    Edge operator*() const;
    const Edge* operator->() const;
    const_iterator& operator++();
    bool operator==(const const_iterator& x) const;
    bool operator!=(const const_iterator& x) const;
  };
 
  virtual const_iterator begin() const;
  virtual const_iterator end() const;

  virtual unsigned nodes() const;
  virtual unsigned links() const;
  virtual void push_back(const Edge& e);
  virtual bool contains(const Edge& e) const;
  virtual bool directed() const;
  virtual void clear(unsigned nodes=0);
  const Graph& operator=(const Graph& x);
  void input(const std::string& format, const std::string& filename);
  void output(const std::string& format, const std::string& filename) const;
  template <class BG> Graph_back_insert_iterator<Graph,BG>
    back_inserter(const BG& bg);
}
\end{verbatim}

The \verb+begin/end+ methods allow one to iterate over the edges. Only a \verb+const_iterator+
is supplied, as it is an error to change the value of an edge. One can
only reset a graph to the empty graph via clear, and construct the
graph incrementally using the \verb+push_back()+ method.

The \verb+contains+ method allows one to test whether a given edge is
in the graph, and \verb+directed+ indicates whether the underlying graph
structure has directed edges or not. A bidirectional graph otherwise
appears as a directed graph where each edge appears twice, once for
each direction.

The input/output methods allow for the graph to be read/written
from/to a file, in a variety of formats, given by the format
parameter. Currently, the following formats are supported:
\begin{tabular}{|l|l|}
\hline
name & description\\
\hline
pajek\index{pajek} & Pajek's .net format\\
lgl\index{lgl} & LGL's .lgl format\\
dot\index{dot} & Graphviz\index{Graphviz} format\\
gengraph & Gengraph\index{Gengraph}\\
\hline
\end{tabular}
Also graphs can be streamed to/from standard I/O streams, and will appear
in dot format. In particular, this means that TCL scripts can access
Graph objects as strings containing the Graphviz representation. See
some of the examples in the \verb+models/netcomplexity_scripts+ directory.

\verb+back_inserter+ creates an output iterator suitable for use with
Boost Graph algorithms. As a simple example, to construct an \EcoLab{}
graph from a Boost Graph, do
\begin{verbatim}
  std::pair<BG::edge_iterator,BG::edge_iterator> r=edges(bg);
  ConcreteGraph<DiGraph> g1;
  copy(r.first, r.second, g1.back_inserter(bg));
\end{verbatim}


This interface can be used in both a dynamic polymorphism fashion (ie
Graph is an abstract base class) and in a static polymorphism fashion.

The \verb+graph.h+ header file provides two concrete graph types -
DiGraph\index{DiGraph} and BiDirectionalGraph\index{BiDirectionalGraph}, which differ just in whether each edge
is directed or not.

\subsection{Boost}

\EcoLab{} has a deliberate policy of not having a dependency on
Boost, but to be as compatible with Boost as far as it is
possible. The \EcoLab{} graph library has been designed to be
interoperable with the Boost Graph Library. The file
\verb+test/test_boostgraph.cc+ gives some examples of interoperation
with Boost.

\subsection{Graph Library Functions}

\verb+Degrees degrees(const G& g)+ returns array of node degrees
return as the following structure:
\begin{verbatim}
struct Degrees
{
  array<unsigned> in_degree, out_degree;
};
\end{verbatim}

\verb+void ErdosRenyi_generate(Graph& g, unsigned nodes, unsigned
links, urand& uni);+

Generate an Erd\"os-R\'enyi random graph with given nodes and
links. Links are attached randomly to the nodes, drawn from uni.

\verb+PreferentialAttach_generate(Graph& g, unsigned nodes, urand& uni,
                                 random_gen& indegree_dist=default_indegree)+

Barabasi-Albert Preferential attachment algorithm. For each node, an
indegree value is drawn from \verb+indegree_dist+ (defaults to a
constant value of 1). This many links are then preferentially attached
to other nodes, according to their outdegree. 

\subsection{Network complexity}

\begin{verbatim}
class BitRep implements Graph
{
  BiDirectionalBitRep symmetrise() const;
  bool operator()(i,j);
  bool next_perm();
};

class BiDirectionalBitRep implements Graph
{
  bool operator()(i,j);
  bool next_perm();
};
 
class NautyRep implements Graph
{
  bool operator()(i,j);
  double lnomega() const;
  NautyRep canonicalise() const;
};

\end{verbatim}
These classes implement the Graph interface by storing the linklist as
a bitset. NautyRep specifically uses the bitset representation of the
Nauty package. One can freely convert between these types and others
implementing the Graph interface. All of these types support direct
setting/testing of the i,j th edge through the operator(i,j).

BitRep and BiDirectionalBitRep allow one to iterate through the
permutations (thus cycling over all graph representations of a given
edge count). This function returns false when no further permutations exist.

NautyRep has member functions for calling the Nauty library. If you
need both the lnomega and canonical representations, then it is more
efficient to call them at the same time via the \verb+ecolab_nauty()+
function.

\begin{verbatim}
void ecolab_nauty(const NautyRep& g, NautyRep& canonical, double& lnomega, bool do_canonical);
\end{verbatim}
call Nauty on g, returning a canonical representation canonical and
lnomega ($\ln\Omega$). If canonical is not needed, then set \verb+do_canonical+ to
false.

\begin{verbatim}
double canonical(BitRep& canon, const Graph& g);
\end{verbatim}
SuperNOVA canonical algorithm. Returns $\ln\Omega$ and canonical
representation (which needn't correspond to that returned by Nauty).

\begin{verbatim}
double complexity(const Graph& g);
\end{verbatim}
Network complexity. If all links have the same weight, this
corresponds to $2\lceil\log_2(n+1)\rceil+\lceil\log_2(n(n-1))\rceil+1+
\log_2\left(\begin{array}{c}n(n-1)\\l\end{array}\right) - \log_2\Omega$.

\section{eco\_strstream class}

\subsection{eco\_string class}\index{eco\_string}

\begin{verbatim}
class eco_string
{
public:
  eco_string();
  eco_string(const eco_string& x);
  eco_string(const char *x);
  ~eco_string();

  eco_string& operator=(const eco_string& x);
  eco_string& operator=(const char* x);
  eco_string operator+(const char* y);
  eco_string& operator+=(const char* y); 
  eco_string operator+(const char y); 
  eco_string& operator+=(const char y); 
  eco_string operator+(const eco_string& y);
  eco_string& operator+=(const eco_string& y);
  operator char *() const;
  int length();
  void freeze(int x);
};
\end{verbatim}

Basic string class, broadly compatible with the STL string class, but
can be used in the classdesc\index{classdesc} generated code. The
pointer returned by {\tt char *} refers to an ordinary C string, but
is no longer valid once the destructor is called. This behaviour can
be altered calling {\tt freeze(1)}. Unlike the GNU string class,
casting the string to a {\tt char *} does not freeze the string.

\subsection{eco\_strstream}\index{eco\_strstream}

This is a modified version of ostrstream for \EcoLab{} use, with
slightly altered behaviour making it more suited for \EcoLab{} use.

In particular, \verb+operator<<+ add spaces in between its arguments,
and a new \verb+operator|+ is defined that is similar to ostrstream
\verb+operator<<+. This makes it easier to construct TCL commands.

Note that \verb+|+ has higher precedence than \verb+<<+, so when
mixing the two, ensure that no \verb+<<+ operator appears to the right
of \verb+|+, or use brackets to ensure the correct interpretaion:

eg.
\begin{verbatim}
    s << a << b | c;
\end{verbatim}
or
\begin{verbatim}
    (s | a) << b | c;
\end{verbatim}
but not
\begin{verbatim}
    s | a << b | c;
\end{verbatim}

In any case, you'll most likely get a compiler warning if you do the
wrong thing.

{\tt eco\_strstream} is derived from {\tt eco\_string} class, so has
all the string behaviour. The \verb+str()+ member is defiend as
equivalent to a \verb+(char*)+ cast for compatibility with ostrstream.
However, the contents of the \verb+char*+ are not frozen, unless
explicitly specified with freeze() (opposite to ostrstream behaviour).

\begin{verbatim}
class eco_strstream: public eco_string
{
  public:
    char* str() const;
    
    template<class T>
    eco_strstream& operator<<(const T x); 

    template<class T>
    eco_strstream& operator|(T);
}

inline ostream& operator<<(ostream& x, eco_strstream& y);
\end{verbatim}


\section{cachedDBM}\label{cachedDBM}

{\tt cachedDBM} implements a notion of persistent objects. First and
foremostly, it has the syntax of std::map, ie it is a template map
object, with a key and value type pair. However, by calling the
\verb+init()+ method, you can attach a database file, so that values
saved in the cachedDBM are stored on disk, to be accessible at a later
time.

The iterator range \verb+begin()+ to \verb+end()+ refers to everything
stored in the database. An alternative interface that iterates over
the database keys is provided by \verb+keys.begin()+ and
\verb+keys.end()+ The database is committed when \verb+begin()+ is
called, unless the database was opened readonly. In this latter case,
there are potentially items stored in the map which will not be
iterated over. The alternative iteration
methods \verb+firstkey()+, \verb+nextkey()+ and \verb+eof()+ is an
older interface for iterating over \verb+keys.begin()+ to
\verb+keys.end()+.  In this instance, the cachedDBM is not committed
when \verb+firstkey()+ is called.

Elements stored in the cachedDBM are not actually written to disk until
\verb+commit()+ is called (or the cachedDBM object is
destroyed). 

Entries in the database can be removed via \verb+del()+. However, if
an item with the same key is in the cache, it will need to be removed via
\verb+erase()+ as well, otherwise it will be reinserted in the
database at commit time.

Only a very simple caching algorithm is employed, but it seems
sufficient for many purposes. If the member \verb+max_elem+ is set,
then this acts as an upper limit to the number of items stored in
memory. If you request a new item to be loaded via the [] operator,
and it will cause the number of items to be exceeded, the cachedDBM
object is committed. At least the oldest quarter of the cache is cleared,
and up to half of the cache. So if \verb+max_elem+=100, then one is
guaranteed that the previous 50 accessed objects will always be in
memory, so can be assigned to a reference.

Classdesc serialisation (XDR serialisation) is used store both keys
and value data. The database file are therefore machine independent.
As a special exception to serialisation rules, \verb+char*+ can be
used as key and data types.

\subsection{Synopsis}

\begin{verbatim}
template<class key, class val>
class cachedDBM_base : public base_map<key,val>
{
public:
  int max_elem;   /* limit number of elements to this value */
  void init(const char *fname, char mode='w'); /* open database file */
  void Init(TCL_args args); /* TCL access to init() */
  void close();    /* commit and close database file */
  bool opened();   /* is a database attached? */
  bool key_exists(const key& k); /* does the key exist in db */
  val& operator[] (const key& k); /* access element with key k */
  void commit();    /* write any changes out to the file, and clear cache */
  void del(key k); /* delete key from database (but not cache!) */    
  key firstkey();  /* return first key in database */
  key nextkey();   /* return next key in database in an iteration */ 
  int eof();       /* true if all keys have been accessed */
  KeyValueIterator begin() const;
  KeyValueIterator end() const;
  KeyIterator keys.begin() const;
  KeyIterator keys.end() const;
};
\end{verbatim}

\subsection{TCL access}

cachedDBM objects are addressable from TCL. Individual
objects stored in the cachedDBM are addressable from TCL by means of
the \verb+elem()+ method, but also the following members are
exported to the TCL interface, allowing scripting access to
manipulating the database:

\begin{itemize}
\item \verb+cachedDBM::max_elem+
\item \verb+cachedDBM::Init+ {\em filename} {\em r$|$w}
\item \verb+cachedDBM::close+
\item \verb+cachedDBM::opened+
\item \verb+cachedDBM::elem+ With one argument, get the element value,
  with two arguments, set it.
\item \verb+cachedDBM::commit+
\item \verb+cachedDBM::firstkey+
\item \verb+cachedDBM::nextkey+
\item \verb+cachedDBM::eof+
\end{itemize}

\subsection{Types of database}

By default, if EcoLab's Makefile detects the presence of Berkeley DB on
your system, it will be used. Otherwise it use the ndbm API. If
libgdbm and/or libgdbm\_compat (the NDBM compatibility layer in GDBM)
these will be added to the linker flags, otherwise it will assume that
ndbm is available as part of the standard system library.

You can't mix and match database types. If you have some data stored
in one type, and need to access it using a different database type,
the utilities convtoNDBM and convtoBDB found in the util directory may
be useful.

\bibliographystyle{plain}
\bibliography{rus}
\printindex
\end{document}
