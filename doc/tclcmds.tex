\section{Tcl variables and Commands defined}

\subsection{global variables}\label{globvar}
The following global variables are defined:

\begin{verbatim}
tclvar nsp;
array repro_rate;
sparse_mat interaction;
array mutation;
iarray density;
iarray tags;
\end{verbatim}
The above data types are defined in \S\ref{array} and \S\ref{sparse_mat}.

\verb|nsp| is the number of species registered in the system, \nsp.
\verb|repro_rate|, \verb|interaction| and \verb|mutation| are the
systems parameters \br, \bbeta, \bmu\ respectively. \verb|density|
is \bn. \verb|tags| is used for tracking species through
extinctions, for lifetime and extinction distributions. A species is
given a non-zero tag once it grows past a threshold (10 individuals),
and that tag is not reused once a species goes extinct. See
\S\ref{track_tags} for details.

The parameters may be initialised by assigning values to Tcl arrays
\verb|repro_rate|, \verb|interaction|,  \verb|mutation| and
\verb|density|. Variables of type \verb|array| or \verb|iarray| may be
initialised by assigning the following array elements:
\begin{description}
\item[list] The value of the array element is either a value, which is
copied into each element of the variable, or it is a Tcl list of
exactly \nsp\ elements, once repetitions have been expanded.
Repetition can be expressed by the syntax $n$\verb|*|{\em tcl-list},
where {\em tcl-list} may be a single value, or a list of values which
is repeated. One may also specify a range by means of the syntax {\em
low}:{\em high}[:{\em step}], where the step parameter may be
ommitted.
\item[random] if this array element is true, then the arrays values
are assigned randomly. Sub indices of this are \verb|seed|,
\verb|maxval| and \verb|minval|. The random values are allocated
uniformly in the range \verb|minval|\ldots\verb|maxval|.
\end{description}

Examples of the preceding (assuming \nsp=5):

\begin{tabular}{lcl}
\verb|set density(list) 10| &$\Longrightarrow$& \verb|density| =
\{10,10,10,10,10\}\\
\verb|set density(list) {1,2*{2,3}}| &$\Longrightarrow$& \verb|density| =
\{1,2,3,2,3\}\\
\verb|set repro_rate(list) .1:.5:.1| &$\Longrightarrow$& \verb|repro_rate|
= \{.1,.2,.3,.4,.5\}\\
\verb|set repro_rate(random) true|\\
\verb|set repro_rate(random,seed) 10|&$\Longrightarrow$&
        \verb|repro_rate| is randomly assigned\\
\verb|set repro_rate(random,maxval) .1| & & values in the range $[-.05,.1]$\\
\verb|set repro_rate(random,minval) -.05| 
\end{tabular}

The diagonal part of \verb|interaction| is treated exactly as if it
were an \verb|array|, so \verb|set interaction(diag,list) 0.1| sets
the diagonal elements of \bbeta\ to 0.1.

The offdiagonal elements of \bbeta\ are set by either the
\verb|random| means above, viz.
\begin{verbatim}
set interaction(offdiag,random) true
set interaction(offdiag,random,minval) -0.01
set interaction(offdiag,random,maxval) 0.01
set interaction(offdiag,random,connectivity) 3
set interaction(offdiag,random,seed) 12
\end{verbatim}
where \verb|connectivity| (default \nsp/2) is the average number of
non-zero elements in each row, or by means of the \verb|list|
mechanism. In this case three lists need to be set, \verb|row|, {\tt
col} which contain the indices of the non-zero elements and
\verb|val|, which contains the interaction strength. So the standard
predator-prey equation may be specified by:
\begin{verbatim}
set repro_rate(list) {.1 -.1}
set interaction(diag,list) {-.0001 0}
set interaction(offdiag,val) {-0.001 0.001}
set interaction(offdiag,row) {0 1}
set interaction(offdiag,col) {1 0}
\end{verbatim}

\subsection{generate}

This implements the basic Lotka-Volterra equations:

\begin{math}
\dot{\bn} = \br*\bn + \bn*\bbeta\bn
\end{math}

with \br\ being the reproduction rate and \bbeta\ being the
interspecies interaction. This is implemented as a single line:

\begin{verbatim}
  density += repro_rate * density + (interaction * density) * density;
\end{verbatim}

\subsection{condense}\label{condense}

This command compacts the systems of equations by removing extinct
species where $n_i=0$.

\subsection{mutate}\label{mutate}

This applies the point mutations to the system. In this case, the
number of mutants each species produces is calculated from
(\verb|sp_sep|)\br\bmu\bn. For each of these new species, mutate
calculates a ``genetic drift'' probabilistically according a Poisson
distribution with mean given by \bmu. It then arranges for the values
of \br, \bbeta\ and \bmu\ to be mutated according to a Gaussian
distribution, with standard error given by the ``genetic drift''. The
skeleton of the code follows:

\begin{verbatim}
  /* calculate the number of mutants each species produces */
  new_sp = (double) sp_sep * repro_rate * mutation * density;

  /* generate index list of old species that mutate to the new */
  new_sp = gen_index(new_sp); 

  /* collect the old phenotypes */
  new_repro_rate = repro_rate[new_sp];
  new_mutation = mutation[new_sp];
... and the same for interaction, but messy ...

  /* calculate the genetic distances for the mutants from the parents */
  array gdist(new_sp.size);
  gdist.fillprand();
  gdist *= new_mutation;

  /* 
    change phenotypes randomly, according a normal distribution 
    with std dev gdist 
  */
  gspread( new_repro_rate, gdist );
  gspread( new_interaction, gdist );
  gspread( new_mutation, gdist );

  /* concatonate new species */
  density <<= new_sp;
  repro_rate <<= new_repro_rate;
  interaction <<= new_interaction;
  mutation <<= new_mutation;

  nsp += new_sp.size;
\end{verbatim}

\subsection{Palette}

The palette variable determines the colours of the different graph
lines used in \verb|display| and \verb|plot| and in colouring the
connected webs of species in \verb|connect_plot|. It takes a Tcl list
of standard X-window colour identifiers which is assigned cyclically
to make up a full palette of \nsp\ entries.

Eg.
\begin{verbatim}
set palette {black red green blue cyan magenta yellow}
\end{verbatim}

\subsection{maxeig}\label{maxeig}

\verb|maxeig| returns the maximum eigenvalue of \bbeta. If this number
is negative, the equilibrium point is stable, if positive, it is
unstable. As reported in Standish (1994)\cite{Standish94}, the mutation
drives the maximum eigenvalue slightly positive, then instabilities
act to push the eigenvalue back to zero.

\subsection{ngt10}

Returns the numbers of species greater than 10 (the threshhold used in
\verb|track_tags|.

\subsection{track\_tags}\label{track_tags}

For each species that has more than some threshold number of entities
(hardwired at 10), \verb|track_tags| assigns a unique tag to that
species. If a species has gone extinct, tag 0 is assigned to it.
\verb|track_tags| returns the list with two entries; the first being a
list of tag values corresponding to the species that have exceeded the
threshold for the first time, and the second being a list of tag
values corresponding to species that have gone extinct since the last
call to \verb|track_tags|. Note that \verb|condense| \S\ref{condense}
will delete tags for extinct species to maintain correspondence, and
\verb|mutate| \S\ref{mutate} will assign the tag value 0 to a newly
created species.

\subsection{checkpoint, reload}\label{checkpoint}

Both of these commands take a filename, and implement
checkpoint/restart, for the globval variables defined in \S\ref{globvar}.

