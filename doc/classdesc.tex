\section{Classdesc}

\input{../classdesc/doc/classdesc-common}

\subsection{TCL\_obj}\label{TCL_obj}

The {\tt TCL\_obj}\index{TCL\_obj} function creates a set of TCL commands that
implement get/set functionality for the class members. For example,
with a class definition:
\begin{verbatim}
class foo: public TCL_obj_t {int a, b; void foobar(int,char**)} bar;
\end{verbatim}
\verb+TCL_obj(&bar,"bar",bar);+ creates the TCL commands {\tt bar.a}
and {\tt bar.b}. To set the value of {\tt bar.a}, use the command {\tt
  bar.a} {\em val} from TCL. To get the value, use {\tt [bar.a]}.

Also created is the TCL command {\tt bar.foobar}, which will run
respective member function of {\tt foo} when called from TCL.

Any nonoverloaded member function can be accessed from TCL, provided
the arguments and return types can be converted from/to TCL
objects. In particular, it is not possible at present to call methods
that take nonconstant references.

Overloaded method types in general cannot be called, but it is
possible to create variable length argument lists by declaring a
method with an \verb+(int,char**)+, or a \verb+(TCL_args)+
signature. Such methods are not easily called from C++, and generally,
one needs to define a set of overloaded functions of a different name
(eg capitalised) suitable for calling
from C++, as well as the variable length argument list for use from
TCL. However, as a special case of an accessor (emulating the
setting/getting of an member attribute), one may make use of the
Accessor\index{Accessor} class, which is equally callable from C++ as
TCL.

Accessor is not easily usable from within the C++98 language standard
(see acessor.h in the test directory), but makes much more sense in
the C++11 standard. For example, assume that \verb+Name()+ and
\verb+Name(const string&)+ have been defined as a getter and setter
method of the attribute \verb+name+, then one may define a member
\begin{verbatim}
Accessor<string> name {
    [this](){return this->Name();}, 
    [this](const std::string& x){return this->Name(x);}
};
\end{verbatim}
where the use of lambdas and brace initialisers makes it easy to
assign code for the getter and setter components of the accessor. This
member will be accessible as an attribute from TCL (just as if name
had been defined as a string member), and also callable from C++ as
\verb+name()+ or \verb+name("someName")+ as appropriate.

One downside of the Accessor class is that it is not copy
constructible, as copying the accessor will copy a reference to the
wrong accessed object. Consequently, if copy construction is required
for the object being accessed (eg for DCAS), then a custom copy
constructor needs to be provided.

As an alternative to \verb+(int,char**)+ arguments for implementing
TCL commands, one can also declare the arguments
\verb+(TCL_args)+\index{TCL\_args}. \verb+TCL_args+ variables can be
assigned to numerical variables or strings variables, and the
appropriate argument is popped off the argument stack:
\begin{verbatim}
int x=args;
double y=args;
\end{verbatim}
assigns the first argument to x and the second to y. This method use
the \verb+Tcl_Obj+ structure, so the values needn't be converted to
strings at all.

The arguments may also be assigned using the \verb+>>+ operator:
\begin{verbatim}
int x; double y;
args >> x >> y;
\end{verbatim}

A third style uses the \verb+[]+ operator:
\begin{verbatim}
int x=args[0]; double y=args[1];
\end{verbatim}
The number of remaining arguments is available as
\verb+TCL_args::count+.

If \verb+operator>>(istream,T)+ is defined, then you may also use the
\verb+>>+ operator to set a variable of type \verb+T+, eg:
\begin{verbatim}
void foo::bar(TCL_args args)
{
  iarray x;
  args>>x;
}
\end{verbatim}
the assignment operator cannot be used for this purpose, unlike simple
types, because nonmember assignment operators are disallowed in the
standard. Type conversion operators do not appear to work.

For technical reasons, the name of the TCL command is available as
\verb+args[-1]+. 

You can create TCL\_args objects by using the \verb+<<+ operator. This
enables the calling of methods taking a TCL\_args object from C++,
viz:
\begin{verbatim}
struct Foo
{
  void bar(TCL_args args) {...}
};

Foo().bar(TCL_args()<<1<<3.5);
\end{verbatim}

The {\tt TCL\_obj\_t} data type also implements \hyperref{checkpoint
  and restart functions}{checkpoint and restart
  functions (\S}{)}{checkpoint/restart},\index{checkpoint}\index{restart}
so that any class inheriting from {\tt TCL\_obj\_t}\index{TCL\_obj\_t}
also gains this functionality, as well as \hyperref{client-server
  functionality}{\S(}{)}{get_vars/data_server}.
\index{data\_server}\index{get\_vars}

A helper macro that performs the above is {\tt
  make\_model}\index{make\_model}, which is used in a declarative
sense, which also initialises the checkpoint functor.


Associated with each of these TCL commands, is an object of type
\verb+member_entry<T>+\index{member\_entry}. Pointers to these objects
are stored in a hash table called
\verb+TCL_obj_properties+\index{TCL\_obj\_properties}. The STL hash
table behaved rather stangely when used for this purpose, so a class
wrapper around TCL hash tables was employed instead:
\begin{verbatim}
template<class T>
struct TCL_obj_hash
{
  struct entry 
  {
    entry& operator=(const T& x);
    operator T();
  };
  entry operator[](const char*x);
};
\end{verbatim}
So objects of \verb+member_entry<T>*+ can be inserted into the hash
table as follows:
\begin{verbatim}
member_entry<T>* m; eco_string d;
TCL_obj_properties[d]=m;
\end{verbatim}
but to extract the data, use \verb+memberPtrCasted+
\begin{verbatim}
if (T* m=TCL_obj_properties[d]->memberPtrCasted<T>())
   ... *m 
\end{verbatim}
will allow you to access the TCL object \verb+d+, if it is castable to
an object of type \verb+T+ (is a \verb+T+, or is derived from a \verb+T+). 

A utility macro\index{declare} allows these objects to be accessed simply:
\begin{verse}
{\tt declare}({\em name},{\em typename}, {\em tcl\_name})
\end{verse}
where {\em name} is the name of a variable (in reality a reference),
of type {\em typename} that will refer to the variable having the TCL
handle {\em tcl\_name}. The macro performs error checking to ensure
such a variable actually exists, and that it is of the same type as
{\em typename}.

Objects can be placed into {\tt TCL\_obj\_properties} by a several
different means:
\begin{enumerate}
\item {\tt make\_model}\index{make\_model}{\em (x)}, which places all
  of the leaf objects of {\em x} (which must be derived from
  \verb+TCL_obj_t+) into {\tt TCL\_obj\_properties}, and also
  completes the construction of the \verb+TCL_obj_t+ object;
\item {\tt register}{\em (x)}\index{register}, which places {\em x} into {\tt
    TCL\_obj\_properties}, as well as the leaf objects --- can also be
    called as \verb+TCL_obj_register(+{\em object},{\em object name});
\item {\tt TCLTYPE}{\em (typename)}\index{TCLTYPE}\label{TCLTYPE},
TCLPOLYTYPE(typename, interface), where {\em
    typename} is defined C++ type, and interface is a base class of typename. This creates the TCL command {\em
    typename}, which takes one argument, a variable name for it to be
  referred to from TCL, and creates an object of that type which it
  registers in {\tt TCL\_obj\_properties}. If {\tt TCLPOLYTYPE} is
  used, the base class type is used for registration - so this object
  can be used wherever a polymorphic type with the specified interface
  is expected. For example, consider the
  following code which creates and initialises an object of type
  distrand and gives it the TCL name PDF (from
  testdistrand.tcl\index{testdistrand.tcl}):\index{distrand}
\item {\tt TCLTYPEARGS}{\em (typename)}\index{TCLTYPEARGS}\label{TCLTYPEARGS},
TCLPOLYTYPEARGS(typename, interface), is similar to the above, except
that the constructor needs to be declared with signature
\verb+(TCL_args)+. This enables constructor arguments to be
passed in from TCL.

\begin{verbatim}
distrand PDF
PDF.min -10
PDF.max 10
PDF.nsamp 100
PDF.width 3
PDF.Init dist
.....
PDF.delete
\end{verbatim}
  This macro also defines an x.delete\index{delete} procedure for
  deleting that object, once no longer desired.
\end{enumerate}

A TCL registered object, particularly dynamically created
\verb+TCLTYPE+ objects can be assigned to a member of type
\verb+TCL_obj_ref+\index{TCL\_obj\_ref}. This is particularly useful
for random number generators:

\begin{verbatim}
class Foo: public TCL_obj_t
{
 public:
   TCL_obj_ref<random_gen> rng;

   ...
     rng->rand();
};
\end{verbatim}

Then the member \verb+Foo::rng+ can be assigned an arbitrary random number
generator within the TCL script, such as the PDF example above.

\begin{verbatim}
distrand PDF
PDF.min -10
...
foo.rng PDF
...
\end{verbatim}

Using \verb+TCL_obj_ref+ also allows that object to be serialised, and
to be reconnected after a restart, provided the object has been
created prior to the restart.

\subsection{Member hooks}\label{Member hooks}\index{Member\_entry\_hook}\index{Member\_entry\_thook}

\verb+TCL_obj_t+ has the following two members, that allow one to
assign a hook that is called after every TCL\_obj call into C++
code. This can be used, for example, to record what methods have been
called on the C++ model. There is one for each type of method
signature called from TCL.

\begin{verbatim}
struct TCL_obj_t
{
    typedef void (*Member_entry_hook)(int argc, CONST84 char **argv);
    Member_entry_hook member_entry_hook;
    typedef void (*Member_entry_thook)(int argc, Tcl_Obj *const argv[]);
    Member_entry_thook member_entry_thook;
};
\end{verbatim}

\subsection{TCL\_obj\_stl}\label{TCL_obj_stl}

The header file \verb+TCL_obj_stl+ provides \verb+TCL_obj+ support for
STL containers. If the \verb+value_type+ of an STL container (vector, deque
or list) or set is streamable to an iostream, then it is possible to
directly access the elements of the container as a simple list:

\begin{verbatim}
std::vector<int> vec;
make_model(vec);
   ...
vec {1 2 3}
set vec_elems [vec]
\end{verbatim}
If the \verb+value_type+ is not streamable, an exception will be thrown. This
feature makes the \verb+#members+ functionality of sets redundant.

The following TCL procedures are defined for the
following STL containers, which can be used from a TCL script or the
object browser to manipulate STL container objects. Procedures that do
not call member names are prefixed by the ``@'' symbol, which is a
valid identifier character in TCL, but is not a valid C++ identifier
character. This avoids any possible clash of member names.

\begin{description}
\item[vector, dequeue, list]\mbox{}
  \begin{description}
  \item[@is\_vector] A ``do nothing'' command, the presence of which
    indicates the object is a vector. @elem is more efficient in this case
  \item[@is\_sequence] A ``do nothing'' command, the presence of which
    indicates the object complies with the sequence concept.
  \item[size] returns the size of the vector
  \item[@elem] takes one argument, the {\em index} of an element. It creates a
    TCL command {\em name}({\em index}) that can be used in the usual
    way to access or modify the element's value.
  \end{description}

\item[set, map]\mbox{}
  \begin{description}
  \item[@is\_set] A ``do nothing'' command, the presence of which
    indicates the object complies with the set concept.
  \item[@is\_map] A ``do nothing'' command, the presence of which
    indicates the object complies with the map concept. @elem can be
    used to lookup elements by key.
  \item[size] Return number of entries in the set or map
  \item[count] Takes a single argument, and returns 1 or 0, according to
    whether that argument present within the set or map (as a member or
    key respectively).
  \item[\#members] Returns list of members of a set
  \item[\#keys] Returns list of keys of a map
  \item[@elem] Returns a TCL command name for accessing individual
    elements of a set or map. In the case of a set, the command accesses
  the $i$th element of the set. In the case of a map, the argument can
  be an arbitrary string (so long as it converts to the key type of
  the map), that can be used to address the map element. For example,
  if the map is \verb+map+<string,string>+, one can create an element
  \verb+m["hello"]="foo"+ by means of the following TCL commands:
\begin{verbatim}
m.@elem hello
m(hello) foo
\end{verbatim}
  \end{description}
\end{description}

\subsubsection{Extending TCL\_obj\_stl for nonstandard containers}

\verb+TCL_obj_stl.h+ uses the
\hyperref{\verb+classdesc::is_sequence+}{(see \S}{)}{is_sequence} and
\hyperref{\verb+classdesc::is_associative_container+}{}{}{is_associative_container}
type trait. This means that to enable TCL\_obj handling of your custom
container \verb+MyContainer+, you need to define the appropriate type
trait prior to including \verb+TCL_obj_stl.h+, eg:
\begin{verbatim}
namespace classdesc
{
  template <> struct is_sequence<MyContainer>: public true_type {};
}
\end{verbatim}


Also, if your custom container is more of a map, then you also need to
define the \verb+ecolab::is_map+\index{ecolab::is\_map} type trait for
it's \verb+value_type+, eg
\begin{verbatim}
namespace classdesc
{
  template <> struct is_associative<MyContainer>: public true_type {};
}
namespace ecolab
{
  template <> struct is_map<MyContainer::value_type>: 
    public is_map<std::pair<MyContainer::key_type, MyContainer::mapped_type> > {};
}
\end{verbatim}


